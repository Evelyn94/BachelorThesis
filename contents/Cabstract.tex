% !Mode:: "TeX:UTF-8"

\begin{Cabstract}{ADS-B}{数据链隐私保护}{格式保护算法}{FFX算法}{ADS-B解码}
鉴于ADS-B系统相对于现有的雷达航空监视系统的优越性,从雷达监视系统过渡到ADS-B系统是一个大势所趋的过程。根据美国联邦航空管理局(Federal Aviation Administration)关于下一代空中交通系统的实施规定,到2020年,所有经过美国的飞机必须配备ADS-B设备。由于ADS-B系统设计之初是没有考虑安全隐私方面的问题,因此,关于ADS-B安全隐私方面的研究也变得越来越迫切。目前的ADS-B系统可能面临的攻击大体可分为主动攻击和被动攻击,本课题主要解决被动攻击中的隐私保护问题。本课题采取的隐私保护的方法是对ADS-B数据进行加密的方法。说到加密第一个面临的问题就是算法的选择,ADS-B消息的加密算法需要满足符合ADS-B系统和ADS-B消息格式的特性。因此,针对这些特性和对各种算法的综合对比分析选出了认为是最适合的算法即格式保护算法中的FFX算法。根据ADS-B数据链的大小不可变特性,消息加密前和加密后的长度不能改变,且不能附加任何额外信息,故庞大臃肿的非对称性加密算法相对对称加密算法则没有优势。因此FFX算法是一种对称的具有格式保护特性的静态算法。对ADS-B消息进行加密还面临到的一个问题是ADS-B消息的哪些部分是可以加密的,哪些部分不能加密否则会导致无法解码等异常情况的出现。

经过仔细分析和实验,若要满足使飞行器对无关人员匿名以达到隐私保护的目的的话只对ADS-B消息中的ICAO24部分进行加密即可,对这部分进行加密即不会影响对ADS-B消息的解码也不会隐藏轨迹从而破坏航行安全性,且加解密的效率很高,因此这种方案可以投入到实际应用中。最后,为了对加解密的结果进行检验,本系统还实现了对ADS-B消息进行解码和可视化,使ADS-B消息包含的位置和飞行唯一标识信息等在地图上以直观的方式展现出来。
\end{Cabstract}
