% !Mode:: "TeX:UTF-8"

\chapter{结束语}

\section{全文总结}
本文中阐述了 1090ES 广播式自动相关监视系统的报文接收与解析过程,主要完成了以下工作:\newline 
1.阐述本文的研究背景,分析了技术发展的国内外现状。\newline 
通过介绍航空监视系统的进化历程分析了新一代航空监视系统较上一代的航空监视系统的优势及其更替的必然性以及课题研究的意义。在分析和比较后,了解到我国的民用航空业目前正处于成长期,飞机的数量、飞行的规模和空域的范围都在不断扩大,需要引进并吸收 ADS-B 技术,以丰富和改进我们自己的空中交通管制系统和体系,并实现与国际最新技术的接轨。国内外在ADS-B隐私的保护上都有一定程度上的研究,针对ADS-B消息加密这种隐私保护方式需要更多考虑的问题是如何让加密算法更适合ADS-B的实际应用而不是仅仅停留在理论研究上和加密算法的时效性对于ADS-B应用来说也是非常重要的因为消息发送或接受的延误会影响ADS-B的性能,从而降低该航空监视系统的安全性。 ADS-B 技术的吸收和应用中,需要不断地研究其基本原理,掌握其技术核心,促进实现国际先进技术与中国本地情况的不断融合。\newline 
2.对本课题中涉及到的相关技术进行深入的研究。\newline
研究了系统中用到的一些基础技术,包括 ADS-B 技术、加密技术以及国外开源ADS-B消息获取网站OpenSky。在 ADS-B 技术中,重点研究了 ADS-B 的工作原理和 ADS-B 数据链和ADS-B消息格式。在ADS-B数据链技术的介绍中分析和对比了三种不同的ADS-B数据链技术,分别为MODE S 1090 ES、UAT和VDL MODE4,本课题的系统模型是建立在MODE S 1090 ES数据链收发方式上的。本课题实现的ADS-B隐私保护系统所需要的原始数据包括ADS-B原始消息,消息的发送和接收时间,接收地点经纬度等一系列信息以及处理这些信息的一些辅助函数。\newline 
3.实现了报文加解密、解码及可视化处理。\newline 
首先,介绍各个部分的代码实现算使用的语言,FFX加/解密算法是使用python2实现的,因为python下对数学编程的包较全面和简单易用故选择用python实现该算法。解码以及可视化处理均是使用java实现的,因为OpenSky网站上提供的原始数据为.avro文件格式,该文件格式是使用Hadoop大数据平台工具进行操作的文件,这些数据操作工具以及函数包只能再Linux平台下使用而Java具有跨平台性好的特点而且OpenSky网站上提供的一些辅助函数均是使用Java实现的,所以本课题实现中的解码及可视化部分代码均采用Java实现。\newline
第二,对.avro文件中提取出的ADS-B消息进行加/解密处理,采用之前分析过的适用于ADS-B应用实际的加解密算法FFX加/解密算法,在FFX算法中需要用户输入密钥参数。然后将经过加/解密的ADS-B消息输出到解码模块。该算法为静态算法,即密钥和其他参数一定的情况下对同一消息进行加/解密的结果是一样的。\newline
第三,对经过加/解密的ADS-B消息进行解码处理。解码所依据的原理是ADS-B的消息格式,即ADS-B消息的各个域所代表的编码含义以及各个域为不同值是所代表的编码信息,根据ADS-B某些域的的值的范围分别根据其编码规则进行合适的编码。然后把解码后得到的消息分别放到相应的类的具体对象中。\newline
第四,对ADS-B解码后得到的信息进行可视化处理,即把它的轨迹、ICAO24和时间等信息展示在Google Map上。要让位置信息和其他标志信息在Google Earth 软件中显示,要先创建.kml文件的框架描述然后建立.kml文件再把这些ADS-B解码后得到的具体信息放在.kml文件中。
最后,描述了关键功能的代码实现,包括加/解密算法、位置解码和.kml文件转换。

\section{后续工作展望}
一、密钥更新:\newline
由于两次密钥相同的加密会产生相同的结果,故此假名多次使用后会增大假名与真实飞机身份的相关性。故需找到最合适的更新周期从而为密钥制定合适的周期更新方案,从而减少飞机和假名之间的相关性,从而极大的增加攻击者通过ADS-B消息跟踪飞机的难度。

二、用随机静默期的方法解决ADS-B隐私保护问题:\newline
ADS-B的隐患有很多,包括主动攻击和被动攻击,目前也提出了针对各种攻击的具体解决方案。关于ADS-B被动攻击中的ADS-B隐私保护的方法不仅有数据链加密这种形式,目前提出的解决方案中较可靠的还有一种叫随机静默期的方法,这种方法的主要工作原理是让ADS-B OUT每隔一段时间就短暂的不播报ADS-B消息,这段不播报ADS-B消息的间期称为随机静默期,通过在随机静默期不播报消息的方式达到一种把飞行器与周围的飞行器混淆的效果。故接下来可以尝试研究用随机静默期这种方式具体的解决ADS-B隐私的保护问题并对比这两种问题的长处和短处。
\section{收获及体会}
国际民航组织(ICAO)早就有意发展 ADS-B 系统,在欧洲和北美对该系统的试验已进行了多年,并已逐渐进入实际应用阶段。在美国阿拉斯加境内和东、西海岸部分地区,ADS-B 系统已进入了实际的应用阶段并取得了丰富的实践经验。我国对 ADS-B 技术的应用基本还是空白。目前除美国以外我国是第一家把UAT 模式的 ADS-B 技术应用于通用航空领域的国家。在我国低雷达覆盖的情况下,ADS-B 用于民航的空中交通管制是非常必要的,特别适用于没有二次雷达覆盖的地区。该系统能实现空中交通管制的自动监控、提高地面指挥的管理水平、工作效率和飞机飞行的安全性,并能节约建设二次雷达系统所需的大量资金,可在短期内使无雷达覆盖区域空中交通管理水平有质的飞跃和提高。ADS-B 技术的进一步推广,即使雷达不能覆盖,也能使民用航空更安全,为我国进一步开放空域,发展民航事业,提供了坚强的物质保证。在ADS-B应用的大前提下,保证ADS-B系统的安全性也变得尤其重要。
在ADS-B隐私保护系统的实现中一个很重要的收获是任何针对实际应用的研究都要紧贴现实不能凭空想象,否则研究出来的东西就会缺乏实际应用价值,例如目前已经存在无数的加密算法,但是要针对ADS-B系统以及ADS-B消息的特点和格式进行加解密算法的选择,否则算出来的算法将没有实际应用意义。



