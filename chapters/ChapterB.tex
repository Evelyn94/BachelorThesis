% !Mode:: "TeX:UTF-8"

\chapter{相关概念和技术}
\section{ADS-B}
\subsection{ADS-B系统的组成和运行}
广播式自动相关监视(ADS-B, Automatic Dependent Surveillance Broadcast)是下一代空中交通管理的关键组成部分。现在使用的航空交通管理系统应用计算机和地面雷达实现飞机追踪和航空管理。虽然目前的航空系统表现的很好,但是它很容易受到干扰(例如天气)导致长时间的延误,更重要的是,这个航空运输管理系统已经快要达到它的工作处理速度的极限了。没有一个运输系统,它的航空交通预期增长将可能导致损失性巨大的航班延误和增加飞行危险系数的。因此,作为应对这个日益增长的担忧,美国国会建立了计划和发展工作小组来管理下一代航空管理系统的升级。ADS-B是自动的因为它不需要飞行员或其他控制者的干预就能自行广播消息。它是独立的监视系统因为飞行器需要依赖全球导航卫星系统(GNSS, Global Navigation Satellite System)来获取自身的的位置信息。更重要的是,它连续像附近的地面基站、附近的飞行器和地面交通工具(例如滑行中的飞行器)广播飞行器的位置信息和其他信息。其中需要涉及到标准信息格式和传输协议。ADS-B消息计划是在目前支持S 模式的1090MHz的数据链路中传输。为了支持ADS-B, S 模式的传播器将会结合一种叫拓展性应答机随机自发报告的特性。拓展性应答机随机自发报告提供了一个从传统S模式的升级和保证了在升级传输期间与现存系统的无缝整合。

ADS-B能提供连续的飞行器位置、身份、速度和其他信息的连续广播\cite{mccallie2011security}。ADS-B设计之初并没有考虑安全问题,即没有安全机制去保护飞行器与航空管理者之间的消息传递的机密性、完整性和可靠性。因此,一个被动攻击者可以随意接收ADS-B消息并解码以获得飞行器的各种隐私信息以获取相关的利益。一个主动攻击者可以自行编码ADS-B消息并将其广播出去以制造并不存在的假飞行器又或者制造泛洪攻击使真正的ADS-B实时消息得不到接收。历史事件已经证明了不加密的数据链路会被有动机的对手利用。

ADS-B有两个功能性操作:(i) ADS-B OUT 和 (ii) ADS-B IN。ADS-B OUT 包含让飞行器或地面交通工具连续的产生ADS-B广播的功能,这个功能向航空控制系统提供了实时位置信息。ADS-B IN 具有接收和显示接收到的来自另一架飞行器的ADS-B OUT 消息,ADS-B IN 也允许飞行器接收地面控制中心提供的服务(例如天气预报更新)。需要注意的是,飞行器可以只装配ADS-B OUT 不装配ADS-B IN,这丝毫不会影响ADS-B OUT的工作。

\pic[htbp]{ADS-B系统的功能性组件及运行模型}{width=0.8\textwidth}{p1}

图2-1介绍了ADS-B系统的功能性组件和模块以及提供了ADS-B运行的总的概括图。ADS-B数据在三个主要组件中交换,这三个组件分别为发送飞行器,接收飞行器和地面基站。\newline
发送飞行器:一架装有ADS-B OUT的飞行器通过ADS-B OUT中的GPS 接收器从GNSS中获取自己的位置信息,随后把位置连同身份信息、海拔高度、速度(包括速率大小和速度方向)通过飞机上的S模式的传播器(transponder)在1090 ES数据链路中广播出去。\newline
接收飞行器:另一架配备有ADS-B IN的在其附近的飞行器通过ADS-B IN组件中的ADS-B 接收器接收到第一架飞行器发送的ADS-B消息并把它传输到数据处理组件(TCAS),在数据处理组件中对ADS-B 消息进行解码和可视化处理则可看到第一架飞行器所发送的ADS-B消息所包含的内容,即轨迹或者飞行器标识符等信息。\newline
地面基站:同理,地面基站也配备有ADS-B接收器,同样可从1090ES数据链路中接收到ADS-B消息并同时把消息传输给空中交通控制系统和将接收到的消息广播出去。
\subsection{ADS-B 数据链}
ADS-B系统主要包括数据链、地面站、ATC系统、机载设备等。其中数据链是ADS-B技术非常重要的一个部分。实现 ADS-B 数据传输的技术称为 ADS-B 数据链,可以将数据链看作是一个专用的网络系统,包含了相关的通信协议、数据链设备和通信终端\cite{he2010adsb}。最初用于军事领域,目前已经逐步向民用领域扩展。目前,在 ADS-B 中使用的数据链一共有三种MODE S 1090 ES、UAT、VDL MODE 4\cite{shi2007adsb}。
\subsubsection{MODE S 1090 ES}
1090 ES (Extended Squitter,扩展电文)的命名中包含两层含义。1090 指的是该数据链的下行传输频带是 1090MHz。ES 指的是对原有 ADS-B 报文长度的扩展。原有的报文长度一般为 56-112 比特。1090 ES 是一种S模式的数据链,支持一对一的询问-应答机制\cite{yang2014adsb}。S是 Selective 的意思,指的是可以对航空器进行选择通信。1090 ES 采用 PPM(脉冲位置调制)编码。它的使用频1090MHz,信息格式是简单的脉冲位置编码,数据传输速率为1Mb/s。1090 ES用发射机和发射天线来传送不同的消息,包括24比特码、高度、呼号等。由于消息格式简单,承载信息能力较弱,所以在一个编码中只能传输一个特定类型的信息。而这些消息的更新率也有所不同,位置消息和速度消息每0.4-0.6s更新一次,标识消息和类型消息每4.8-5.2s更新一次,航路点消息每1.6-1.8s更新一次。 1090ES采用扩展型断续振荡的方式,由112个信息脉冲构成的S模式ADS-B长应答信号通过机载设备每隔1s广播一次。112位信息脉冲串的前88位为消息位、后24位为奇偶校验位。具体信息内容包括经度、纬度、方位和速度等信息,如图2-2所示。

\pic[htbp]{1090ES的消息格式}{width=0.8\textwidth}{p2}

1090  ES 是由 ICAO 推荐的、唯一一个可以在全球范围内使用的数据链技术,得到了美国、欧洲、亚洲等大部分国家的承认和应用。在第11次国际航空会议上,已经将1090 ES 作为 ADS-B 主要数据链技术,并制定了相关的协议和标准\cite{yang2011jiyu}。
\subsubsection{UAT}
UAT 的全称是 Universal Access Transceiver,即通用访问收发机。UAT 是美国专门为 ADS-B 设计的数据链,它的特点是成本低、使用方便、通用性强、具有地面站向航空器上行广播数据的能力、传输带宽较大\cite{li2012jiyu}。 UAT 采用时分复用的方式传递数据,基本传输单元为 UAT 帧,每 1 秒钟为一帧。每一个 UTC 秒开始一帧数据,每一帧传输分为两个部分,前 188 毫秒由地面站发送上行数据,后 812 毫秒由机载 ADS-B OUT 设备下行发送 ADS-B 报文\cite{lv2007adsb}。 UAT 地面站上行传输采用固定时隙接入方式,机载 ADS-B OUT 设备下行传输采用随机接入方式。传输速率为 1Mbps,工作频带为 978MHz。UAT的最小时间度量单位是MSO,每个MSO时长为250ms,一帧共4000个MSO。如图2-3所示。 地面站最小信息传输单位是时隙,一个时隙有22个MSO。因此第一段由32个时隙构成,每个地面站被分配一个时隙。第二段被飞机和地面车辆所共享,每个飞机/车辆在ADS-B段中随机选择MSO进行传送。
 
\pic[htbp]{UAT的帧格式}{width=0.8\textwidth}{p3}

\subsubsection{VDL MODE4}
VDL MODE4又称为 VDL-4,全称是 VHF Data Link-Mode 4,是基于 STDMA(Self-Organized Time-Division Multiple Access,自组织时分多址)的 ADS-B 数据链技术\cite{ding2010jiyu}。它以 ADS-B 应用为主,同时支持 TIS-B 和 FIS-B。 VDL-4 由 ICAO 和 ETSI(European Telecommunications Standards Institute,欧洲电信标准协会)共同推荐的数据链技术,有相应的 OSI 标准。 VDL-4 利用 GNSS 进行定位和严格的时间同步,具有 2 个全领标示信道和一个附加信道,可以高效传输重复性信息,支持各种实时应用\cite{qiu2011adsb}。 无需地面系统的支持,VDL-4便可建立空空通信或空地通信,使用VIP(VDL Mode-4 Interface Protocol,VDL 模式 4 接口协议)进行工作。 VDL-4 的帧时间长度达到了 60 秒,分为 4500 个等长时隙,每个时隙 13.33 毫秒。传输速率为 19.2kbps 。 VDL MODE 4采用自组织式时分多路(STDMA)和超长帧技术,参与通信的所有设备都由 GNSS 进行严格的时间同步,STDMA中使用者的共同时间标准是UTC。从而避免了数据发送的时隙冲突,并可以实现数据发送的计划性。
表2-1是对三种数据链技术的比较\cite{su2007sanzhong}。 

\pictable[h]{三种数据链技术比较}{width=\textwidth}{t1}

这三种数据链都能满足当前ADS-B应用的基本要求,但都不甚完美。由于欧洲和美国两大商用飞机制造基地的产品生产标准不同,在选用地空数据链时,出于兼容现有机载设备、兼顾终极发展目标的考虑,政策取向也各有侧重。 根据民航总局关于ADS-B的技术政策,并考虑到我国未来空管系统与国际接轨问题及在全球范围内的相互操作性(目前只有S模式1090  ES数据链技术是被各个国家、地区和组织所接受的标准),我国在实施ADS-B项目计划时优先考虑使用1090 ES作为数据链路技术。同时,考虑UAT机载设备和地面站的性价比、功能特性和适用范围,在通用航空飞行活动频繁的特殊区域可以考虑采用UAT作为支持ADS-B数据链技术,暂不考虑采用VDL MODE 4作为我国ADS-B系统的数据链路。


\subsection{ADS-B消息格式}
图2-4所示,ADS-B消息由112 bits 组成并分成5个区。

\pic[htbp]{ADS-B 1090ES 消息格式}{0.8\textwidth}{p4}

下面将对 ADS-B 报文结构的基本字段进行进一步的说明。\newline 
1.DF(Downlink Format)字段\newline 
DF 字段长度是 5 位,用于区分不同的下行链路格式(Downlink Format)。DF 的值可以是 17、18 或 19。 
DF=17 用于 S 模式应答机发出 ADS-B 报文。 
DF=18 用于非 S 模式应答机发出 ADS-B 报文或 TIS-B 报文。 
DF=19 用于军事用途,非军事应用不会涉及到该类型报文。 
2.CA/CF(Capability/Code Format)字段\newline 
CA/CF 字段的长度是 3 位,在不同的 DF 值下有不同的含义。 
DF=17 时,该字段是 CA 字段,含义是 S 模式应答机的能力。 
DF=18 时,该字段是 CF 字段,含义是编码格式(Code Format),用于区分 ME 字段的内容、AA 地址的类型、以及两类特殊的报文。CF=0 或 1 时,表明该报文是 ADS-B报文。 
3.AA(ICAO24 Aircraft Address)字段\newline 
AA 字段的长度是 24 位,包含了发射装置的地址信息。地址的类型有两类:ICAO 地址和非 ICAO 地址。ICAO 地址是飞机的地址,非ICAO 地址是匿名地址、地面车辆地址或表面障碍物地址。 
4.DATA(Data frame)字段\newline 
DATA 字段的长度是 56 位,包含了 ADS-B 报文的业务数据,称之为 ADS-B 业务报文。关于 DATA 字段的格式将在 ADS-B 业务报文中进一步说明。 
5.PI(Parity Check)字段\newline 
PI字段的长度是 24 位,是一个下行链路字段,含义是奇偶性(Parity)和一致性(Identity)。该字段包含了编码标签(Code Label,CL)和询问器编码(Interrogator Code,IC)的奇偶性。在 ADS-B 报文中,CL 和 IC 的值都是 0,也就是说 PI 字段的内容填充
了 24 个 0。\newline 
因此,可以总结出本系统应接收并处理的 ADS-B 报文格式为:DF=17。

下面具体介绍Data frame区域的各个字段的含义:
如图2-5所示,Data frame(33bit-88bit)中的33bit-37bit为TC(Type Code),Type Code的值不同,则这条ADS-B消息所包含的消息也不同。如图2-6所示,当Type Code 为不同值时ADS-B消息所包含的不同的信息。

\pic[htbp]{Type Code位置}{0.8\textwidth}{p5}

\pic[htbp]{Type Code值与ADS-B消息含义的对应}{0.8\textwidth}{p6}
\section{FPE}
根据已有的研究成果,刘哲理等人\cite{liu2012ge}从两个方面对FPE进行了定义,包括基本FPE和一般化FPE。基本FPE强调密文和明文具有相同的格式,即确保密文和明文属于相同的消息间;一般化FPE则强调待加密消息空间的复杂性决定FPE问题的复杂性。

定义1 基本FPE:FPE可简单描述为一个密码: 
$E \colon K \times X \to X$                   (1)
其中K 为密钥空间,X 为消息空间\cite{bellare2010ffx}。 
基本FPE描述了FPE要解决的问题,即密文与明文处于相同的消息空间。例如,对n位的信用卡号进行保留格式加密,要求密文与明文的格式相同,都是n位十进制数字组成的字符串。即明文与密文都处于消息空间$/{0,1,...,9/}^n$内。根据基本FPE的定义,分组密码也是某种形式的FPE,分组密码是字符串集合$/{0,1/}^n$上的置换。但是,FPE的消息空间的复杂性要远远高于分组密码,如日期型的消息空间“YYYY-MM-DD ”,其格式不仅有长度为10的限制,而且还有特定位为‘-’,且年、月、日必须在合法范围内等特定格式的要求。 
为了更完整地描述FPE问题,定义集合$\Omega$为格式空间,任意一个给定的格式$\omega \in \Omega$ ,可确定一个由$\omega$确定的消息空间 X 的子空间 $X_\omega$,FPE与集合{$X_\omega$} ,$\omega \in \Omega$有关。 $X_\omega$称为由格式$\omega$ 确定的待加密消息空间 X 上的一个分片,每个分片 $X_\omega$都是一个有限集。给定密钥k ,格式$\omega$ 和调整因子t,FPE就是一个定义在 $X_\omega$上的置换$E_K ^\omega,t$。 

定义2 一般化FPE:一般化FPE可描述为一个密码: 
$E \colon K \times \Omega \times T \times X \to X \cup {\perp}$                   (2) 
其中K 为密钥空间,$\Omega$为格式空间,T 为调整因子空间, X 为消息空间。所有空间非空且 $\perp\notinX$\cite{bellare2009format}。 
可通过算法三元组
$E_FPE$= ( setup , encryption, decryption)来描述一般化 FPE ,
其中: 
算法setup: 
初始化系统参数 params 。不同的FPE模型需要初始化不同的参数,通常包括以下3个部分: 
(1)初始化对称密码算法(要求对称密码足够安全)的参数,比如Feistel网络的分组长度、轮函数和轮次数等; 
(2)初始化待解决的问题域,包括需要保留的格式$\omega$ 以及由$\omega$确定的消息空间分片 $X_\omega$; 
(3)初始化加(解)密使用的对称密钥k ,要求安全存储该密钥,不对外公开。 
算法encryption: 
输入明文x和调整因子$t\inT$,返回消息空间分片$ X_\omega$内的密文 y 或者$\perp$。该算法执行$E_K^\Omega,T(X)=E(K,\Omega,T,X),E_K \Omega,T(.)$是 $X_\Omega$上的一个置换。如果$x\in X_\omega$,则返回$y=E_K ^\omega,t(x)$,否则返回$\perp$。 

算法decryption: 
输入密文y和调整因子$t \in T$,返回相同消息空间分片 $X_\omega$内的明文x或者$\perp$。该算法是encryption的逆运算:如果 $y \in X_\omega$,则返回
$x=D_t^\omega$,t(y),否则返回$\perp$。

\subsection{加密算法选择上的分析}
根据选题‘ADS-B数据链隐私保护算法的设计与实现’,本文只考虑被动攻击的防范问题即ADS-B信息隐私的保护问题。对于公共航班和货机,由于它们具有固定时刻表、公开的航线和永久性的身份标识符,所以并不存在位置隐私的保护需求。本文主要考虑的是航班以外的民航专用航空(general aviation)的位置隐私安全的保护问题。专用航空包括很大范围的飞行器,包括私人飞机、执法和紧急服务飞机和商务飞机。

本文采用加密ADS-B报文的方式来保护飞机的位置隐私。加密算法大体分为两种:对称加密算法和非对称加密算法。对称加密算法使用相同的密钥进行加密和解密。非对称加密算法使用唯一的公私钥对分别进行加密和解密。下面分析一下这两算不同的算法在ADS-B数据链加密应用上的好处与瓶颈。

公共密钥基础设施(The public key infrastructure ,PKI) 使用的就是非对称算法。PKI常常用于保证未知结构网络的安全性,但是关联节点间的身份标识符是预先决定并持续不变的。而且PKI不仅能提供机密性服务也能提供认证性服务。但是PKI需要一个可信的第三方来对密钥对的产生和认证进行统一的管理。而且公钥算法要求使用目标接收方的公钥进行加密,如果这个消息需要同时发送给很多不同的人,那么这条消息则需要被用不同的公钥加密并分别发送给对应的接收方。而且ADS-B消息的长度固定且很小,要装下一个庞大的证书在具体实践上是几乎不可能的。虽然飞机能够知道它附近有哪些飞机和ATC,但是这种广播式的逐一发送效率太低的缺陷足以掩盖ADS-B相对于过时的方案PSR的优点,且不说全球的飞行器的密钥对存储和管理的问题。

对称密钥算法相对公钥算法来说效率更高,规模性也更能得到保证因为只有一个密钥被同时用于加密和解密。关于密钥的分发和管理,可以使用飞行员控制数据链路通信(the controller pilot data link communication ,CPDLC)来初次协商或者更新密钥。在初始化飞行计划期间飞机会和ATC进行一起身份确认以进行同步。因此,密钥交换可以包含在CPDLC登录中,这为密钥的安全交换提供了一个机会。至于密钥的管理,可以由ATC控制员随机产生并通过CPDLC传输给每个相关飞行员。由于ADS-B广播消息的日期信息只有很小的改变,为了保护系统免受已知明文攻击,块加密(e.g., AES and 3DES)比流加密更适用于ADS-B的实际操作环境。下行链路格式区和CA字段固定为8 bits, 用于向接收方表明消息的处理方式,不能加密。这意味着只能加密剩下的104 bits。但是目前的快加密算法一般以64 或128 bits 分块。通常情况下如果消息不能满足分块大小的话可以通过填充的方式来解决,但是对于ADS-B消息而言不行,因为下层的ADS-B广播框架已经设计成112 bits的固定数据长度,因此需要一个能支持任意块大小的加密算法。

\subsection{FFX}
Bellare等人在2010年提出了 FFX 模型[34]\cite{black2002ciphers},该模型对FFSEM 模型在消息空间、Feistel 网络等方面进行了扩展。FFX 模型使用了非平衡Feistel 网络,能够处理消息空间$Char^n$上的FPE问题,$Char^n$为长度(字符个数)为 n 的字符串构成的集合。 
FFX 模型通过将固定字符表与其索引表(数字集合)建立双射关系,将字符串中的每个字符编码为数字串,对数字串进行Feistel 加密运算,从而实现对消息空间$Char^n$上的保留格式加密。
格式保护加密不仅应该做到传统块加密(例如AES)能够做到的事,如使输入明文如输出密文的大小一样,而且还能比传统加密更加普遍化,而不是仅仅能加密大小为64bits 或128bits的信息(例如AES)。
实现格式保护加密的一种方式使使用Feistel-based模式的操作。作为基础的轮功能是基于传统的块加密算法,例如AES。两种实现了FPE要求的模式分别是FFX和BPS。但是FFX比BPS在模式上要更加的开放和更实用,因此本文选择的是FFX模式的FPE算法。
FFX又分为A2和A10两种模式,FFX-A2模式用于加密8-128bits的二进制比特串而FFX-A10用于加密4-36bits的十进制数字。对于长度充足的字符串,两种模式都使用12轮的Feistel。但是轮数会随着消息变短而迅速增长,对于允许的最短消息长度FFX-A10可以达到24轮或者FFX-A2可以达到36轮。
FFX[radix]模式。这是一种新的FFX参数收集方式。首先,这样做拓展了允许的radix值,任何可行的radix值都可以被使用,而不仅仅是2或者10。其次,还扩大了允许的加密消息长度,允许有效的任意长度字符串被加密。最后,在轮数的选择上FFX[radix]模式比FFX-A2或FFX-A10要更进步一些,FFX[radix]的轮数是常量而不是随着消息的长度变化而变化。Radix 的可选值介于2到$2^{16}$之间。Radix的数值等于被加密的字符串的字符种类的个数。例如,被加密字符串是一个二进制比特串的话,radix就等于2,因为比特串里只有0或者1两种类型的字符。FFX[radix]是使用基于AES的Feistel模型。FFX[radix]有效的统一和拓展了FFX-A2和FFX-A10模式。所以即使加密轮数一样,FFX[2]与FFX[10]也和FFX-A2和FFX-A10不一样。
块加密加功能$\boxplus$由$a_1\cdots a_n $\boxplus$ b_1\cdots b_n = c_1\cdots c_n$定义,$c_1\cdots c_n$是唯一满足\newline$\sum c_i [radix]^{(n-i)}=(\sum a_i [radix]^{(n-i)}+ \sum b_i [radix]^{(n-i)})$的字符串。
块加密$\boxminus$功能的运算特性是当X$\boxminus$Y=Z,则Y$\boxplus$Z=X。
$[s]^i$表示s是由i个字节编码的字符串,例如i=1,则s是由一个字节即四个比特编码的字符串,即s是一个16进制的数字。

\pic[htbp]{FFX Encrypt&Decrypt}{width=0.8\textwidth}{p7}

\pic[htbp]{F函数}{width=0.8\textwidth}{p8}

\pic[htbp]{FFX加密算法参数}{width=0.8\textwidth}{p9}

\section{OpenSky}
Opensky(https://opensky-network.org)是一个专门的在它的感知范围内收集ADS-B数据并用于飞行研究的感知网络。目前的传感器部署范围是欧洲中部,如图2-10所示:

\pic[htbp]{OpenSky传感器分布图}{0.8\textwidth}{p10}

这些传感器由志愿者们提供和部署,然后志愿者们将他们的信息发送到一个数据处理中心,在那里所有的数据被解码、评估和存储在一个大型数据库里。这些数据不仅包括ADS-B原始数据,还包括时间戳、传感器位置等可用于深层次研究的数据。它的部件组成细节描述如图2-11:

\pic[htbp]{OpenSky数据处理中心组成部件}{0.8\textwidth}{p11}
\section{本章小结}
本章对该课题接下来章节中需要用的的概念和技术做了详细的讲解。首先介绍了ADS-B系统的概念、组成、运行和设计,ADS-B系统的网络和ADS-B消息的格式,使读者对本课题的主要研究对象ADS-B有系统和深入的认识。然后介绍了格式保持加密这种算法,并具体分析了为什么要选择这种算法,也介绍了所选择的格式保护算法的具体的某一种模式(FFX)。最后,向读者介绍一个对本次毕业设计提供了重要帮助的一个非常实用的公益研究性质的网站。该网站为本次课题提供了研究所必须的ADS-B原始数据以及对处理这些数据很实用的一些具体函数的实现。
