% !Mode:: "TeX:UTF-8"

\chapter{设计与实现}

\section{设计模型}

\subsection{系统模型}
本系统假设的应用模型如图一所示,GNSS把飞机的位置信息发送给飞机的GPS Receiver,再由ADS-B Transmitter将飞机自身的身份标识符、3D位置、速度、飞行目的地和其他空间信息编码成ADS-B消息并对其进行加密然后以一定的周期广播出去。地面上的空中交通控制中心(Air Traffic Control, ATC) 和其他附近一定范围内飞机的ADS-B Receiver都能接收到ADS-B信息并对其进行处理或广播。只有预先与发送ADS-B消息飞行器进行密钥协商的的其他飞行器或地面基站的ADS-B IN才能解码出真实的发送方的ICAO24。
本课题由于只研究ADS-B的数据链隐私保护即ADS-B消息的加解密,故并没有模拟ADS-B消息的发送和接收过程。加密过程相当于在ADS-B OUT 中实现,解密过程和解码显示过程相当于在ADS-B IN中实现。

\subsection{网络模型}
本课题的网络模型设计两种通信模式,其中GPS与ADS-B Out的GPS Receiver之间通过GNSS数据链路通信,而ADS-B Out、ADS-B In 和ADS-B Ground Stations 之间通过MODE S 1090 ES数据链路通信。本课题重点讨论ADS-B Out、ADS-B In 和ADS-B Ground Stations 之间的通信模式。在本课题中这三者之间通信采用的是MODE S 1090 ES 的ADS-B数据链技术。
1090MHz拓展性应答机随机自发报告消息是为商用航空设计的而且也是ADS-B OUT的国际化标准。1090ES采用扩展型断续振荡的方式,由112个信息脉冲构成的S模式ADS-B长应答信号通过机载设备每隔1s广播一次。112位信息脉冲串的前88位为消息位、后24位为奇偶校验位。具体信息内容包括经度、纬度、方位和速度等信息。
下面简单介绍下1090MHz拓展性应答机随机自发报告(extended squitter)消息的格式。这个拓展性应答机随机自发报告消息长112bits,其中包含了56bits的ADS-B信息。前同步码(Preamable)包含了一个特定的用于实现同步的比特序列。下行链路(Downlink Format)格式域为5bits,这\newline 5bits表明了消息的类型—这个域被设置为17的话则代表了这是一条拓展性应答机随机自发报告消息。3bits的容量域(Capability)代表了S模式传输器的通信容量能力。飞行器地址(Aircraft Address)是24bits的传播器也及装载了该传播器的飞行器的唯一标识符。不存在两架不同的飞行器拥有统一个统一标识符。ADS-B数据(ADS-B Data)长56bits也包含了相关的监视数据(例如:身份信息,位置信息,速度信息,紧急码和质量等级)。奇偶校验值(Parity Check)是一个24bits的用于让接受者在接收到消息时检测传输错误的语。脉冲式调整方案被用于消息的编码和传输。ADS-B使用了一个广播式的通信范例。飞行器之间传输信息时并没有考虑敌人会接收信息,任何距离飞行器一定范围的敌人都可以接收并解码传输的消息。另外,传输的协议不要求消息接收的确认也没有实现任何的保持在线功能(例如:如果一架飞行器的ADS-B消息在一个确定的时间段内没有被地面基站或其他飞行器所接收到,这个系统将不会再询问这架飞行器的状态)。

本课题重点在于加解密算法的研究和实现故并未考虑ADS-B具体网络传输过程的模拟,只做原理性的介绍。

\subsection{ADS-B安全模型}
\subsubsection{不考虑商用航空,只考虑通用航空}
商用航空主要指固定的航班,根据航空飞行发,商用航班不允许匿名或隐藏轨迹,不仅因为这样做不利于飞行安全而且也不便于旅客查询航班的信息,并且商用航空每架飞机都有指定的航行路线,根据航行路线就可以推出航班号,因此根本没有匿名的必要。而通用航空则完全不同,通用航空包含一大部分领域的飞机,包括私人飞机、法院或其他紧急服务的飞机到商业飞机。这些飞机出于个人、政治、医疗或商业愿意不愿意暴露飞行动机和飞行目的地及喜好,因此存在保证保证他们的身份匿名的需求。

\subsubsection{不考虑主动攻击如ghost plane; dos}
针对ADS-B的攻击种类繁多,统分为主动攻击和被动攻击。本课题主要解决ADS-B的数据链隐私保护即保护ADS-B消息的机密性,防止ADS-B消息被无关人员窃听的被动攻击防御,故在本文中不考虑ADS-B假消息的注入及泛洪攻击等主动攻击的情形。

\subsubsection{遵守航空法不隐藏轨迹}
出于航行安全的考虑,航空安全法不建议任何通过ADS-B通信的飞行器隐藏轨迹,但是隐藏飞行器的唯一标识符并不会影响飞行安全且可以达到保护飞行器的隐私的效果,因此本课题通过加密ICAO24(通用的航空的唯一标识),ADS-B消息的一部分,且加密ICAO24并不会改变ADS-B消息中的飞行轨迹也不会影响ADS-B消息的解码而且可以达到保护通用航空隐私的目的。

\section{系统流程图}

第一步:从OPenSky的网站上下载到包含ADS-B的原始消息的.avro文件。.avro文件是apache下的一个数据序列化系统的文件,它拥有丰富的数据结构类型。本次用的的.avro文件的数据结构如下:

{
\tt
{\hlstd {\hllin 01\ }$\{$\leavevmode\par
{\hllin 02\ }"name":\ "ModeSEncodedMessage",\leavevmode\par
{\hllin 03\ }"type":\ "record",\leavevmode\par
{\hllin 04\ }"namespace":\ "org.opensky.avro.v2",\leavevmode\par
{\hllin 05\ }"fields":\ [\leavevmode\par
{\hllin 06\ }$\{$"name":\ "sensorType",\ "type":\ "string"$\}$,\leavevmode\par
{\hllin 07\ }$\{$"name":\ "sensorLatitude",\ "type":\ ["double",\ "null"]$\}$,\leavevmode\par
{\hllin 08\ }$\{$"name":\ "sensorLongitude",\ "type":\ ["double",\ "null"]$\}$,\leavevmode\par
{\hllin 09\ }$\{$"name":\ "sensorAltitude",\ "type":\ ["double",\ "null"]$\}$,\leavevmode\par
{\hllin 10\ }$\{$"name":\ "timeAtServer",\ "type":\ "double"$\}$,\leavevmode\par
{\hllin 11\ }$\{$"name":\ "timeAtSensor",\ "type":\ ["double",\ "null"]$\}$,\leavevmode\par
{\hllin 12\ }$\{$"name":\ "timestamp",\ "type":\ ["double",\ "null"]$\}$,\leavevmode\par
{\hllin 13\ }$\{$"name":\ "rawMessage",\ "type":\ "string"$\}$,\leavevmode\par
{\hllin 14\ }$\{$"name":\ "sensorSerialNumber",\ "type":\ "int"$\}$,\leavevmode\par
{\hllin 15\ }$\{$"name":\ "RSSIPacket",\ "type":\ ["double",\ "null"]$\}$,\leavevmode\par
{\hllin 16\ }$\{$"name":\ "RSSIPreamble",\ "type":\ ["double",\ "null"]$\}$,\leavevmode\par
{\hllin 17\ }$\{$"name":\ "SNR",\ "type":\ ["double",\ "null"]$\}$,\leavevmode\par
{\hllin 18\ }$\{$"name":\ "confidence",\ "type":\ ["double",\ "null"]$\}$\leavevmode\par
{\hllin 19\ }]\leavevmode\par
{\hllin 20\ }$\}$}\leavevmode\par
}


第二步:对原始的.avro文件按时间顺序进行排序,因为ADS-B消息解码位置信息时需要用到两条连续的ADS-B消息,所以要先对消息进行排序,否则无法对ADS-B消息的位置信息进行解码。

第三步:整个.avro包包含若大量的ADS-B消息,故也包含了若干条轨迹信息,本课题主要研究ADS-B消息的加解密,故只需从若干条轨迹中抽取出其中的一两条来进行试验分析即可。

第四步:对选取的一条轨迹的ADS-B消息在ADS-B OUT 中用FFX算法进行加密。

第五步:把加密过的若干条在公益轨迹上的ADS-B消息发送给ADS-B IN,本课题中没有模拟发送过程。

第六步:在ADS-B IN1中对加密过的ADS-B消息用FFX算法进行解密。

第七步:在ADS-B IN1中对解密的ADS-B消息进行解码。

第八步:把解码后的信息转换为.kml文件并在Google Earth上显示与未经加密的轨迹进行对比以并验证加解密效果。.kml文件是google公司创建的一种地理信息文件,用于描述和保存时间、地点、经纬度、海拔高度等轨迹参数。可用Google Earth打开。

第九步:在另一个ADS-B IN,即ADS-B IN2中直接对接收到的经过加密的ADS-B消息进行解码并转换为.kml文件在Google Earth上显示,与原始的未经过加密的ADS-B轨迹信息和经过解密的ADS-B轨迹信息进行对比以验证试验结果。

总的流程概括如下:\newline
sample.avro \to sorted.avro \to select.avro \to FFX Encrypt/Decrypt(ICAO24) \to decoder \to select.kml

\section{模块设计}
\subsection{加解密模块} 
FFX 模型的算法可以描述为:\newline
算法setup:\newline 
FFX 模型的初始化阶段确定\newline 
(1)字母表 $Chars=\{char_0, char_1,…, char_{(radix-1)}\}$及其基数 radix ;\newline 
(2)确定非平衡Feistel网络类型;\newline 
(3)消息空间$Char^n$中的字符串长度(字符个数) n ;\newline 
(4)Feistel网络使用的轮次数r ,伪随机函数$f_k$及其运算类型,和调整因子t等。\newline
算法encryption:\newline 
输入为明文字符串 x、对称密钥k 和调整因子t,输出为密文字符串 y ,字符串x和 y 同属于消息空间$Char^n$,都是由字母表 Chars 中的字符组成的长度为n 的字符串。\newline
加密过程描述为:\newline 
建立字母表Chars与 $Chars' = \{0,1,2,\cdots ,radix-1\}$的映射,从而将字符串中的每个字符$char_i$编码为相应的第 i 个数字,注意 Chars'中每个数字前面的0和其他字符一样计入长度。举例:Chars = ${a,b,c,\cdots,z}$,x = acz,将字符串 x 编码为010326 。\newline 
然后,执行r 轮的非平衡Feistel 网络的运算:\newline
(1)将字符串 x 的编码作为输入,并分割为两部分L和R ,$|L|\neq|R|$;\newline
(2)执行伪随机函数$f_k$,对 L 和$f_k (R)$执行下面某种类型的运算得到 L':(i)$c_i =( a_i + b_i)$  mod  radix ,当 radix = 2时,该运算为异或运算;(ii)$\sum c_i radix^(n-i) =( \sum a_i radix^{(n-i)} +\sum b_i radix^{(n-i)})$ mod  radix;\newline
(3)连接 L'与R 得到输出 L'$\parallel$R,并作为下一轮运算的输入。\newline 
最后,将非平衡Feistel 网络得到的密文中的数字编码映射回字母表中的字母,从而得到密文 y 。\newline
算法decryption:\newline 
解密算法为算法encryption的逆过程。 FFX 模型的工作原理如下图所示。

\pic[htbp]{FFX模型}{0.8\textwidth}{p12}

与FFSEM相比,FFX模型具有更广的适用范围,而且避免了Cycle-walking,具有较高的效率。\newline
FFX算法的三个主要步骤:\newline
step1:赋值\newline
把加密原文长度赋值给n,把分段长度赋值给l,把加密轮数赋值给r,把两个分段分别赋值给A和B。\newline
step2:Feistel转化\newline
把两段消息放入Feistel轮变化中经过r轮转化后输出。\newline
step3:合并\newline
把从Feisel轮变化中输出的两段消息合并。


\subsection{解码模块}
step1:建立从.avro文件中提取所需的ADS-B rawMessage各区域的类;

step2:先判断downlink format(DF)是否为17,是则继续step2;

step3:再判断第33-第37位的Type Code(TC)的值根据TC的不同判断ADS-B rawMessage包含的是什么信息,再针对rawMessage的类型对rawMessage分别进行不同的解码;

step4:把解码出来的ADS-B信息放在.kml文件的对应的位置。

下面对解码的最重要部分位置解码进行详细的讲解,ADS-B位置解码分为广域位置解码(全球解码)和局域位置解码(本地解码)
\subsubsection{广域位置解码(全球解码)} 
广域解码方式,全球解码需要两组编码数据来完成解码,即一组奇编码数据和一组偶编码数据,与局域解码方式相比,不需要提供基准位置信息,任何位置都可以解码。 采用接收到的偶编码(由YZ0,XZ0表示)和奇编码(由 YZ1,XZ1表示)的两个位置消息,共同来产生全球位置的纬度Rlat 和经度Rlon。对于空中位置,偶编码和奇编码位置消息之间的时间间隔不超过 10s,这是由 3 n mile 的最大允许间距决定的。
 
解码步骤主要分为以下几个部分:\newline 
(1)将经纬度的分区的数量 Dlati 计算出来。\newline 
对于空中位置和地面位置 CPR 分别计算分区的数量,CPR 奇编码计算的为$Dlati_1$ ,CPR 偶编码计算的为$Dlati_0$。\newline
空中位置:	
$Dlati_i=\frac{360}{4\times NZ-i}= \begin{cases} 6.00, & \mbox{偶编码} \\ 6.10, & \mbox{奇编码} \end{cases}$\newline
地面位置:
$Dlati_i=\frac{90}{4\times NZ-i}= \begin{cases} 0, & \mbox{偶编码} \\ 1, & \mbox{奇编码} \end{cases}$\newline
(2)计算纬度区域索引值 j 。\newline
$j = floor(\frac{59*YZ_0-60*YZ_1}{2^{17}} + \frac{1}{2})$\newline
当 j <0 时, j =偶纬度Zone的编号减 60, j =奇纬度Zone的编号减 59;\newline 
当 j >0 or j = 0 时, j =偶纬度Zone的编号, j =奇纬度Zone的编号。 \newline
(3)计算纬度值$Rlat_i$。\newline
$Rlat_i = Dlat_i(MOD(j,60-i)+\frac{YZ_i}{2^{17}})$\newline
若求出的纬度绝对值大于 $90^{\circ}$,则应减去 $360^{\circ}$。  因为纬度的取值范围为$-90^{\circ} ~ +90^{\circ}$。\newline 
(4)如果NL($Rlat_0$)不等于NL($Rlat_1$),则表示两次收到的位置信息不再同一个纬度区域内,这种情况说明飞行器在跨越纬度区域,不能解码,需等待接收到位置信息中经纬度区域值相等时解码。如果相等,则计算$Dlon_i$的值。$Dlon_i = \frac{360}{n_i}, n_i$ 为[NL($Rlat_i$)-1]中大的那个数。\newline 
(5)然后计算经度索引值 m 。\newline
$m = floor(\frac{XZ_0*(NL-1)-XZ_1*NL}{2^{17}}+\frac{1}{2})NL=NL(Rlat_i)$\newline
(6)最后计算经度$Rlon_i$。\newline
$Rloni=Dloni(\frac{MoD(m,ni)+XZ_i}{2^{17}})n_i = max([NL(Rlat_i)-1],1)$\newline
若求出的经度绝对值大于 $180^{\circ}$,则应减去 $360^{\circ}$。又因为无论是广域位置解码还是局域位置解码都仍然存在两种情况,即:空中位置和地面位置。但上述算法两种对于两种类型解码均适用。因为经度的取值范围为$-180^{\circ} ~ +180^{\circ}$。

\subsubsection{局域位置解码(本地解码)}
对于某个参考点(参考点可以是由全球解码确认的以前跟踪的某个位置,也可以是本机的位置,假定其纬度和经度分别为lats、lons ),CPR 算法将通过解码获得本地明确的地理位置(纬度$Rlat_i$和经度$Rlon_i$)。对于空中位置,该位置在真实位置的 180 n mile 内;对于地面位置,该位置在真实位置的 45 n mile(180/4 n mile)以内。\newline
解码步骤:\newline 
(1)首先计算纬度 Zone 的尺寸$Dlat_i$。\newline

空中位置:	
$Dlati_i=\frac{360}{4\times NZ-i}= \begin{cases} 6.00, & \mbox{偶编码} \\ 6.10, & \mbox{奇编码} \end{cases}$\newline
地面位置:
$Dlati_i=\frac{90}{4\times NZ-i}= \begin{cases} 0, & \mbox{偶编码} \\ 1, & \mbox{奇编码} \end{cases}$\newline
(2)计算纬度索引 j 。\newline
$j=floor(\frac{lat_s}{Dlat_i})+floor[\frac{1}{2}+\frac{MOD(lat_s,Dlat_i)}{Dlat_i}-\frac{YZ_i}{2^17}]$
纬度索引 j 的值与参考点的纬度lats的取值有关。\newline 
(3)解码纬度位置$Rlat_i$。\newline
$Rlat_i=Dlat_i\times(j+\frac{YZ_i}{2^{17}})$
本地解码后的纬度位置$Rlat_i$与参考点的纬度lats的取值有关。\newline 
(4)由$Rlat_i$确定东西向经度 Zone 的尺寸$Dlon_i$。\newline

空中位置:
$Dlati_i= \begin{cases} \frac{360}{[NL(Rlat_i)-i}, & NL(Rlat_i)-i>0 \\ 360, & NL(Rlat_i)-i=0 \end{cases}$\newline

地面位置:
$Dlati_i= \begin{cases} \frac{90}{[NL(Rlat_i)-i}, & NL(Rlat_i)-i>0 \\ 90, & NL(Rlat_i)-i=0 \end{cases}$\newline

(5)采用参考点的经度 lons 、$Dlon_i$和XZi计算经度索引 m 。\newline
$m=floor(\frac{lon_s}{Dlon_i})+floor[\frac{1}{2}+\frac{MOD(lon_s,Dlon_i)}{Dlon_i}-\frac{XZ_i}{2^{17}}]$;经度索引m 的值与参考点的经度lons 的取值有关。\newline 
(6)解码经度位置$Rlon_i$。\newline
$Rlon_i=Dlon_i\times(m+\frac{XZ_i}{2^{17}})$;本地解码后的经度位置$Rlon_i$与参考点的经度 lons 的取值有关。 

\subsection{可视化文件类型转换模块}
step1:创建装换类中的飞机消息格式内部类;\newline
step2:在转换的类中编写创建.kml文件的函数且将飞机消息格式内部类作为函数的输入参数;\newline
step3:通过添加函数把解码出来的信息对应放在创建.kml文件的函数的参数的飞机消息格式内部类的各个部分;

\section{算法编程}
\subsection{加解密部分}
\subsubsection{F函数部分代码}

{
\tt
{\hlstd {\hllin 01\ }def\ F(self,\ n,\ T,\ i,\ B):\leavevmode\par
{\hllin 02\ }}{\hlstd\ \ \ \ }{\hlstd if\ T\ $\mathord{=}$$\mathord{=}$\ 0:\leavevmode\par
{\hllin 03\ }}{\hlstd\ \ \ \ \ \ \ \ }{\hlstd t\ $\mathord{=}$\ 0\leavevmode\par
{\hllin 04\ }}{\hlstd\ \ \ \ }{\hlstd else:\leavevmode\par
{\hllin 05\ }}{\hlstd\ \ \ \ \ \ \ \ }{\hlstd t\ $\mathord{=}$\ len(T)\leavevmode\par
{\hllin 06\ }\leavevmode\par
{\hllin 07\ }}{\hlstd\ \ \ \ }{\hlstd beta\ $\mathord{=}$\ math.ceil(n\ /\ 2.0)\leavevmode\par
{\hllin 08\ }}{\hlstd\ \ \ \ }{\hlstd b\ $\mathord{=}$\ int(math.ceil(math.ceil(beta\ *\ math.log(self.\_{}radix,\ 2))\ /\ 8.0))\leavevmode\par
{\hllin 09\ }}{\hlstd\ \ \ \ }{\hlstd d\ $\mathord{=}$\ 4\ *\ int(math.ceil(b\ /\ 4.0))\leavevmode\par
{\hllin 10\ }\leavevmode\par
{\hllin 11\ }}{\hlstd\ \ \ \ }{\hlstd if\ self.isEven(i):\leavevmode\par
{\hllin 12\ }}{\hlstd\ \ \ \ \ \ \ \ }{\hlstd m\ $\mathord{=}$\ int(math.floor(n\ /\ 2.0))\leavevmode\par
{\hllin 13\ }}{\hlstd\ \ \ \ }{\hlstd else:\leavevmode\par
{\hllin 14\ }}{\hlstd\ \ \ \ \ \ \ \ }{\hlstd m\ $\mathord{=}$\ int(math.ceil(n\ /\ 2.0))\leavevmode\par
{\hllin 15\ }\leavevmode\par
{\hllin 16\ }}{\hlstd\ \ \ \ }{\hlstd if\ not\ self.\_{}P.get(n):\leavevmode\par
{\hllin 17\ }}{\hlstd\ \ \ \ \ \ \ \ }{\hlstd P\ $\mathord{=}$\ '$\backslash$x01'}{\hlstd\ \ }{\hlstd \#{}\ vers\leavevmode\par
{\hllin 18\ }}{\hlstd\ \ \ \ \ \ \ \ }{\hlstd P\ $\mathord{+}$$\mathord{=}$\ '$\backslash$x02'}{\hlstd\ \ }{\hlstd \#{}\ method\leavevmode\par
{\hllin 19\ }}{\hlstd\ \ \ \ \ \ \ \ }{\hlstd P\ $\mathord{+}$$\mathord{=}$\ '$\backslash$x01'}{\hlstd\ \ }{\hlstd \#{}\ addition\leavevmode\par
{\hllin 20\ }}{\hlstd\ \ \ \ \ \ \ \ }{\hlstd P\ $\mathord{+}$$\mathord{=}$\ long\_{}to\_{}bytes(self.\_{}radix,\ 3)\leavevmode\par
{\hllin 21\ }}{\hlstd\ \ \ \ \ \ \ \ }{\hlstd P\ $\mathord{+}$$\mathord{=}$\ '$\backslash$x0a'\ \#{}\ always\ ten\leavevmode\par
{\hllin 22\ }}{\hlstd\ \ \ \ \ \ \ \ }{\hlstd P\ $\mathord{+}$$\mathord{=}$\ long\_{}to\_{}bytes(self.split(n)\%{}256,\ 1)\leavevmode\par
{\hllin 23\ }}{\hlstd\ \ \ \ \ \ \ \ }{\hlstd P\ $\mathord{+}$$\mathord{=}$\ long\_{}to\_{}bytes(n,\ 4)\leavevmode\par
{\hllin 24\ }}{\hlstd\ \ \ \ \ \ \ \ }{\hlstd P\ $\mathord{+}$$\mathord{=}$\ long\_{}to\_{}bytes(t,\ 4)\leavevmode\par
{\hllin 25\ }}{\hlstd\ \ \ \ \ \ \ \ }{\hlstd self.\_{}P[n]\ $\mathord{=}$\ P\leavevmode\par
{\hllin 26\ }\leavevmode\par
{\hllin 27\ }}{\hlstd\ \ \ \ }{\hlstd if\ T\ $\mathord{=}$$\mathord{=}$\ 0:\leavevmode\par
{\hllin 28\ }}{\hlstd\ \ \ \ \ \ \ \ }{\hlstd Q\ $\mathord{=}$\ ''\leavevmode\par
{\hllin 29\ }}{\hlstd\ \ \ \ }{\hlstd else:\leavevmode\par
{\hllin 30\ }}{\hlstd\ \ \ \ \ \ \ \ }{\hlstd Q\ $\mathord{=}$\ str(T)\leavevmode\par
{\hllin 31\ }}{\hlstd\ \ \ \ \ \ \ \ \ \ \ \ }{\hlstd \leavevmode\par
{\hllin 32\ }}{\hlstd\ \ \ \ }{\hlstd Q\ $\mathord{+}$$\mathord{=}$\ '$\backslash$x00'\ *\ ((($\mathord{-}$1\ *\ t)\ $\mathord{-}$\ b\ $\mathord{-}$\ 1)\ \%{}\ 16)\leavevmode\par
{\hllin 33\ }}{\hlstd\ \ \ \ }{\hlstd Q\ $\mathord{+}$$\mathord{=}$\ long\_{}to\_{}bytes(i,\ blocksize$\mathord{=}$1)\leavevmode\par
{\hllin 34\ }}{\hlstd\ \ \ \ \ \ \ \ }{\hlstd \leavevmode\par
{\hllin 35\ }}{\hlstd\ \ \ \ }{\hlstd \_{}B\_{}as\_{}bytes\ $\mathord{=}$\ long\_{}to\_{}bytes(B)\leavevmode\par
{\hllin 36\ }}{\hlstd\ \ \ \ }{\hlstd Q\ $\mathord{+}$$\mathord{=}$\ '$\backslash$x00'\ *\ (b\ $\mathord{-}$\ len(\_{}B\_{}as\_{}bytes))\leavevmode\par
{\hllin 37\ }}{\hlstd\ \ \ \ }{\hlstd Q\ $\mathord{+}$$\mathord{=}$\ \_{}B\_{}as\_{}bytes[$\mathord{-}$b:]\leavevmode\par
{\hllin 38\ }\leavevmode\par
{\hllin 39\ }}{\hlstd\ \ \ \ }{\hlstd \_{}cbc\ $\mathord{=}$\ AES.new(self.\_{}K,\ AES.MODE\_{}CBC,\ '$\backslash$x00'\ *\ 16)\leavevmode\par
{\hllin 40\ }\leavevmode\par
{\hllin 41\ }}{\hlstd\ \ \ \ }{\hlstd assert\ len(self.\_{}P[n])\ \%{}\ 16\ $\mathord{=}$$\mathord{=}$\ 0\leavevmode\par
{\hllin 42\ }}{\hlstd\ \ \ \ }{\hlstd assert\ len(Q)\ \%{}\ 16\ $\mathord{=}$$\mathord{=}$\ 0\leavevmode\par
{\hllin 43\ }\leavevmode\par
{\hllin 44\ }}{\hlstd\ \ \ \ }{\hlstd Y\ $\mathord{=}$\ \_{}cbc.encrypt(self.\_{}P[n]\ $\mathord{+}$\ Q)[$\mathord{-}$16:]\leavevmode\par
{\hllin 45\ }\leavevmode\par
{\hllin 46\ }}{\hlstd\ \ \ \ }{\hlstd i\ $\mathord{=}$\ 1\leavevmode\par
{\hllin 47\ }}{\hlstd\ \ \ \ }{\hlstd TMP\ $\mathord{=}$\ Y\leavevmode\par
{\hllin 48\ }}{\hlstd\ \ \ \ }{\hlstd while\ len(TMP)\ $\mathord{<}$\ (d\ $\mathord{+}$\ 4):\leavevmode\par
{\hllin 49\ }}{\hlstd\ \ \ \ \ \ \ \ }{\hlstd left\ $\mathord{=}$\ FFXInteger(bytes\_{}to\_{}long(Y),\ radix$\mathord{=}$self.\_{}radix,\ blocksize$\mathord{=}$32)\leavevmode\par
{\hllin 50\ }}{\hlstd\ \ \ \ \ \ \ \ }{\hlstd right\ $\mathord{=}$\ FFXInteger(str(i),\ radix$\mathord{=}$10,\ blocksize$\mathord{=}$16)\leavevmode\par
{\hllin 51\ }}{\hlstd\ \ \ \ \ \ \ \ }{\hlstd X\ $\mathord{=}$\ self.add(left,\ right)\leavevmode\par
{\hllin 52\ }}{\hlstd\ \ \ \ \ \ \ \ }{\hlstd TMP\ $\mathord{+}$$\mathord{=}$\ self.\_{}ecb.encrypt(X.to\_{}bytes(16))\leavevmode\par
{\hllin 53\ }}{\hlstd\ \ \ \ \ \ \ \ }{\hlstd i\ $\mathord{+}$$\mathord{=}$\ 1\leavevmode\par
{\hllin 54\ }\leavevmode\par
{\hllin 55\ }}{\hlstd\ \ \ \ }{\hlstd y\ $\mathord{=}$\ bytes\_{}to\_{}long(TMP[:(d\ $\mathord{+}$\ 4)])\leavevmode\par
{\hllin 56\ }}{\hlstd\ \ \ \ }{\hlstd z\ $\mathord{=}$\ y\ \%{}\ (self.\_{}radix\ **\ m)\leavevmode\par
{\hllin 57\ }\leavevmode\par
{\hllin 58\ }}{\hlstd\ \ \ \ }{\hlstd return\ z}\leavevmode\par
}

        
F函数的步骤是首先对各个变量进行赋值:
t = $|T|$;
beta = $\lceil n/2 \rceil$;
b = $\lceil \lceil beta log_2(radix)\rceil /8 \rceil$;
d = 4 * $\lceil b/4 \rceil$;
如果i是偶数:m = $\lfloor n/2 \rfloor$;
如果i是奇数: m = $\lceil n/2 \rceil$;
P是vers、method、addition、radix、rnds(n)、split(n)、n、t的连接;
Q是T、0、i、B的连接;
Y = CBC-$MAC_K$(P$\parallel$Q);
Y = first d + 4 bytes of($Y \parallel AES_K(Y+[1]^{16})\parallel AES_K(Y + [2]^{16} \parallelAES_K(Y+[3]^{16})\cdots$);
y = $NUM_2(Y)$;
z = y mod $radix^m$;
z = $STR_radix^m(z)$;      
F函数的作用是对输入的其中的一半消息A或者B进行基于AES加密和参数连接的变化,且输出长度与输入长度一致的输出值,让同样长度的输出值进入下一轮迭代,每一轮的输出的长度都相等。
       
\subsubsection{加密函数部分代码}

{
\tt
{\hlstd {\hllin 01\ }def\ encrypt(self,\ T,\ X):\leavevmode\par
{\hllin 02\ }}{\hlstd\ \ \ \ }{\hlstd retval\ $\mathord{=}$\ ''\leavevmode\par
{\hllin 03\ }\leavevmode\par
{\hllin 04\ }}{\hlstd\ \ \ \ }{\hlstd n\ $\mathord{=}$\ len(X)\leavevmode\par
{\hllin 05\ }}{\hlstd\ \ \ \ }{\hlstd l\ $\mathord{=}$\ self.split(n)\leavevmode\par
{\hllin 06\ }}{\hlstd\ \ \ \ }{\hlstd r\ $\mathord{=}$\ 10\ \#{}self.rnds(n)\leavevmode\par
{\hllin 07\ }}{\hlstd\ \ \ \ }{\hlstd A\ $\mathord{=}$\ X[:l]\leavevmode\par
{\hllin 08\ }}{\hlstd\ \ \ \ }{\hlstd B\ $\mathord{=}$\ X[l:]\leavevmode\par
{\hllin 09\ }}{\hlstd\ \ \ \ }{\hlstd for\ i\ in\ range(r):\leavevmode\par
{\hllin 10\ }}{\hlstd\ \ \ \ \ \ \ \ }{\hlstd C\ $\mathord{=}$\ self.add(A,\ self.F(n,\ T,\ i,\ B))\leavevmode\par
{\hllin 11\ }}{\hlstd\ \ \ \ \ \ \ \ }{\hlstd A\ $\mathord{=}$\ B\leavevmode\par
{\hllin 12\ }}{\hlstd\ \ \ \ \ \ \ \ }{\hlstd B\ $\mathord{=}$\ C\leavevmode\par
{\hllin 13\ }\leavevmode\par
{\hllin 14\ }}{\hlstd\ \ \ \ }{\hlstd retval\ $\mathord{=}$\ FFXInteger(str(A)\ $\mathord{+}$\ str(B),\ radix$\mathord{=}$self.\_{}radix)\leavevmode\par
{\hllin 15\ }}{\hlstd\ \ \ \ \ \ \ \ }{\hlstd \leavevmode\par
{\hllin 16\ }}{\hlstd\ \ \ \ }{\hlstd return\ retval}\leavevmode\par
}


为加密过程中需要用到的各个参数赋值:
n = X的长度;
l = 分开的块的块大小,这里为输入原文的一半;
A、B分别为输入原文的前半段和后半段;
然后对A、B进行r轮的迭代变换,最后再把经过r轮变换的A、B连接在一起组成和输入原文长度一致的输出值。
\subsubsection{解密函数部分代码}
       
{
\tt
{\hlstd {\hllin 01\ }def\ decrypt(self,\ T,\ Y):\leavevmode\par
{\hllin 02\ }}{\hlstd\ \ \ \ }{\hlstd retval\ $\mathord{=}$\ ''\leavevmode\par
{\hllin 03\ }}{\hlstd\ \ \ \ \ \ \ \ }{\hlstd \leavevmode\par
{\hllin 04\ }}{\hlstd\ \ \ \ }{\hlstd n\ $\mathord{=}$\ len(Y)\leavevmode\par
{\hllin 05\ }}{\hlstd\ \ \ \ }{\hlstd l\ $\mathord{=}$\ self.split(n)\leavevmode\par
{\hllin 06\ }}{\hlstd\ \ \ \ }{\hlstd r\ $\mathord{=}$\ 10\ \#{}self.rnds(n)\leavevmode\par
{\hllin 07\ }\leavevmode\par
{\hllin 08\ }}{\hlstd\ \ \ \ }{\hlstd A\ $\mathord{=}$\ Y[:l]\leavevmode\par
{\hllin 09\ }}{\hlstd\ \ \ \ }{\hlstd B\ $\mathord{=}$\ Y[l:]\leavevmode\par
{\hllin 10\ }}{\hlstd\ \ \ \ }{\hlstd for\ i\ in\ range(r\ $\mathord{-}$\ 1,\ $\mathord{-}$1,\ $\mathord{-}$1):\leavevmode\par
{\hllin 11\ }}{\hlstd\ \ \ \ \ \ \ \ }{\hlstd C\ $\mathord{=}$\ B\leavevmode\par
{\hllin 12\ }}{\hlstd\ \ \ \ \ \ \ \ }{\hlstd B\ $\mathord{=}$\ A\leavevmode\par
{\hllin 13\ }}{\hlstd\ \ \ \ \ \ \ \ }{\hlstd A\ $\mathord{=}$\ self.sub(C,\ self.F(n,\ T,\ i,\ B))\leavevmode\par
{\hllin 14\ }}{\hlstd\ \ \ \ \ \ \ \ \ \ \ \ }{\hlstd \leavevmode\par
{\hllin 15\ }}{\hlstd\ \ \ \ }{\hlstd retval\ $\mathord{=}$\ FFXInteger(str(A)\ $\mathord{+}$\ str(B),\ radix$\mathord{=}$self.\_{}radix)\leavevmode\par
{\hllin 16\ }\leavevmode\par
{\hllin 17\ }}{\hlstd\ \ \ \ }{\hlstd return\ retval}\leavevmode\par
}

        
同理对n、l、r三个变量赋值,在解密中,与加密不同的是A这时被赋值为密文的后半段,B被赋值为前半段;循环也是与加密中的循环流程相反.        
\subsubsection{把加密部分的代码植入java的解码和可视化系统中}

{
\tt
{\hlstd {\hllin 01\ }public\ class\ ModeSEncodedMessage\ extends\ org.apache.avro.specific.SpecificRecordBase\ implements\ org.apache.avro.specific.SpecificRecord\ $\{$\leavevmode\par
{\hllin 02\ }}{\hlstd\ \ }{\hlstd public\ static\ final\ org.apache.avro.Schema\ SCHEMA\${}\ $\mathord{=}$\ new\ org.apache.avro.Schema.Parser().parse("$\{$$\backslash$"type$\backslash$":$\backslash$"record$\backslash$",$\backslash$"name$\backslash$":$\backslash$"ModeSEncodedMessage$\backslash$",$\backslash$"namespace$\backslash$":$\backslash$"org.opensky.sample$\backslash$",$\backslash$"fields$\backslash$":[$\{$$\backslash$"name$\backslash$":$\backslash$"sensorType$\backslash$",$\backslash$"type$\backslash$":$\backslash$"string$\backslash$"$\}$,$\{$$\backslash$"name$\backslash$":$\backslash$"sensorLatitude$\backslash$",$\backslash$"type$\backslash$":[$\backslash$"double$\backslash$",$\backslash$"null$\backslash$"]$\}$,$\{$$\backslash$"name$\backslash$":$\backslash$"sensorLongitude$\backslash$",$\backslash$"type$\backslash$":[$\backslash$"double$\backslash$",$\backslash$"null$\backslash$"]$\}$,$\{$$\backslash$"name$\backslash$":$\backslash$"sensorAltitude$\backslash$",$\backslash$"type$\backslash$":[$\backslash$"double$\backslash$",$\backslash$"null$\backslash$"]$\}$,$\{$$\backslash$"name$\backslash$":$\backslash$"timeAtServer$\backslash$",$\backslash$"type$\backslash$":$\backslash$"double$\backslash$"$\}$,$\{$$\backslash$"name$\backslash$":$\backslash$"timeAtSensor$\backslash$",$\backslash$"type$\backslash$":[$\backslash$"double$\backslash$",$\backslash$"null$\backslash$"]$\}$,$\{$$\backslash$"name$\backslash$":$\backslash$"timestamp$\backslash$",$\backslash$"type$\backslash$":[$\backslash$"double$\backslash$",$\backslash$"null$\backslash$"]$\}$,$\{$$\backslash$"name$\backslash$":$\backslash$"rawMessage$\backslash$",$\backslash$"type$\backslash$":$\backslash$"string$\backslash$"$\}$,$\{$$\backslash$"name$\backslash$":$\backslash$"sensorSerialNumber$\backslash$",$\backslash$"type$\backslash$":$\backslash$"int$\backslash$"$\}$,$\{$$\backslash$"name$\backslash$":$\backslash$"RSSIPacket$\backslash$",$\backslash$"type$\backslash$":[$\backslash$"double$\backslash$",$\backslash$"null$\backslash$"]$\}$,$\{$$\backslash$"name$\backslash$":$\backslash$"RSSIPreamble$\backslash$",$\backslash$"type$\backslash$":[$\backslash$"double$\backslash$",$\backslash$"null$\backslash$"]$\}$,$\{$$\backslash$"name$\backslash$":$\backslash$"SNR$\backslash$",$\backslash$"type$\backslash$":[$\backslash$"double$\backslash$",$\backslash$"null$\backslash$"]$\}$,$\{$$\backslash$"name$\backslash$":$\backslash$"confidence$\backslash$",$\backslash$"type$\backslash$":[$\backslash$"double$\backslash$",$\backslash$"null$\backslash$"]$\}$]$\}$");\leavevmode\par
{\hllin 03\ }}{\hlstd\ \ }{\hlstd public\ static\ org.apache.avro.Schema\ getClassSchema()\ $\{$\ return\ SCHEMA\${};\ $\}$\leavevmode\par
{\hllin 04\ }}{\hlstd\ \ }{\hlstd @Deprecated\ public\ java.lang.CharSequence\ sensorType;\leavevmode\par
{\hllin 05\ }}{\hlstd\ \ }{\hlstd @Deprecated\ public\ java.lang.Double\ sensorLatitude;\leavevmode\par
{\hllin 06\ }}{\hlstd\ \ }{\hlstd @Deprecated\ public\ java.lang.Double\ sensorLongitude;\leavevmode\par
{\hllin 07\ }}{\hlstd\ \ }{\hlstd @Deprecated\ public\ java.lang.Double\ sensorAltitude;\leavevmode\par
{\hllin 08\ }}{\hlstd\ \ }{\hlstd @Deprecated\ public\ double\ timeAtServer;\leavevmode\par
{\hllin 09\ }}{\hlstd\ \ }{\hlstd @Deprecated\ public\ java.lang.Double\ timeAtSensor;\leavevmode\par
{\hllin 10\ }}{\hlstd\ \ }{\hlstd @Deprecated\ public\ java.lang.Double\ timestamp;\leavevmode\par
{\hllin 11\ }}{\hlstd\ \ }{\hlstd @Deprecated\ public\ java.lang.CharSequence\ rawMessage;\leavevmode\par
{\hllin 12\ }}{\hlstd\ \ }{\hlstd @Deprecated\ public\ int\ sensorSerialNumber;\leavevmode\par
{\hllin 13\ }}{\hlstd\ \ }{\hlstd @Deprecated\ public\ java.lang.Double\ RSSIPacket;\leavevmode\par
{\hllin 14\ }}{\hlstd\ \ }{\hlstd @Deprecated\ public\ java.lang.Double\ RSSIPreamble;\leavevmode\par
{\hllin 15\ }}{\hlstd\ \ }{\hlstd @Deprecated\ public\ java.lang.Double\ SNR;\leavevmode\par
{\hllin 16\ }}{\hlstd\ \ }{\hlstd @Deprecated\ public\ java.lang.Double\ confidence;\leavevmode\par
{\hllin 17\ }\leavevmode\par
{\hllin 18\ }}{\hlstd\ \ }{\hlstd /**\leavevmode\par
{\hllin 19\ }}{\hlstd\ \ \ }{\hlstd *\ Default\ constructor.}{\hlstd\ \ }{\hlstd Note\ that\ this\ does\ not\ initialize\ fields\leavevmode\par
{\hllin 20\ }}{\hlstd\ \ \ }{\hlstd *\ to\ their\ default\ values\ from\ the\ schema.}{\hlstd\ \ }{\hlstd If\ that\ is\ desired\ then\leavevmode\par
{\hllin 21\ }}{\hlstd\ \ \ }{\hlstd *\ one\ should\ use\ $\mathord{<}$code$\mathord{>}$newBuilder()$\mathord{<}$/code$\mathord{>}$.\ \leavevmode\par
{\hllin 22\ }}{\hlstd\ \ \ }{\hlstd */\leavevmode\par
{\hllin 23\ }}{\hlstd\ \ }{\hlstd public\ ModeSEncodedMessage()\ $\{$$\}$\leavevmode\par
{\hllin 24\ }}{\hlstd\ \ }{\hlstd private}{\hlstd\ \ }{\hlstd CharSequence\ FFXEncrypt(CharSequence\ rawMessage)$\{$\leavevmode\par
{\hllin 25\ }}{\hlstd\ \ \ \ }{\hlstd System.err.println("unencrypted\ raw\ message\ is\ "\ $\mathord{+}$\ rawMessage);\leavevmode\par
{\hllin 26\ }}{\hlstd\ \ \ \ }{\hlstd try\ $\{$\leavevmode\par
{\hllin 27\ }}{\hlstd\ \ \ \ \ \ }{\hlstd Process\ proc\ $\mathord{=}$\ Runtime.getRuntime().exec("python\ $\backslash$"C:$\backslash$$\backslash$Users$\backslash$$\backslash$Serlina$\backslash$$\backslash$PycharmProjects$\backslash$$\backslash$FinalyearProject$\backslash$$\backslash$ADS$\mathord{-}$B\ out$\backslash$$\backslash$libffx$\backslash$$\backslash$example.py$\backslash$"\ "\ $\mathord{+}$\ rawMessage);\leavevmode\par
{\hllin 28\ }}{\hlstd\ \ \ \ \ \ }{\hlstd BufferedReader\ bufferedReader\ $\mathord{=}$\ new\ BufferedReader(new\ InputStreamReader(proc.getInputStream()));\leavevmode\par
{\hllin 29\ }}{\hlstd\ \ \ \ \ \ }{\hlstd rawMessage\ $\mathord{=}$\ bufferedReader.readLine();\leavevmode\par
{\hllin 30\ }}{\hlstd\ \ \ \ \ \ }{\hlstd proc.waitFor();\leavevmode\par
{\hllin 31\ }}{\hlstd\ \ \ \ }{\hlstd $\}$\leavevmode\par
{\hllin 32\ }}{\hlstd\ \ \ \ }{\hlstd catch\ (Exception\ e)$\{$\leavevmode\par
{\hllin 33\ }}{\hlstd\ \ \ \ \ \ }{\hlstd e.printStackTrace();\leavevmode\par
{\hllin 34\ }}{\hlstd\ \ \ \ }{\hlstd $\}$\leavevmode\par
{\hllin 35\ }}{\hlstd\ \ \ \ }{\hlstd System.err.println("encrypted\ raw\ message\ is\ "\ $\mathord{+}$\ rawMessage);\leavevmode\par
{\hllin 36\ }}{\hlstd\ \ \ \ }{\hlstd return\ rawMessage;\leavevmode\par
{\hllin 37\ }}{\hlstd\ \ }{\hlstd $\}$\leavevmode\par
{\hllin 38\ }}{\hlstd\ \ }{\hlstd private}{\hlstd\ \ }{\hlstd CharSequence\ FFXDecrypt(CharSequence\ encryptedMessage)$\{$\leavevmode\par
{\hllin 39\ }}{\hlstd\ \ \ \ }{\hlstd System.err.println("encrypted\ raw\ message\ is\ "\ $\mathord{+}$\ encryptedMessage);\leavevmode\par
{\hllin 40\ }}{\hlstd\ \ \ \ }{\hlstd try\ $\{$\leavevmode\par
{\hllin 41\ }}{\hlstd\ \ \ \ \ \ }{\hlstd Process\ proc\ $\mathord{=}$\ Runtime.getRuntime().exec("python\ $\backslash$"C:$\backslash$$\backslash$Users$\backslash$$\backslash$Serlina$\backslash$$\backslash$PycharmProjects$\backslash$$\backslash$FinalyearProject$\backslash$$\backslash$ADS$\mathord{-}$B\ in$\backslash$$\backslash$libffx$\backslash$$\backslash$example.py$\backslash$"\ "\ $\mathord{+}$\ encryptedMessage);\leavevmode\par
{\hllin 42\ }}{\hlstd\ \ \ \ \ \ }{\hlstd BufferedReader\ bufferedReader\ $\mathord{=}$\ new\ BufferedReader(new\ InputStreamReader(proc.getInputStream()));\leavevmode\par
{\hllin 43\ }}{\hlstd\ \ \ \ \ \ }{\hlstd rawMessage\ $\mathord{=}$\ bufferedReader.readLine();\leavevmode\par
{\hllin 44\ }}{\hlstd\ \ \ \ \ \ }{\hlstd proc.waitFor();\leavevmode\par
{\hllin 45\ }}{\hlstd\ \ \ \ }{\hlstd $\}$\leavevmode\par
{\hllin 46\ }}{\hlstd\ \ \ \ }{\hlstd catch\ (Exception\ e)$\{$\leavevmode\par
{\hllin 47\ }}{\hlstd\ \ \ \ \ \ }{\hlstd e.printStackTrace();\leavevmode\par
{\hllin 48\ }}{\hlstd\ \ \ \ }{\hlstd $\}$\leavevmode\par
{\hllin 49\ }}{\hlstd\ \ \ \ }{\hlstd System.err.println("decrypted\ raw\ message\ is\ "\ $\mathord{+}$\ rawMessage);\leavevmode\par
{\hllin 50\ }}{\hlstd\ \ \ \ }{\hlstd return\ rawMessage;\leavevmode\par
{\hllin 51\ }}{\hlstd\ \ }{\hlstd $\}$\leavevmode\par
{\hllin 52\ }}{\hlstd\ \ }{\hlstd /**\leavevmode\par
{\hllin 53\ }}{\hlstd\ \ \ }{\hlstd *\ All$\mathord{-}$args\ constructor.\leavevmode\par
{\hllin 54\ }}{\hlstd\ \ \ }{\hlstd */\leavevmode\par
{\hllin 55\ }}{\hlstd\ \ }{\hlstd public\ ModeSEncodedMessage(java.lang.CharSequence\ sensorType,\ java.lang.Double\ sensorLatitude,\ java.lang.Double\ sensorLongitude,\ java.lang.Double\ sensorAltitude,\ java.lang.Double\ timeAtServer,\ java.lang.Double\ timeAtSensor,\ java.lang.Double\ timestamp,\ java.lang.CharSequence\ rawMessage,\ java.lang.Integer\ sensorSerialNumber,\ java.lang.Double\ RSSIPacket,\ java.lang.Double\ RSSIPreamble,\ java.lang.Double\ SNR,\ java.lang.Double\ confidence)\ $\{$\leavevmode\par
{\hllin 56\ }}{\hlstd\ \ \ \ }{\hlstd this.sensorType\ $\mathord{=}$\ sensorType;\leavevmode\par
{\hllin 57\ }}{\hlstd\ \ \ \ }{\hlstd this.sensorLatitude\ $\mathord{=}$\ sensorLatitude;\leavevmode\par
{\hllin 58\ }}{\hlstd\ \ \ \ }{\hlstd this.sensorLongitude\ $\mathord{=}$\ sensorLongitude;\leavevmode\par
{\hllin 59\ }}{\hlstd\ \ \ \ }{\hlstd this.sensorAltitude\ $\mathord{=}$\ sensorAltitude;\leavevmode\par
{\hllin 60\ }}{\hlstd\ \ \ \ }{\hlstd this.timeAtServer\ $\mathord{=}$\ timeAtServer;\leavevmode\par
{\hllin 61\ }}{\hlstd\ \ \ \ }{\hlstd this.timeAtSensor\ $\mathord{=}$\ timeAtSensor;\leavevmode\par
{\hllin 62\ }}{\hlstd\ \ \ \ }{\hlstd this.timestamp\ $\mathord{=}$\ timestamp;\leavevmode\par
{\hllin 63\ }}{\hlstd\ \ \ \ }{\hlstd this.rawMessage\ $\mathord{=}$\ rawMessage;\leavevmode\par
{\hllin 64\ }}{\hlstd\ \ \ \ }{\hlstd this.sensorSerialNumber\ $\mathord{=}$\ sensorSerialNumber;\leavevmode\par
{\hllin 65\ }}{\hlstd\ \ \ \ }{\hlstd this.RSSIPacket\ $\mathord{=}$\ RSSIPacket;\leavevmode\par
{\hllin 66\ }}{\hlstd\ \ \ \ }{\hlstd this.RSSIPreamble\ $\mathord{=}$\ RSSIPreamble;\leavevmode\par
{\hllin 67\ }}{\hlstd\ \ \ \ }{\hlstd this.SNR\ $\mathord{=}$\ SNR;\leavevmode\par
{\hllin 68\ }}{\hlstd\ \ \ \ }{\hlstd this.confidence\ $\mathord{=}$\ confidence;\leavevmode\par
{\hllin 69\ }}{\hlstd\ \ }{\hlstd $\}$\leavevmode\par
{\hllin 70\ }\leavevmode\par
{\hllin 71\ }}{\hlstd\ \ }{\hlstd public\ org.apache.avro.Schema\ getSchema()\ $\{$\ return\ SCHEMA\${};\ $\}$\leavevmode\par
{\hllin 72\ }}{\hlstd\ \ }{\hlstd //\ Used\ by\ DatumWriter.}{\hlstd\ \ }{\hlstd Applications\ should\ not\ call.\ \leavevmode\par
{\hllin 73\ }}{\hlstd\ \ }{\hlstd public\ java.lang.Object\ get(int\ field\${})\ $\{$\leavevmode\par
{\hllin 74\ }}{\hlstd\ \ \ \ }{\hlstd switch\ (field\${})\ $\{$\leavevmode\par
{\hllin 75\ }}{\hlstd\ \ \ \ }{\hlstd case\ 0:\ return\ sensorType;\leavevmode\par
{\hllin 76\ }}{\hlstd\ \ \ \ }{\hlstd case\ 1:\ return\ sensorLatitude;\leavevmode\par
{\hllin 77\ }}{\hlstd\ \ \ \ }{\hlstd case\ 2:\ return\ sensorLongitude;\leavevmode\par
{\hllin 78\ }}{\hlstd\ \ \ \ }{\hlstd case\ 3:\ return\ sensorAltitude;\leavevmode\par
{\hllin 79\ }}{\hlstd\ \ \ \ }{\hlstd case\ 4:\ return\ timeAtServer;\leavevmode\par
{\hllin 80\ }}{\hlstd\ \ \ \ }{\hlstd case\ 5:\ return\ timeAtSensor;\leavevmode\par
{\hllin 81\ }}{\hlstd\ \ \ \ }{\hlstd case\ 6:\ return\ timestamp;\leavevmode\par
{\hllin 82\ }}{\hlstd\ \ \ \ }{\hlstd case\ 7:\ return\ rawMessage;\leavevmode\par
{\hllin 83\ }}{\hlstd\ \ \ \ }{\hlstd case\ 8:\ return\ sensorSerialNumber;\leavevmode\par
{\hllin 84\ }}{\hlstd\ \ \ \ }{\hlstd case\ 9:\ return\ RSSIPacket;\leavevmode\par
{\hllin 85\ }}{\hlstd\ \ \ \ }{\hlstd case\ 10:\ return\ RSSIPreamble;\leavevmode\par
{\hllin 86\ }}{\hlstd\ \ \ \ }{\hlstd case\ 11:\ return\ SNR;\leavevmode\par
{\hllin 87\ }}{\hlstd\ \ \ \ }{\hlstd case\ 12:\ return\ confidence;\leavevmode\par
{\hllin 88\ }}{\hlstd\ \ \ \ }{\hlstd default:\ throw\ new\ org.apache.avro.AvroRuntimeException("Bad\ index");\leavevmode\par
{\hllin 89\ }}{\hlstd\ \ \ \ }{\hlstd $\}$\leavevmode\par
{\hllin 90\ }}{\hlstd\ \ }{\hlstd $\}$\leavevmode\par
{\hllin 91\ }}{\hlstd\ \ }{\hlstd //\ Used\ by\ DatumReader.}{\hlstd\ \ }{\hlstd Applications\ should\ not\ call.\ \leavevmode\par
{\hllin 92\ }}{\hlstd\ \ }{\hlstd @SuppressWarnings(value$\mathord{=}$"unchecked")\leavevmode\par
{\hllin 93\ }}{\hlstd\ \ }{\hlstd public\ void\ put(int\ field\${},\ java.lang.Object\ value\${})\ $\{$\leavevmode\par
{\hllin 94\ }}{\hlstd\ \ \ \ }{\hlstd switch\ (field\${})\ $\{$\leavevmode\par
{\hllin 95\ }}{\hlstd\ \ \ \ }{\hlstd case\ 0:\ sensorType\ $\mathord{=}$\ (java.lang.CharSequence)value\${};\ break;\leavevmode\par
{\hllin 96\ }}{\hlstd\ \ \ \ }{\hlstd case\ 1:\ sensorLatitude\ $\mathord{=}$\ (java.lang.Double)value\${};\ break;\leavevmode\par
{\hllin 97\ }}{\hlstd\ \ \ \ }{\hlstd case\ 2:\ sensorLongitude\ $\mathord{=}$\ (java.lang.Double)value\${};\ break;\leavevmode\par
{\hllin 98\ }}{\hlstd\ \ \ \ }{\hlstd case\ 3:\ sensorAltitude\ $\mathord{=}$\ (java.lang.Double)value\${};\ break;\leavevmode\par
{\hllin 99\ }}{\hlstd\ \ \ \ }{\hlstd case\ 4:\ timeAtServer\ $\mathord{=}$\ (java.lang.Double)value\${};\ break;\leavevmode\par
{\hllin 100\ }}{\hlstd\ \ \ \ }{\hlstd case\ 5:\ timeAtSensor\ $\mathord{=}$\ (java.lang.Double)value\${};\ break;\leavevmode\par
{\hllin 101\ }}{\hlstd\ \ \ \ }{\hlstd case\ 6:\ timestamp\ $\mathord{=}$\ (java.lang.Double)value\${};\ break;\leavevmode\par
{\hllin 102\ }}{\hlstd\ \ \ \ }{\hlstd case\ 7:\ rawMessage\ $\mathord{=}$\ (java.lang.CharSequence)value\${};\ break;\leavevmode\par
{\hllin 103\ }}{\hlstd\ \ \ \ }{\hlstd case\ 8:\ sensorSerialNumber\ $\mathord{=}$\ (java.lang.Integer)value\${};\ break;\leavevmode\par
{\hllin 104\ }}{\hlstd\ \ \ \ }{\hlstd case\ 9:\ RSSIPacket\ $\mathord{=}$\ (java.lang.Double)value\${};\ break;\leavevmode\par
{\hllin 105\ }}{\hlstd\ \ \ \ }{\hlstd case\ 10:\ RSSIPreamble\ $\mathord{=}$\ (java.lang.Double)value\${};\ break;\leavevmode\par
{\hllin 106\ }}{\hlstd\ \ \ \ }{\hlstd case\ 11:\ SNR\ $\mathord{=}$\ (java.lang.Double)value\${};\ break;\leavevmode\par
{\hllin 107\ }}{\hlstd\ \ \ \ }{\hlstd case\ 12:\ confidence\ $\mathord{=}$\ (java.lang.Double)value\${};\ break;\leavevmode\par
{\hllin 108\ }}{\hlstd\ \ \ \ }{\hlstd default:\ throw\ new\ org.apache.avro.AvroRuntimeException("Bad\ index");\leavevmode\par
{\hllin 109\ }}{\hlstd\ \ \ \ }{\hlstd $\}$\leavevmode\par
{\hllin 110\ }}{\hlstd\ \ }{\hlstd $\}$\leavevmode\par
{\hllin 111\ }\leavevmode\par
{\hllin 112\ }}{\hlstd\ \ }{\hlstd /**\leavevmode\par
{\hllin 113\ }}{\hlstd\ \ \ }{\hlstd *\ Gets\ the\ value\ of\ the\ 'sensorType'\ field.\leavevmode\par
{\hllin 114\ }}{\hlstd\ \ \ }{\hlstd */\leavevmode\par
{\hllin 115\ }}{\hlstd\ \ }{\hlstd public\ java.lang.CharSequence\ getSensorType()\ $\{$\leavevmode\par
{\hllin 116\ }}{\hlstd\ \ \ \ }{\hlstd return\ sensorType;\leavevmode\par
{\hllin 117\ }}{\hlstd\ \ }{\hlstd $\}$\leavevmode\par
{\hllin 118\ }\leavevmode\par
{\hllin 119\ }}{\hlstd\ \ }{\hlstd /**\leavevmode\par
{\hllin 120\ }}{\hlstd\ \ \ }{\hlstd *\ Sets\ the\ value\ of\ the\ 'sensorType'\ field.\leavevmode\par
{\hllin 121\ }}{\hlstd\ \ \ }{\hlstd *\ @param\ value\ the\ value\ to\ set.\leavevmode\par
{\hllin 122\ }}{\hlstd\ \ \ }{\hlstd */\leavevmode\par
{\hllin 123\ }}{\hlstd\ \ }{\hlstd public\ void\ setSensorType(java.lang.CharSequence\ value)\ $\{$\leavevmode\par
{\hllin 124\ }}{\hlstd\ \ \ \ }{\hlstd this.sensorType\ $\mathord{=}$\ value;\leavevmode\par
{\hllin 125\ }}{\hlstd\ \ }{\hlstd $\}$\leavevmode\par
{\hllin 126\ }\leavevmode\par
{\hllin 127\ }...\leavevmode\par
{\hllin 128\ }...\leavevmode\par
{\hllin 129\ }...\leavevmode\par
{\hllin 130\ }\leavevmode\par
{\hllin 131\ }}{\hlstd\ \ }{\hlstd /**\leavevmode\par
{\hllin 132\ }}{\hlstd\ \ \ }{\hlstd *\ Gets\ the\ value\ of\ the\ 'rawMessage'\ field.\leavevmode\par
{\hllin 133\ }}{\hlstd\ \ \ }{\hlstd */\leavevmode\par
{\hllin 134\ }}{\hlstd\ \ }{\hlstd public\ java.lang.CharSequence\ getRawMessage()\ $\{$\leavevmode\par
{\hllin 135\ }}{\hlstd\ \ \ \ }{\hlstd return\ FFXEncrypt(rawMessage);\leavevmode\par
{\hllin 136\ }}{\hlstd\ \ }{\hlstd $\}$\leavevmode\par
{\hllin 137\ }\leavevmode\par
{\hllin 138\ }}{\hlstd\ \ }{\hlstd /**\leavevmode\par
{\hllin 139\ }}{\hlstd\ \ \ }{\hlstd *\ Sets\ the\ value\ of\ the\ 'rawMessage'\ field.\leavevmode\par
{\hllin 140\ }}{\hlstd\ \ \ }{\hlstd *\ @param\ value\ the\ value\ to\ set.\leavevmode\par
{\hllin 141\ }}{\hlstd\ \ \ }{\hlstd */\leavevmode\par
{\hllin 142\ }}{\hlstd\ \ }{\hlstd public\ void\ setRawMessage(java.lang.CharSequence\ value)\ $\{$\leavevmode\par
{\hllin 143\ }}{\hlstd\ \ \ \ }{\hlstd this.rawMessage\ $\mathord{=}$\ value;\leavevmode\par
{\hllin 144\ }}{\hlstd\ \ }{\hlstd $\}$\leavevmode\par
{\hllin 145\ }}{\hlstd\ \ }{\hlstd \leavevmode\par
{\hllin 146\ }...\leavevmode\par
{\hllin 147\ }...\leavevmode\par
{\hllin 148\ }...\leavevmode\par
{\hllin 149\ }\leavevmode\par
{\hllin 150\ }}{\hlstd\ \ }{\hlstd /**\leavevmode\par
{\hllin 151\ }}{\hlstd\ \ \ }{\hlstd *\ Gets\ the\ value\ of\ the\ 'confidence'\ field.\leavevmode\par
{\hllin 152\ }}{\hlstd\ \ \ }{\hlstd */\leavevmode\par
{\hllin 153\ }}{\hlstd\ \ }{\hlstd public\ java.lang.Double\ getConfidence()\ $\{$\leavevmode\par
{\hllin 154\ }}{\hlstd\ \ \ \ }{\hlstd return\ confidence;\leavevmode\par
{\hllin 155\ }}{\hlstd\ \ }{\hlstd $\}$\leavevmode\par
{\hllin 156\ }\leavevmode\par
{\hllin 157\ }}{\hlstd\ \ }{\hlstd /**\leavevmode\par
{\hllin 158\ }}{\hlstd\ \ \ }{\hlstd *\ Sets\ the\ value\ of\ the\ 'confidence'\ field.\leavevmode\par
{\hllin 159\ }}{\hlstd\ \ \ }{\hlstd *\ @param\ value\ the\ value\ to\ set.\leavevmode\par
{\hllin 160\ }}{\hlstd\ \ \ }{\hlstd */\leavevmode\par
{\hllin 161\ }}{\hlstd\ \ }{\hlstd public\ void\ setConfidence(java.lang.Double\ value)\ $\{$\leavevmode\par
{\hllin 162\ }}{\hlstd\ \ \ \ }{\hlstd this.confidence\ $\mathord{=}$\ value;\leavevmode\par
{\hllin 163\ }}{\hlstd\ \ }{\hlstd $\}$\leavevmode\par
{\hllin 164\ }\leavevmode\par
{\hllin 165\ }}{\hlstd\ \ }{\hlstd /**\ Creates\ a\ new\ ModeSEncodedMessage\ RecordBuilder\ */\leavevmode\par
{\hllin 166\ }}{\hlstd\ \ }{\hlstd public\ static\ org.opensky.example.ModeSEncodedMessage.Builder\ newBuilder()\ $\{$\leavevmode\par
{\hllin 167\ }}{\hlstd\ \ \ \ }{\hlstd return\ new\ org.opensky.example.ModeSEncodedMessage.Builder();\leavevmode\par
{\hllin 168\ }}{\hlstd\ \ }{\hlstd $\}$\leavevmode\par
{\hllin 169\ }}{\hlstd\ \ }{\hlstd \leavevmode\par
{\hllin 170\ }}{\hlstd\ \ }{\hlstd /**\ Creates\ a\ new\ ModeSEncodedMessage\ RecordBuilder\ by\ copying\ an\ existing\ Builder\ */\leavevmode\par
{\hllin 171\ }}{\hlstd\ \ }{\hlstd public\ static\ org.opensky.example.ModeSEncodedMessage.Builder\ newBuilder(org.opensky.example.ModeSEncodedMessage.Builder\ other)\ $\{$\leavevmode\par
{\hllin 172\ }}{\hlstd\ \ \ \ }{\hlstd return\ new\ org.opensky.example.ModeSEncodedMessage.Builder(other);\leavevmode\par
{\hllin 173\ }}{\hlstd\ \ }{\hlstd $\}$\leavevmode\par
{\hllin 174\ }}{\hlstd\ \ }{\hlstd \leavevmode\par
{\hllin 175\ }}{\hlstd\ \ }{\hlstd /**\ Creates\ a\ new\ ModeSEncodedMessage\ RecordBuilder\ by\ copying\ an\ existing\ ModeSEncodedMessage\ instance\ */\leavevmode\par
{\hllin 176\ }}{\hlstd\ \ }{\hlstd public\ static\ org.opensky.example.ModeSEncodedMessage.Builder\ newBuilder(org.opensky.example.ModeSEncodedMessage\ other)\ $\{$\leavevmode\par
{\hllin 177\ }}{\hlstd\ \ \ \ }{\hlstd return\ new\ org.opensky.example.ModeSEncodedMessage.Builder(other);\leavevmode\par
{\hllin 178\ }}{\hlstd\ \ }{\hlstd $\}$\leavevmode\par
{\hllin 179\ }}{\hlstd\ \ }{\hlstd \leavevmode\par
{\hllin 180\ }}{\hlstd\ \ }{\hlstd /**\leavevmode\par
{\hllin 181\ }}{\hlstd\ \ \ }{\hlstd *\ RecordBuilder\ for\ ModeSEncodedMessage\ instances.\leavevmode\par
{\hllin 182\ }}{\hlstd\ \ \ }{\hlstd */\leavevmode\par
{\hllin 183\ }}{\hlstd\ \ }{\hlstd public\ static\ class\ Builder\ extends\ org.apache.avro.specific.SpecificRecordBuilderBase$\mathord{<}$ModeSEncodedMessage$\mathord{>}$\leavevmode\par
{\hllin 184\ }}{\hlstd\ \ \ \ }{\hlstd implements\ org.apache.avro.data.RecordBuilder$\mathord{<}$ModeSEncodedMessage$\mathord{>}$\ $\{$\leavevmode\par
{\hllin 185\ }\leavevmode\par
{\hllin 186\ }}{\hlstd\ \ \ \ }{\hlstd private\ java.lang.CharSequence\ sensorType;\leavevmode\par
{\hllin 187\ }}{\hlstd\ \ \ \ }{\hlstd private\ java.lang.Double\ sensorLatitude;\leavevmode\par
{\hllin 188\ }}{\hlstd\ \ \ \ }{\hlstd private\ java.lang.Double\ sensorLongitude;\leavevmode\par
{\hllin 189\ }}{\hlstd\ \ \ \ }{\hlstd private\ java.lang.Double\ sensorAltitude;\leavevmode\par
{\hllin 190\ }}{\hlstd\ \ \ \ }{\hlstd private\ double\ timeAtServer;\leavevmode\par
{\hllin 191\ }}{\hlstd\ \ \ \ }{\hlstd private\ java.lang.Double\ timeAtSensor;\leavevmode\par
{\hllin 192\ }}{\hlstd\ \ \ \ }{\hlstd private\ java.lang.Double\ timestamp;\leavevmode\par
{\hllin 193\ }}{\hlstd\ \ \ \ }{\hlstd private\ java.lang.CharSequence\ rawMessage;\leavevmode\par
{\hllin 194\ }}{\hlstd\ \ \ \ }{\hlstd private\ int\ sensorSerialNumber;\leavevmode\par
{\hllin 195\ }}{\hlstd\ \ \ \ }{\hlstd private\ java.lang.Double\ RSSIPacket;\leavevmode\par
{\hllin 196\ }}{\hlstd\ \ \ \ }{\hlstd private\ java.lang.Double\ RSSIPreamble;\leavevmode\par
{\hllin 197\ }}{\hlstd\ \ \ \ }{\hlstd private\ java.lang.Double\ SNR;\leavevmode\par
{\hllin 198\ }}{\hlstd\ \ \ \ }{\hlstd private\ java.lang.Double\ confidence;\leavevmode\par
{\hllin 199\ }\leavevmode\par
{\hllin 200\ }}{\hlstd\ \ \ \ }{\hlstd /**\ Creates\ a\ new\ Builder\ */\leavevmode\par
{\hllin 201\ }}{\hlstd\ \ \ \ }{\hlstd private\ Builder()\ $\{$\leavevmode\par
{\hllin 202\ }}{\hlstd\ \ \ \ \ \ }{\hlstd super(org.opensky.example.ModeSEncodedMessage.SCHEMA\${});\leavevmode\par
{\hllin 203\ }}{\hlstd\ \ \ \ }{\hlstd $\}$\leavevmode\par
{\hllin 204\ }}{\hlstd\ \ \ \ }{\hlstd \leavevmode\par
{\hllin 205\ }}{\hlstd\ \ \ \ }{\hlstd /**\ Creates\ a\ Builder\ by\ copying\ an\ existing\ Builder\ */\leavevmode\par
{\hllin 206\ }}{\hlstd\ \ \ \ }{\hlstd private\ Builder(org.opensky.example.ModeSEncodedMessage.Builder\ other)\ $\{$\leavevmode\par
{\hllin 207\ }}{\hlstd\ \ \ \ \ \ }{\hlstd super(other);\leavevmode\par
{\hllin 208\ }}{\hlstd\ \ \ \ \ \ }{\hlstd if\ (isValidValue(fields()[0],\ other.sensorType))\ $\{$\leavevmode\par
{\hllin 209\ }}{\hlstd\ \ \ \ \ \ \ \ }{\hlstd this.sensorType\ $\mathord{=}$\ data().deepCopy(fields()[0].schema(),\ other.sensorType);\leavevmode\par
{\hllin 210\ }}{\hlstd\ \ \ \ \ \ \ \ }{\hlstd fieldSetFlags()[0]\ $\mathord{=}$\ true;\leavevmode\par
{\hllin 211\ }}{\hlstd\ \ \ \ \ \ }{\hlstd $\}$\leavevmode\par
{\hllin 212\ }\leavevmode\par
{\hllin 213\ }...\leavevmode\par
{\hllin 214\ }...\leavevmode\par
{\hllin 215\ }...\leavevmode\par
{\hllin 216\ }}{\hlstd\ \ \ \ \ \ }{\hlstd if\ (isValidValue(fields()[7],\ other.rawMessage))\ $\{$\leavevmode\par
{\hllin 217\ }}{\hlstd\ \ \ \ \ \ \ \ }{\hlstd this.rawMessage\ $\mathord{=}$\ data().deepCopy(fields()[7].schema(),\ other.rawMessage);\leavevmode\par
{\hllin 218\ }}{\hlstd\ \ \ \ \ \ \ \ }{\hlstd fieldSetFlags()[7]\ $\mathord{=}$\ true;\leavevmode\par
{\hllin 219\ }}{\hlstd\ \ \ \ \ \ }{\hlstd $\}$\leavevmode\par
{\hllin 220\ }...\leavevmode\par
{\hllin 221\ }...\leavevmode\par
{\hllin 222\ }...\leavevmode\par
{\hllin 223\ }}{\hlstd\ \ \ \ \ \ }{\hlstd if\ (isValidValue(fields()[12],\ other.confidence))\ $\{$\leavevmode\par
{\hllin 224\ }}{\hlstd\ \ \ \ \ \ \ \ }{\hlstd this.confidence\ $\mathord{=}$\ data().deepCopy(fields()[12].schema(),\ other.confidence);\leavevmode\par
{\hllin 225\ }}{\hlstd\ \ \ \ \ \ \ \ }{\hlstd fieldSetFlags()[12]\ $\mathord{=}$\ true;\leavevmode\par
{\hllin 226\ }}{\hlstd\ \ \ \ \ \ }{\hlstd $\}$\leavevmode\par
{\hllin 227\ }}{\hlstd\ \ \ \ }{\hlstd $\}$\leavevmode\par
{\hllin 228\ }}{\hlstd\ \ \ \ }{\hlstd \leavevmode\par
{\hllin 229\ }}{\hlstd\ \ \ \ }{\hlstd /**\ Creates\ a\ Builder\ by\ copying\ an\ existing\ ModeSEncodedMessage\ instance\ */\leavevmode\par
{\hllin 230\ }}{\hlstd\ \ \ \ }{\hlstd private\ Builder(org.opensky.example.ModeSEncodedMessage\ other)\ $\{$\leavevmode\par
{\hllin 231\ }}{\hlstd\ \ \ \ \ \ \ \ \ \ \ \ }{\hlstd super(org.opensky.example.ModeSEncodedMessage.SCHEMA\${});\leavevmode\par
{\hllin 232\ }}{\hlstd\ \ \ \ \ \ }{\hlstd if\ (isValidValue(fields()[0],\ other.sensorType))\ $\{$\leavevmode\par
{\hllin 233\ }}{\hlstd\ \ \ \ \ \ \ \ }{\hlstd this.sensorType\ $\mathord{=}$\ data().deepCopy(fields()[0].schema(),\ other.sensorType);\leavevmode\par
{\hllin 234\ }}{\hlstd\ \ \ \ \ \ \ \ }{\hlstd fieldSetFlags()[0]\ $\mathord{=}$\ true;\leavevmode\par
{\hllin 235\ }}{\hlstd\ \ \ \ \ \ }{\hlstd $\}$\leavevmode\par
{\hllin 236\ }\leavevmode\par
{\hllin 237\ }...\leavevmode\par
{\hllin 238\ }...\leavevmode\par
{\hllin 239\ }...\leavevmode\par
{\hllin 240\ }}{\hlstd\ \ \ \ \ \ }{\hlstd if\ (isValidValue(fields()[7],\ other.rawMessage))\ $\{$\leavevmode\par
{\hllin 241\ }}{\hlstd\ \ \ \ \ \ \ \ }{\hlstd this.rawMessage\ $\mathord{=}$\ data().deepCopy(fields()[7].schema(),\ other.rawMessage);\leavevmode\par
{\hllin 242\ }}{\hlstd\ \ \ \ \ \ \ \ }{\hlstd fieldSetFlags()[7]\ $\mathord{=}$\ true;\leavevmode\par
{\hllin 243\ }}{\hlstd\ \ \ \ \ \ }{\hlstd $\}$\leavevmode\par
{\hllin 244\ }...\leavevmode\par
{\hllin 245\ }...\leavevmode\par
{\hllin 246\ }...\leavevmode\par
{\hllin 247\ }}{\hlstd\ \ \ \ \ \ }{\hlstd if\ (isValidValue(fields()[12],\ other.confidence))\ $\{$\leavevmode\par
{\hllin 248\ }}{\hlstd\ \ \ \ \ \ \ \ }{\hlstd this.confidence\ $\mathord{=}$\ data().deepCopy(fields()[12].schema(),\ other.confidence);\leavevmode\par
{\hllin 249\ }}{\hlstd\ \ \ \ \ \ \ \ }{\hlstd fieldSetFlags()[12]\ $\mathord{=}$\ true;\leavevmode\par
{\hllin 250\ }}{\hlstd\ \ \ \ \ \ }{\hlstd $\}$\leavevmode\par
{\hllin 251\ }}{\hlstd\ \ \ \ }{\hlstd $\}$\leavevmode\par
{\hllin 252\ }\leavevmode\par
{\hllin 253\ }}{\hlstd\ \ \ \ }{\hlstd /**\ Gets\ the\ value\ of\ the\ 'sensorType'\ field\ */\leavevmode\par
{\hllin 254\ }}{\hlstd\ \ \ \ }{\hlstd public\ java.lang.CharSequence\ getSensorType()\ $\{$\leavevmode\par
{\hllin 255\ }}{\hlstd\ \ \ \ \ \ }{\hlstd return\ sensorType;\leavevmode\par
{\hllin 256\ }}{\hlstd\ \ \ \ }{\hlstd $\}$\leavevmode\par
{\hllin 257\ }}{\hlstd\ \ \ \ }{\hlstd \leavevmode\par
{\hllin 258\ }}{\hlstd\ \ \ \ }{\hlstd /**\ Sets\ the\ value\ of\ the\ 'sensorType'\ field\ */\leavevmode\par
{\hllin 259\ }}{\hlstd\ \ \ \ }{\hlstd public\ org.opensky.example.ModeSEncodedMessage.Builder\ setSensorType(java.lang.CharSequence\ value)\ $\{$\leavevmode\par
{\hllin 260\ }}{\hlstd\ \ \ \ \ \ }{\hlstd validate(fields()[0],\ value);\leavevmode\par
{\hllin 261\ }}{\hlstd\ \ \ \ \ \ }{\hlstd this.sensorType\ $\mathord{=}$\ value;\leavevmode\par
{\hllin 262\ }}{\hlstd\ \ \ \ \ \ }{\hlstd fieldSetFlags()[0]\ $\mathord{=}$\ true;\leavevmode\par
{\hllin 263\ }}{\hlstd\ \ \ \ \ \ }{\hlstd return\ this;\ \leavevmode\par
{\hllin 264\ }}{\hlstd\ \ \ \ }{\hlstd $\}$\leavevmode\par
{\hllin 265\ }}{\hlstd\ \ \ \ }{\hlstd \leavevmode\par
{\hllin 266\ }}{\hlstd\ \ \ \ }{\hlstd /**\ Checks\ whether\ the\ 'sensorType'\ field\ has\ been\ set\ */\leavevmode\par
{\hllin 267\ }}{\hlstd\ \ \ \ }{\hlstd public\ boolean\ hasSensorType()\ $\{$\leavevmode\par
{\hllin 268\ }}{\hlstd\ \ \ \ \ \ }{\hlstd return\ fieldSetFlags()[0];\leavevmode\par
{\hllin 269\ }}{\hlstd\ \ \ \ }{\hlstd $\}$\leavevmode\par
{\hllin 270\ }}{\hlstd\ \ \ \ }{\hlstd \leavevmode\par
{\hllin 271\ }}{\hlstd\ \ \ \ }{\hlstd /**\ Clears\ the\ value\ of\ the\ 'sensorType'\ field\ */\leavevmode\par
{\hllin 272\ }}{\hlstd\ \ \ \ }{\hlstd public\ org.opensky.example.ModeSEncodedMessage.Builder\ clearSensorType()\ $\{$\leavevmode\par
{\hllin 273\ }}{\hlstd\ \ \ \ \ \ }{\hlstd sensorType\ $\mathord{=}$\ null;\leavevmode\par
{\hllin 274\ }}{\hlstd\ \ \ \ \ \ }{\hlstd fieldSetFlags()[0]\ $\mathord{=}$\ false;\leavevmode\par
{\hllin 275\ }}{\hlstd\ \ \ \ \ \ }{\hlstd return\ this;\leavevmode\par
{\hllin 276\ }}{\hlstd\ \ \ \ }{\hlstd $\}$\leavevmode\par
{\hllin 277\ }\leavevmode\par
{\hllin 278\ }...\leavevmode\par
{\hllin 279\ }...\leavevmode\par
{\hllin 280\ }...\leavevmode\par
{\hllin 281\ }\leavevmode\par
{\hllin 282\ }}{\hlstd\ \ \ \ }{\hlstd /**\ Gets\ the\ value\ of\ the\ 'rawMessage'\ field\ */\leavevmode\par
{\hllin 283\ }}{\hlstd\ \ \ \ }{\hlstd public\ java.lang.CharSequence\ getRawMessage()\ $\{$\leavevmode\par
{\hllin 284\ }}{\hlstd\ \ \ \ \ \ }{\hlstd return\ rawMessage;\leavevmode\par
{\hllin 285\ }}{\hlstd\ \ \ \ }{\hlstd $\}$\leavevmode\par
{\hllin 286\ }}{\hlstd\ \ \ \ }{\hlstd \leavevmode\par
{\hllin 287\ }}{\hlstd\ \ \ \ }{\hlstd /**\ Sets\ the\ value\ of\ the\ 'rawMessage'\ field\ */\leavevmode\par
{\hllin 288\ }}{\hlstd\ \ \ \ }{\hlstd public\ org.opensky.example.ModeSEncodedMessage.Builder\ setRawMessage(java.lang.CharSequence\ value)\ $\{$\leavevmode\par
{\hllin 289\ }}{\hlstd\ \ \ \ \ \ }{\hlstd validate(fields()[7],\ value);\leavevmode\par
{\hllin 290\ }}{\hlstd\ \ \ \ \ \ }{\hlstd this.rawMessage\ $\mathord{=}$\ value;\leavevmode\par
{\hllin 291\ }}{\hlstd\ \ \ \ \ \ }{\hlstd fieldSetFlags()[7]\ $\mathord{=}$\ true;\leavevmode\par
{\hllin 292\ }}{\hlstd\ \ \ \ \ \ }{\hlstd return\ this;\ \leavevmode\par
{\hllin 293\ }}{\hlstd\ \ \ \ }{\hlstd $\}$\leavevmode\par
{\hllin 294\ }}{\hlstd\ \ \ \ }{\hlstd \leavevmode\par
{\hllin 295\ }}{\hlstd\ \ \ \ }{\hlstd /**\ Checks\ whether\ the\ 'rawMessage'\ field\ has\ been\ set\ */\leavevmode\par
{\hllin 296\ }}{\hlstd\ \ \ \ }{\hlstd public\ boolean\ hasRawMessage()\ $\{$\leavevmode\par
{\hllin 297\ }}{\hlstd\ \ \ \ \ \ }{\hlstd return\ fieldSetFlags()[7];\leavevmode\par
{\hllin 298\ }}{\hlstd\ \ \ \ }{\hlstd $\}$\leavevmode\par
{\hllin 299\ }}{\hlstd\ \ \ \ }{\hlstd \leavevmode\par
{\hllin 300\ }}{\hlstd\ \ \ \ }{\hlstd /**\ Clears\ the\ value\ of\ the\ 'rawMessage'\ field\ */\leavevmode\par
{\hllin 301\ }}{\hlstd\ \ \ \ }{\hlstd public\ org.opensky.example.ModeSEncodedMessage.Builder\ clearRawMessage()\ $\{$\leavevmode\par
{\hllin 302\ }}{\hlstd\ \ \ \ \ \ }{\hlstd rawMessage\ $\mathord{=}$\ null;\leavevmode\par
{\hllin 303\ }}{\hlstd\ \ \ \ \ \ }{\hlstd fieldSetFlags()[7]\ $\mathord{=}$\ false;\leavevmode\par
{\hllin 304\ }}{\hlstd\ \ \ \ \ \ }{\hlstd return\ this;\leavevmode\par
{\hllin 305\ }}{\hlstd\ \ \ \ }{\hlstd $\}$\leavevmode\par
{\hllin 306\ }\leavevmode\par
{\hllin 307\ }...\leavevmode\par
{\hllin 308\ }...\leavevmode\par
{\hllin 309\ }...\leavevmode\par
{\hllin 310\ }\leavevmode\par
{\hllin 311\ }}{\hlstd\ \ \ \ }{\hlstd /**\ Gets\ the\ value\ of\ the\ 'confidence'\ field\ */\leavevmode\par
{\hllin 312\ }}{\hlstd\ \ \ \ }{\hlstd public\ java.lang.Double\ getConfidence()\ $\{$\leavevmode\par
{\hllin 313\ }}{\hlstd\ \ \ \ \ \ }{\hlstd return\ confidence;\leavevmode\par
{\hllin 314\ }}{\hlstd\ \ \ \ }{\hlstd $\}$\leavevmode\par
{\hllin 315\ }}{\hlstd\ \ \ \ }{\hlstd \leavevmode\par
{\hllin 316\ }}{\hlstd\ \ \ \ }{\hlstd /**\ Sets\ the\ value\ of\ the\ 'confidence'\ field\ */\leavevmode\par
{\hllin 317\ }}{\hlstd\ \ \ \ }{\hlstd public\ org.opensky.example.ModeSEncodedMessage.Builder\ setConfidence(java.lang.Double\ value)\ $\{$\leavevmode\par
{\hllin 318\ }}{\hlstd\ \ \ \ \ \ }{\hlstd validate(fields()[12],\ value);\leavevmode\par
{\hllin 319\ }}{\hlstd\ \ \ \ \ \ }{\hlstd this.confidence\ $\mathord{=}$\ value;\leavevmode\par
{\hllin 320\ }}{\hlstd\ \ \ \ \ \ }{\hlstd fieldSetFlags()[12]\ $\mathord{=}$\ true;\leavevmode\par
{\hllin 321\ }}{\hlstd\ \ \ \ \ \ }{\hlstd return\ this;\ \leavevmode\par
{\hllin 322\ }}{\hlstd\ \ \ \ }{\hlstd $\}$\leavevmode\par
{\hllin 323\ }}{\hlstd\ \ \ \ }{\hlstd \leavevmode\par
{\hllin 324\ }}{\hlstd\ \ \ \ }{\hlstd /**\ Checks\ whether\ the\ 'confidence'\ field\ has\ been\ set\ */\leavevmode\par
{\hllin 325\ }}{\hlstd\ \ \ \ }{\hlstd public\ boolean\ hasConfidence()\ $\{$\leavevmode\par
{\hllin 326\ }}{\hlstd\ \ \ \ \ \ }{\hlstd return\ fieldSetFlags()[12];\leavevmode\par
{\hllin 327\ }}{\hlstd\ \ \ \ }{\hlstd $\}$\leavevmode\par
{\hllin 328\ }}{\hlstd\ \ \ \ }{\hlstd \leavevmode\par
{\hllin 329\ }}{\hlstd\ \ \ \ }{\hlstd /**\ Clears\ the\ value\ of\ the\ 'confidence'\ field\ */\leavevmode\par
{\hllin 330\ }}{\hlstd\ \ \ \ }{\hlstd public\ org.opensky.example.ModeSEncodedMessage.Builder\ clearConfidence()\ $\{$\leavevmode\par
{\hllin 331\ }}{\hlstd\ \ \ \ \ \ }{\hlstd confidence\ $\mathord{=}$\ null;\leavevmode\par
{\hllin 332\ }}{\hlstd\ \ \ \ \ \ }{\hlstd fieldSetFlags()[12]\ $\mathord{=}$\ false;\leavevmode\par
{\hllin 333\ }}{\hlstd\ \ \ \ \ \ }{\hlstd return\ this;\leavevmode\par
{\hllin 334\ }}{\hlstd\ \ \ \ }{\hlstd $\}$\leavevmode\par
{\hllin 335\ }\leavevmode\par
{\hllin 336\ }}{\hlstd\ \ \ \ }{\hlstd @Override\leavevmode\par
{\hllin 337\ }}{\hlstd\ \ \ \ }{\hlstd public\ ModeSEncodedMessage\ build()\ $\{$\leavevmode\par
{\hllin 338\ }}{\hlstd\ \ \ \ \ \ }{\hlstd try\ $\{$\leavevmode\par
{\hllin 339\ }}{\hlstd\ \ \ \ \ \ \ \ }{\hlstd ModeSEncodedMessage\ record\ $\mathord{=}$\ new\ ModeSEncodedMessage();\leavevmode\par
{\hllin 340\ }}{\hlstd\ \ \ \ \ \ \ \ }{\hlstd record.sensorType\ $\mathord{=}$\ fieldSetFlags()[0]\ ?\ this.sensorType\ :\ (java.lang.CharSequence)\ defaultValue(fields()[0]);\leavevmode\par
{\hllin 341\ }}{\hlstd\ \ \ \ \ \ \ \ }{\hlstd record.sensorLatitude\ $\mathord{=}$\ fieldSetFlags()[1]\ ?\ this.sensorLatitude\ :\ (java.lang.Double)\ defaultValue(fields()[1]);\leavevmode\par
{\hllin 342\ }}{\hlstd\ \ \ \ \ \ \ \ }{\hlstd record.sensorLongitude\ $\mathord{=}$\ fieldSetFlags()[2]\ ?\ this.sensorLongitude\ :\ (java.lang.Double)\ defaultValue(fields()[2]);\leavevmode\par
{\hllin 343\ }}{\hlstd\ \ \ \ \ \ \ \ }{\hlstd record.sensorAltitude\ $\mathord{=}$\ fieldSetFlags()[3]\ ?\ this.sensorAltitude\ :\ (java.lang.Double)\ defaultValue(fields()[3]);\leavevmode\par
{\hllin 344\ }}{\hlstd\ \ \ \ \ \ \ \ }{\hlstd record.timeAtServer\ $\mathord{=}$\ fieldSetFlags()[4]\ ?\ this.timeAtServer\ :\ (java.lang.Double)\ defaultValue(fields()[4]);\leavevmode\par
{\hllin 345\ }}{\hlstd\ \ \ \ \ \ \ \ }{\hlstd record.timeAtSensor\ $\mathord{=}$\ fieldSetFlags()[5]\ ?\ this.timeAtSensor\ :\ (java.lang.Double)\ defaultValue(fields()[5]);\leavevmode\par
{\hllin 346\ }}{\hlstd\ \ \ \ \ \ \ \ }{\hlstd record.timestamp\ $\mathord{=}$\ fieldSetFlags()[6]\ ?\ this.timestamp\ :\ (java.lang.Double)\ defaultValue(fields()[6]);\leavevmode\par
{\hllin 347\ }}{\hlstd\ \ \ \ \ \ \ \ }{\hlstd record.rawMessage\ $\mathord{=}$\ fieldSetFlags()[7]\ ?\ this.rawMessage\ :\ (java.lang.CharSequence)\ defaultValue(fields()[7]);\leavevmode\par
{\hllin 348\ }}{\hlstd\ \ \ \ \ \ \ \ }{\hlstd record.sensorSerialNumber\ $\mathord{=}$\ fieldSetFlags()[8]\ ?\ this.sensorSerialNumber\ :\ (java.lang.Integer)\ defaultValue(fields()[8]);\leavevmode\par
{\hllin 349\ }}{\hlstd\ \ \ \ \ \ \ \ }{\hlstd record.RSSIPacket\ $\mathord{=}$\ fieldSetFlags()[9]\ ?\ this.RSSIPacket\ :\ (java.lang.Double)\ defaultValue(fields()[9]);\leavevmode\par
{\hllin 350\ }}{\hlstd\ \ \ \ \ \ \ \ }{\hlstd record.RSSIPreamble\ $\mathord{=}$\ fieldSetFlags()[10]\ ?\ this.RSSIPreamble\ :\ (java.lang.Double)\ defaultValue(fields()[10]);\leavevmode\par
{\hllin 351\ }}{\hlstd\ \ \ \ \ \ \ \ }{\hlstd record.SNR\ $\mathord{=}$\ fieldSetFlags()[11]\ ?\ this.SNR\ :\ (java.lang.Double)\ defaultValue(fields()[11]);\leavevmode\par
{\hllin 352\ }}{\hlstd\ \ \ \ \ \ \ \ }{\hlstd record.confidence\ $\mathord{=}$\ fieldSetFlags()[12]\ ?\ this.confidence\ :\ (java.lang.Double)\ defaultValue(fields()[12]);\leavevmode\par
{\hllin 353\ }}{\hlstd\ \ \ \ \ \ \ \ }{\hlstd return\ record;\leavevmode\par
{\hllin 354\ }}{\hlstd\ \ \ \ \ \ }{\hlstd $\}$\ catch\ (Exception\ e)\ $\{$\leavevmode\par
{\hllin 355\ }}{\hlstd\ \ \ \ \ \ \ \ }{\hlstd throw\ new\ org.apache.avro.AvroRuntimeException(e);\leavevmode\par
{\hllin 356\ }}{\hlstd\ \ \ \ \ \ }{\hlstd $\}$\leavevmode\par
{\hllin 357\ }}{\hlstd\ \ \ \ }{\hlstd $\}$\leavevmode\par
{\hllin 358\ }}{\hlstd\ \ }{\hlstd $\}$\leavevmode\par
{\hllin 359\ }$\}$}\leavevmode\par
}
  

\subsection{解码部分}
step1:确定输入的ADS-B消息是奇数帧还是偶数帧,只有连续的两个ADS-B消息一个为奇数帧一个为偶数帧才利用这两条ADS-B消息进行位置解码,这就是为什么.avro文件在用来解码前要先利用OpenSky提供的函数对.avro文件中的数据进行排序的原因。\newline
step2:再判断该ADS-B消息适用于全球解码或者本地解码,根据其特征对其使用不同的解码方式。

{
\tt
{\hlstd {\hllin 01\ }public\ class\ Decoder\ $\{$\leavevmode\par
{\hllin 02\ }\leavevmode\par
{\hllin 03\ }\ public\ static\ ModeSReply\ genericDecoder\ (String\ raw\_{}message)\ throws\ BadFormatException,\ UnspecifiedFormatError\ $\{$\leavevmode\par
{\hllin 04\ }}{\hlstd\ \ }{\hlstd ModeSReply\ modes\ $\mathord{=}$\ new\ ModeSReply(raw\_{}message);\leavevmode\par
{\hllin 05\ }}{\hlstd\ \ }{\hlstd \leavevmode\par
{\hllin 06\ }}{\hlstd\ \ }{\hlstd if\ (modes.getDownlinkFormat()\ $\mathord{=}$$\mathord{=}$\ 4)\ $\{$\leavevmode\par
{\hllin 07\ }}{\hlstd\ \ \ }{\hlstd return\ new\ AltitudeReply(modes);\leavevmode\par
{\hllin 08\ }}{\hlstd\ \ }{\hlstd $\}$\leavevmode\par
{\hllin 09\ }}{\hlstd\ \ }{\hlstd else\ if\ (modes.getDownlinkFormat()\ $\mathord{=}$$\mathord{=}$\ 5)\ $\{$\leavevmode\par
{\hllin 10\ }}{\hlstd\ \ \ }{\hlstd return\ new\ IdentifyReply(modes);\leavevmode\par
{\hllin 11\ }}{\hlstd\ \ }{\hlstd $\}$\leavevmode\par
{\hllin 12\ }}{\hlstd\ \ }{\hlstd else\ if\ (modes.getDownlinkFormat()\ $\mathord{=}$$\mathord{=}$\ 11)\ $\{$\leavevmode\par
{\hllin 13\ }}{\hlstd\ \ \ }{\hlstd return\ new\ AllCallReply(modes);\leavevmode\par
{\hllin 14\ }}{\hlstd\ \ }{\hlstd $\}$\leavevmode\par
{\hllin 15\ }}{\hlstd\ \ }{\hlstd else\ if\ (modes.getDownlinkFormat()\ $\mathord{=}$$\mathord{=}$\ 17\ ||\ modes.getDownlinkFormat()\ $\mathord{=}$$\mathord{=}$\ 18)\ $\{$\leavevmode\par
{\hllin 16\ }}{\hlstd\ \ \ }{\hlstd ExtendedSquitter\ es1090\ $\mathord{=}$\ new\ ExtendedSquitter(modes);\leavevmode\par
{\hllin 17\ }\leavevmode\par
{\hllin 18\ }}{\hlstd\ \ \ }{\hlstd //\ what\ kind\ of\ extended\ squitter?\leavevmode\par
{\hllin 19\ }}{\hlstd\ \ \ }{\hlstd byte\ ftc\ $\mathord{=}$\ es1090.getFormatTypeCode();\leavevmode\par
{\hllin 20\ }\leavevmode\par
{\hllin 21\ }}{\hlstd\ \ \ }{\hlstd if\ (ftc\ $\mathord{>}$$\mathord{=}$\ 1\ \&{}\&{}\ ftc\ $\mathord{<}$$\mathord{=}$\ 4)\ //\ identification\ message\leavevmode\par
{\hllin 22\ }}{\hlstd\ \ \ \ }{\hlstd return\ new\ IdentificationMsg(es1090);\leavevmode\par
{\hllin 23\ }\leavevmode\par
{\hllin 24\ }}{\hlstd\ \ \ }{\hlstd if\ (ftc\ $\mathord{>}$$\mathord{=}$\ 5\ \&{}\&{}\ ftc\ $\mathord{<}$$\mathord{=}$\ 8)\ //\ surface\ position\ message\leavevmode\par
{\hllin 25\ }}{\hlstd\ \ \ \ }{\hlstd return\ new\ SurfacePositionMsg(es1090);\leavevmode\par
{\hllin 26\ }\leavevmode\par
{\hllin 27\ }}{\hlstd\ \ \ }{\hlstd if\ ((ftc\ $\mathord{>}$$\mathord{=}$\ 9\ \&{}\&{}\ ftc\ $\mathord{<}$$\mathord{=}$\ 18)\ ||\ (ftc\ $\mathord{>}$$\mathord{=}$\ 20\ \&{}\&{}\ ftc\ $\mathord{<}$$\mathord{=}$\ 22))\ //\ airborne\ position\ message\leavevmode\par
{\hllin 28\ }}{\hlstd\ \ \ \ }{\hlstd return\ new\ AirbornePositionMsg(es1090);\leavevmode\par
{\hllin 29\ }\leavevmode\par
{\hllin 30\ }}{\hlstd\ \ \ }{\hlstd if\ (ftc\ $\mathord{=}$$\mathord{=}$\ 19)\ $\{$\ //\ possible\ velocity\ message,\ check\ subtype\leavevmode\par
{\hllin 31\ }}{\hlstd\ \ \ \ }{\hlstd int\ subtype\ $\mathord{=}$\ es1090.getMessage()[0]\&{}0x7;\leavevmode\par
{\hllin 32\ }\leavevmode\par
{\hllin 33\ }}{\hlstd\ \ \ \ }{\hlstd if\ (subtype\ $\mathord{=}$$\mathord{=}$\ 1\ ||\ subtype\ $\mathord{=}$$\mathord{=}$\ 2)\ //\ velocity\ over\ ground\leavevmode\par
{\hllin 34\ }}{\hlstd\ \ \ \ \ }{\hlstd return\ new\ VelocityOverGroundMsg(es1090);\leavevmode\par
{\hllin 35\ }}{\hlstd\ \ \ \ }{\hlstd else\ if\ (subtype\ $\mathord{=}$$\mathord{=}$\ 3\ ||\ subtype\ $\mathord{=}$$\mathord{=}$\ 4)\ //\ airspeed\ \&{}\ heading\leavevmode\par
{\hllin 36\ }}{\hlstd\ \ \ \ \ }{\hlstd return\ new\ AirspeedHeadingMsg(es1090);\leavevmode\par
{\hllin 37\ }}{\hlstd\ \ \ }{\hlstd $\}$\leavevmode\par
{\hllin 38\ }\leavevmode\par
{\hllin 39\ }}{\hlstd\ \ \ }{\hlstd if\ (ftc\ $\mathord{=}$$\mathord{=}$\ 28)\ $\{$\ //\ aircraft\ status\ message,\ check\ subtype\leavevmode\par
{\hllin 40\ }}{\hlstd\ \ \ \ }{\hlstd int\ subtype\ $\mathord{=}$\ es1090.getMessage()[0]\&{}0x7;\leavevmode\par
{\hllin 41\ }\leavevmode\par
{\hllin 42\ }}{\hlstd\ \ \ \ }{\hlstd if\ (subtype\ $\mathord{=}$$\mathord{=}$\ 1)\ //\ emergency/priority\ status\leavevmode\par
{\hllin 43\ }}{\hlstd\ \ \ \ \ }{\hlstd return\ new\ EmergencyOrPriorityStatusMsg(es1090);\leavevmode\par
{\hllin 44\ }}{\hlstd\ \ \ \ }{\hlstd if\ (subtype\ $\mathord{=}$$\mathord{=}$\ 2)\ //\ TCAS\ resolution\ advisory\ report\leavevmode\par
{\hllin 45\ }}{\hlstd\ \ \ \ \ }{\hlstd return\ new\ TCASResolutionAdvisoryMsg(es1090);\leavevmode\par
{\hllin 46\ }}{\hlstd\ \ \ }{\hlstd $\}$\leavevmode\par
{\hllin 47\ }\leavevmode\par
{\hllin 48\ }}{\hlstd\ \ \ }{\hlstd if\ (ftc\ $\mathord{=}$$\mathord{=}$\ 31)\ $\{$\ //\ operational\ status\ message\leavevmode\par
{\hllin 49\ }}{\hlstd\ \ \ \ }{\hlstd int\ subtype\ $\mathord{=}$\ es1090.getMessage()[0]\&{}0x7;\leavevmode\par
{\hllin 50\ }\leavevmode\par
{\hllin 51\ }}{\hlstd\ \ \ \ }{\hlstd if\ (subtype\ $\mathord{=}$$\mathord{=}$\ 0\ ||\ subtype\ $\mathord{=}$$\mathord{=}$\ 1)\ //\ airborne\ or\ surface?\leavevmode\par
{\hllin 52\ }}{\hlstd\ \ \ \ \ }{\hlstd return\ new\ OperationalStatusMsg(es1090);\leavevmode\par
{\hllin 53\ }}{\hlstd\ \ \ }{\hlstd $\}$\leavevmode\par
{\hllin 54\ }\leavevmode\par
{\hllin 55\ }}{\hlstd\ \ \ }{\hlstd return\ es1090;\ //\ unknown\ extended\ squitter\leavevmode\par
{\hllin 56\ }}{\hlstd\ \ }{\hlstd $\}$\leavevmode\par
{\hllin 57\ }}{\hlstd\ \ }{\hlstd else\ if\ (modes.getDownlinkFormat()\ $\mathord{=}$$\mathord{=}$\ 20)\ $\{$\leavevmode\par
{\hllin 58\ }}{\hlstd\ \ \ }{\hlstd return\ new\ CommBAltitudeReply(modes);\leavevmode\par
{\hllin 59\ }}{\hlstd\ \ }{\hlstd $\}$\leavevmode\par
{\hllin 60\ }}{\hlstd\ \ }{\hlstd else\ if\ (modes.getDownlinkFormat()\ $\mathord{=}$$\mathord{=}$\ 21)\ $\{$\leavevmode\par
{\hllin 61\ }}{\hlstd\ \ \ }{\hlstd return\ new\ CommBIdentifyReply(modes);\leavevmode\par
{\hllin 62\ }}{\hlstd\ \ }{\hlstd $\}$\leavevmode\par
{\hllin 63\ }\leavevmode\par
{\hllin 64\ }}{\hlstd\ \ }{\hlstd return\ modes;\ //\ unknown\ mode\ s\ reply\leavevmode\par
{\hllin 65\ }\ $\}$\leavevmode\par
{\hllin 66\ }\leavevmode\par
{\hllin 67\ }$\}$}\leavevmode\par
}


\subsubsection{全球解码}

{
\tt
{\hlstd {\hllin 01\ }public\ Position\ getGlobalPosition(SurfacePositionMsg\ other)\ throws\ MissingInformationException,\ \leavevmode\par
{\hllin 02\ }\ PositionStraddleError,\ BadFormatException\ $\{$\leavevmode\par
{\hllin 03\ }\ if\ (!tools.areEqual(other.getIcao24(),\ getIcao24()))\leavevmode\par
{\hllin 04\ }}{\hlstd\ \ \ }{\hlstd throw\ new\ IllegalArgumentException(\leavevmode\par
{\hllin 05\ }}{\hlstd\ \ \ \ \ }{\hlstd String.format("Transmitter\ of\ other\ message\ (\%{}s)\ not\ equal\ to\ this\ (\%{}s).",\leavevmode\par
{\hllin 06\ }}{\hlstd\ \ \ \ \ }{\hlstd tools.toHexString(other.getIcao24()),\ tools.toHexString(this.getIcao24())));\leavevmode\par
{\hllin 07\ }}{\hlstd\ \ }{\hlstd \leavevmode\par
{\hllin 08\ }\ if\ (other.isOddFormat()\ $\mathord{=}$$\mathord{=}$\ this.isOddFormat())\leavevmode\par
{\hllin 09\ }}{\hlstd\ \ }{\hlstd throw\ new\ BadFormatException("Expected\ "$\mathord{+}$(isOddFormat()?"even":"odd")$\mathord{+}$"\ message\ format.",\ other.toString());\leavevmode\par
{\hllin 10\ }\leavevmode\par
{\hllin 11\ }\ if\ (!horizontal\_{}position\_{}available)\leavevmode\par
{\hllin 12\ }}{\hlstd\ \ }{\hlstd throw\ new\ MissingInformationException("No\ position\ information\ available!");\leavevmode\par
{\hllin 13\ }\ if\ (!other.hasPosition())\leavevmode\par
{\hllin 14\ }}{\hlstd\ \ }{\hlstd throw\ new\ MissingInformationException("Other\ message\ has\ no\ position\ information.");\leavevmode\par
{\hllin 15\ }\leavevmode\par
{\hllin 16\ }\ SurfacePositionMsg\ even\ $\mathord{=}$\ isOddFormat()?other:this;\leavevmode\par
{\hllin 17\ }\ SurfacePositionMsg\ odd\ $\mathord{=}$\ isOddFormat()?this:other;\leavevmode\par
{\hllin 18\ }\leavevmode\par
{\hllin 19\ }\ //\ Helper\ for\ latitude\ single(Number\ of\ zones\ NZ\ is\ set\ to\ 15)\leavevmode\par
{\hllin 20\ }\ double\ Dlat0\ $\mathord{=}$\ 90.0/60.0;\leavevmode\par
{\hllin 21\ }\ double\ Dlat1\ $\mathord{=}$\ 90.0/59.0;\leavevmode\par
{\hllin 22\ }\leavevmode\par
{\hllin 23\ }\ //\ latitude\ index\leavevmode\par
{\hllin 24\ }\ double\ j\ $\mathord{=}$\ Math.floor((59.0*even.getCPREncodedLatitude()$\mathord{-}$60.0*odd.getCPREncodedLatitude())/((double)(1$\mathord{<}$$\mathord{<}$17))$\mathord{+}$0.5);\leavevmode\par
{\hllin 25\ }\leavevmode\par
{\hllin 26\ }\ //\ global\ latitudes\leavevmode\par
{\hllin 27\ }\ double\ Rlat0\ $\mathord{=}$\ Dlat0\ *\ (mod(j,60)$\mathord{+}$even.getCPREncodedLatitude()/((double)(1$\mathord{<}$$\mathord{<}$17)));\leavevmode\par
{\hllin 28\ }\ double\ Rlat1\ $\mathord{=}$\ Dlat1\ *\ (mod(j,59)$\mathord{+}$odd.getCPREncodedLatitude()/((double)(1$\mathord{<}$$\mathord{<}$17)));\leavevmode\par
{\hllin 29\ }\leavevmode\par
{\hllin 30\ }\ //\ Southern\ hemisphere?\leavevmode\par
{\hllin 31\ }\ if\ (Rlat0\ $\mathord{>}$$\mathord{=}$\ 270\ \&{}\&{}\ Rlat0\ $\mathord{<}$$\mathord{=}$\ 360)\ Rlat0\ $\mathord{-}$$\mathord{=}$\ 360;\leavevmode\par
{\hllin 32\ }\ if\ (Rlat1\ $\mathord{>}$$\mathord{=}$\ 270\ \&{}\&{}\ Rlat1\ $\mathord{<}$$\mathord{=}$\ 360)\ Rlat1\ $\mathord{-}$$\mathord{=}$\ 360;\leavevmode\par
{\hllin 33\ }\leavevmode\par
{\hllin 34\ }\ //\ Northern\ hemisphere?\leavevmode\par
{\hllin 35\ }\ if\ (Rlat0\ $\mathord{<}$$\mathord{=}$\ $\mathord{-}$270\ \&{}\&{}\ Rlat0\ $\mathord{>}$$\mathord{=}$\ $\mathord{-}$360)\ Rlat0\ $\mathord{+}$$\mathord{=}$\ 360;\leavevmode\par
{\hllin 36\ }\ if\ (Rlat1\ $\mathord{<}$$\mathord{=}$\ $\mathord{-}$270\ \&{}\&{}\ Rlat1\ $\mathord{>}$$\mathord{=}$\ $\mathord{-}$360)\ Rlat1\ $\mathord{+}$$\mathord{=}$\ 360;\leavevmode\par
{\hllin 37\ }\leavevmode\par
{\hllin 38\ }\ //\ ensure\ that\ the\ number\ of\ even\ longitude\ zones\ are\ equal\leavevmode\par
{\hllin 39\ }\ if\ (NL(Rlat0)\ !$\mathord{=}$\ NL(Rlat1))\leavevmode\par
{\hllin 40\ }}{\hlstd\ \ }{\hlstd throw\ new\ org.opensky.libadsb.exceptions.PositionStraddleError(\leavevmode\par
{\hllin 41\ }}{\hlstd\ \ \ }{\hlstd "The\ two\ given\ position\ straddle\ a\ transition\ latitude\ "$\mathord{+}$\leavevmode\par
{\hllin 42\ }}{\hlstd\ \ \ }{\hlstd "and\ cannot\ be\ decoded.\ Wait\ for\ positions\ where\ they\ are\ equal.");\leavevmode\par
{\hllin 43\ }\leavevmode\par
{\hllin 44\ }\ //\ Helper\ for\ longitude\leavevmode\par
{\hllin 45\ }\ double\ Dlon0\ $\mathord{=}$\ 90.0/Math.max(1.0,\ NL(Rlat0));\leavevmode\par
{\hllin 46\ }\ double\ Dlon1\ $\mathord{=}$\ 90.0/Math.max(1.0,\ NL(Rlat1)$\mathord{-}$1);\leavevmode\par
{\hllin 47\ }\leavevmode\par
{\hllin 48\ }\ //\ longitude\ index\leavevmode\par
{\hllin 49\ }\ double\ NL\_{}helper\ $\mathord{=}$\ NL(isOddFormat()?Rlat1:Rlat0);\ //\ assuming\ that\ this\ is\ the\ newer\ message\leavevmode\par
{\hllin 50\ }\ double\ m\ $\mathord{=}$\ Math.floor((even.getCPREncodedLongitude()*(NL\_{}helper$\mathord{-}$1)$\mathord{-}$odd.getCPREncodedLongitude()*NL\_{}helper)/((double)(1$\mathord{<}$$\mathord{<}$17))$\mathord{+}$0.5);\leavevmode\par
{\hllin 51\ }\leavevmode\par
{\hllin 52\ }\ //\ global\ longitude\leavevmode\par
{\hllin 53\ }\ double\ Rlon0\ $\mathord{=}$\ Dlon0\ *\ (mod(m,Math.max(1.0,\ NL(Rlat0)))\ $\mathord{+}$\ even.getCPREncodedLongitude()/((double)(1$\mathord{<}$$\mathord{<}$17)));\leavevmode\par
{\hllin 54\ }\ double\ Rlon1\ $\mathord{=}$\ Dlon1\ *\ (mod(m,Math.max(1.0,\ NL(Rlat1)$\mathord{-}$1))\ $\mathord{+}$\ odd.getCPREncodedLongitude()/((double)(1$\mathord{<}$$\mathord{<}$17)));\leavevmode\par
{\hllin 55\ }\leavevmode\par
{\hllin 56\ }\ //\ correct\ longitude\leavevmode\par
{\hllin 57\ }\ if\ (Rlon0\ $\mathord{<}$\ $\mathord{-}$180\ \&{}\&{}\ Rlon0\ $\mathord{>}$\ $\mathord{-}$360)\ Rlon0\ $\mathord{+}$$\mathord{=}$\ 360;\leavevmode\par
{\hllin 58\ }\ if\ (Rlon1\ $\mathord{<}$\ $\mathord{-}$180\ \&{}\&{}\ Rlon1\ $\mathord{>}$\ $\mathord{-}$360)\ Rlon1\ $\mathord{+}$$\mathord{=}$\ 360;\leavevmode\par
{\hllin 59\ }\ if\ (Rlon0\ $\mathord{>}$\ 180\ \&{}\&{}\ Rlon0\ $\mathord{<}$\ 360)\ Rlon0\ $\mathord{-}$$\mathord{=}$\ 360;\leavevmode\par
{\hllin 60\ }\ if\ (Rlon1\ $\mathord{>}$\ 180\ \&{}\&{}\ Rlon1\ $\mathord{<}$\ 360)\ Rlon1\ $\mathord{-}$$\mathord{=}$\ 360;\leavevmode\par
{\hllin 61\ }\leavevmode\par
{\hllin 62\ }\ return\ new\ Position(isOddFormat()?Rlon1:Rlon0,\leavevmode\par
{\hllin 63\ }}{\hlstd\ \ \ \ \ \ }{\hlstd isOddFormat()?Rlat1:Rlat0,\leavevmode\par
{\hllin 64\ }}{\hlstd\ \ \ \ \ \ }{\hlstd 0.0);\leavevmode\par
{\hllin 65\ }$\}$}\leavevmode\par
}

			
\subsubsection{本地解码}

{
\tt
{\hlstd {\hllin 01\ }public\ Position\ getLocalPosition(Position\ ref)\ throws\ MissingInformationException\ $\{$\leavevmode\par
{\hllin 02\ }\ if\ (!horizontal\_{}position\_{}available)\leavevmode\par
{\hllin 03\ }}{\hlstd\ \ }{\hlstd throw\ new\ MissingInformationException("No\ position\ information\ available!");\leavevmode\par
{\hllin 04\ }}{\hlstd\ \ }{\hlstd \leavevmode\par
{\hllin 05\ }\ //\ latitude\ zone\ size\leavevmode\par
{\hllin 06\ }\ double\ Dlat\ $\mathord{=}$\ isOddFormat()\ ?\ 90.0/59.0\ :\ 90.0/60.0;\leavevmode\par
{\hllin 07\ }}{\hlstd\ \ }{\hlstd \leavevmode\par
{\hllin 08\ }\ //\ latitude\ zone\ index\leavevmode\par
{\hllin 09\ }\ double\ j\ $\mathord{=}$\ Math.floor(ref.getLatitude()/Dlat)\ $\mathord{+}$\ Math.floor(0.5$\mathord{+}$(mod(ref.getLatitude(),\ Dlat))/Dlat$\mathord{-}$getCPREncodedLatitude()/((double)(1$\mathord{<}$$\mathord{<}$17)));\leavevmode\par
{\hllin 10\ }}{\hlstd\ \ }{\hlstd \leavevmode\par
{\hllin 11\ }\ //\ decoded\ position\ latitude\leavevmode\par
{\hllin 12\ }\ double\ Rlat\ $\mathord{=}$\ Dlat*(j$\mathord{+}$getCPREncodedLatitude()/((double)(1$\mathord{<}$$\mathord{<}$17)));\leavevmode\par
{\hllin 13\ }}{\hlstd\ \ }{\hlstd \leavevmode\par
{\hllin 14\ }\ //\ longitude\ zone\ size\leavevmode\par
{\hllin 15\ }\ double\ Dlon\ $\mathord{=}$\ 90.0/Math.max(1.0,\ NL(Rlat)$\mathord{-}$(isOddFormat()?1.0:0.0));\leavevmode\par
{\hllin 16\ }}{\hlstd\ \ }{\hlstd \leavevmode\par
{\hllin 17\ }\ //\ longitude\ zone\ coordinate\leavevmode\par
{\hllin 18\ }\ double\ m\ $\mathord{=}$\leavevmode\par
{\hllin 19\ }}{\hlstd\ \ \ }{\hlstd Math.floor(ref.getLongitude()/Dlon)\ $\mathord{+}$\leavevmode\par
{\hllin 20\ }}{\hlstd\ \ \ }{\hlstd Math.floor(0.5$\mathord{+}$(mod(ref.getLongitude(),Dlon))/Dlon$\mathord{-}$(float)getCPREncodedLongitude()/((double)(1$\mathord{<}$$\mathord{<}$17)));\leavevmode\par
{\hllin 21\ }}{\hlstd\ \ }{\hlstd \leavevmode\par
{\hllin 22\ }\ //\ and\ finally\ the\ longitude\leavevmode\par
{\hllin 23\ }\ double\ Rlon\ $\mathord{=}$\ Dlon\ *\ (m\ $\mathord{+}$\ getCPREncodedLongitude()/((double)(1$\mathord{<}$$\mathord{<}$17)));\leavevmode\par
{\hllin 24\ }}{\hlstd\ \ }{\hlstd \leavevmode\par
{\hllin 25\ }//}{\hlstd\ \ }{\hlstd System.out.println("Loc:\ EncLon:\ "$\mathord{+}$getCPREncodedLongitude()$\mathord{+}$\leavevmode\par
{\hllin 26\ }//}{\hlstd\ \ \ \ }{\hlstd "\ m:\ "$\mathord{+}$m$\mathord{+}$"\ Dlon:\ "$\mathord{+}$Dlon$\mathord{+}$\ "\ Rlon2:\ "$\mathord{+}$Rlon2);\leavevmode\par
{\hllin 27\ }}{\hlstd\ \ }{\hlstd \leavevmode\par
{\hllin 28\ }\ return\ new\ Position(Rlon,\ Rlat,\ 0.0);\leavevmode\par
{\hllin 29\ }$\}$}\leavevmode\par
}


\subsection{可视化文件类型转换部分}

{
\tt
{\hlstd {\hllin 01\ }public\ class\ Avro2Kml\ $\{$}{\hlstd\ \ }{\hlstd \leavevmode\par
{\hllin 02\ }\ /**\leavevmode\par
{\hllin 03\ }}{\hlstd\ \ }{\hlstd *\ This\ class\ is\ a\ container\ for\ all\ information\leavevmode\par
{\hllin 04\ }}{\hlstd\ \ }{\hlstd *\ about\ flights\ that\ are\ relevant\ for\ the\ KML\leavevmode\par
{\hllin 05\ }}{\hlstd\ \ }{\hlstd *\ generation\leavevmode\par
{\hllin 06\ }}{\hlstd\ \ }{\hlstd */\leavevmode\par
{\hllin 07\ }\ private\ class\ Flight\ $\{$\leavevmode\par
{\hllin 08\ }}{\hlstd\ \ }{\hlstd public\ String\ icao24;\leavevmode\par
{\hllin 09\ }}{\hlstd\ \ }{\hlstd public\ char[]\ callsign;\leavevmode\par
{\hllin 10\ }}{\hlstd\ \ }{\hlstd public\ double\ first;\ //\ first\ message\ seen\leavevmode\par
{\hllin 11\ }}{\hlstd\ \ }{\hlstd public\ double\ last;\ //\ last\ message\ seen\leavevmode\par
{\hllin 12\ }}{\hlstd\ \ }{\hlstd public\ List$\mathord{<}$Coordinate$\mathord{>}$\ coords;\ //\ ordered\ list\ of\ coordinates\leavevmode\par
{\hllin 13\ }}{\hlstd\ \ }{\hlstd public\ List$\mathord{<}$Integer$\mathord{>}$\ serials;\ //\ flight\ seen\ by\ these\ sensors\leavevmode\par
{\hllin 14\ }}{\hlstd\ \ }{\hlstd public\ PositionDecoder\ dec;\ //\ stateful\ position\ decoder\leavevmode\par
{\hllin 15\ }}{\hlstd\ \ }{\hlstd boolean\ contains\_{}unreasonable;\ //\ true\ if\ there\ was\ one\ position\ with\ unreasonable\ flag\leavevmode\par
{\hllin 16\ }}{\hlstd\ \ }{\hlstd \leavevmode\par
{\hllin 17\ }}{\hlstd\ \ }{\hlstd public\ Flight\ ()\ $\{$\leavevmode\par
{\hllin 18\ }}{\hlstd\ \ \ }{\hlstd coords\ $\mathord{=}$\ new\ ArrayList$\mathord{<}$Coordinate$\mathord{>}$();\leavevmode\par
{\hllin 19\ }}{\hlstd\ \ \ }{\hlstd dec\ $\mathord{=}$\ new\ PositionDecoder();\leavevmode\par
{\hllin 20\ }}{\hlstd\ \ \ }{\hlstd callsign\ $\mathord{=}$\ new\ char[0];\leavevmode\par
{\hllin 21\ }}{\hlstd\ \ \ }{\hlstd contains\_{}unreasonable\ $\mathord{=}$\ false;\leavevmode\par
{\hllin 22\ }}{\hlstd\ \ \ }{\hlstd serials\ $\mathord{=}$\ new\ ArrayList$\mathord{<}$Integer$\mathord{>}$();\leavevmode\par
{\hllin 23\ }}{\hlstd\ \ }{\hlstd $\}$\leavevmode\par
{\hllin 24\ }\ $\}$\leavevmode\par
{\hllin 25\ }\ \leavevmode\par
{\hllin 26\ }\ /**\leavevmode\par
{\hllin 27\ }}{\hlstd\ \ }{\hlstd *\ Class\ for\ generating\ the\ kml\leavevmode\par
{\hllin 28\ }}{\hlstd\ \ }{\hlstd */\leavevmode\par
{\hllin 29\ }\ private\ class\ OskyKml\ $\{$\leavevmode\par
{\hllin 30\ }}{\hlstd\ \ }{\hlstd private\ Kml\ kml;\leavevmode\par
{\hllin 31\ }}{\hlstd\ \ }{\hlstd private\ Document\ doc;\leavevmode\par
{\hllin 32\ }}{\hlstd\ \ }{\hlstd private\ Style\ style;\leavevmode\par
{\hllin 33\ }}{\hlstd\ \ }{\hlstd private\ SimpleDateFormat\ date\_{}formatter;\leavevmode\par
{\hllin 34\ }}{\hlstd\ \ }{\hlstd private\ Folder\ unreasonable;\leavevmode\par
{\hllin 35\ }}{\hlstd\ \ }{\hlstd private\ Folder\ reasonable;\leavevmode\par
{\hllin 36\ }}{\hlstd\ \ }{\hlstd private\ Folder\ empty;\leavevmode\par
{\hllin 37\ }}{\hlstd\ \ }{\hlstd private\ int\ num\_{}flights;\leavevmode\par
{\hllin 38\ }}{\hlstd\ \ }{\hlstd \leavevmode\par
{\hllin 39\ }}{\hlstd\ \ }{\hlstd public\ OskyKml\ ()\ $\{$\leavevmode\par
{\hllin 40\ }}{\hlstd\ \ \ }{\hlstd //\ prepare\ KML\leavevmode\par
{\hllin 41\ }}{\hlstd\ \ \ }{\hlstd kml\ $\mathord{=}$\ new\ Kml();\leavevmode\par
{\hllin 42\ }}{\hlstd\ \ \ }{\hlstd doc\ $\mathord{=}$\ kml.createAndSetDocument()\leavevmode\par
{\hllin 43\ }}{\hlstd\ \ \ \ \ }{\hlstd .withName("OpenSky\ Network");\leavevmode\par
{\hllin 44\ }}{\hlstd\ \ \ }{\hlstd \leavevmode\par
{\hllin 45\ }}{\hlstd\ \ \ }{\hlstd style\ $\mathord{=}$\ doc.createAndAddStyle()\leavevmode\par
{\hllin 46\ }}{\hlstd\ \ \ \ \ }{\hlstd .withId("reasonable");\leavevmode\par
{\hllin 47\ }}{\hlstd\ \ \ }{\hlstd style.createAndSetLineStyle()\leavevmode\par
{\hllin 48\ }}{\hlstd\ \ \ }{\hlstd .withColor("ffffffff")\leavevmode\par
{\hllin 49\ }}{\hlstd\ \ \ }{\hlstd .withColorMode(ColorMode.NORMAL)\leavevmode\par
{\hllin 50\ }}{\hlstd\ \ \ }{\hlstd .withWidth(1);\leavevmode\par
{\hllin 51\ }\leavevmode\par
{\hllin 52\ }}{\hlstd\ \ \ }{\hlstd style\ $\mathord{=}$\ doc.createAndAddStyle()\leavevmode\par
{\hllin 53\ }}{\hlstd\ \ \ \ \ }{\hlstd .withId("unreasonable");\leavevmode\par
{\hllin 54\ }}{\hlstd\ \ \ }{\hlstd style.createAndSetLineStyle()\leavevmode\par
{\hllin 55\ }}{\hlstd\ \ \ }{\hlstd .withColor("ffd5d5ff")\leavevmode\par
{\hllin 56\ }}{\hlstd\ \ \ }{\hlstd .withColorMode(ColorMode.NORMAL)\leavevmode\par
{\hllin 57\ }}{\hlstd\ \ \ }{\hlstd .withWidth(1);\leavevmode\par
{\hllin 58\ }}{\hlstd\ \ \ }{\hlstd \leavevmode\par
{\hllin 59\ }}{\hlstd\ \ \ }{\hlstd unreasonable\ $\mathord{=}$\ doc.createAndAddFolder()\leavevmode\par
{\hllin 60\ }}{\hlstd\ \ \ \ \ }{\hlstd .withName("Unreasonable\ Flights");\leavevmode\par
{\hllin 61\ }}{\hlstd\ \ \ }{\hlstd reasonable\ $\mathord{=}$\ doc.createAndAddFolder()\leavevmode\par
{\hllin 62\ }}{\hlstd\ \ \ \ \ }{\hlstd .withName("Reasonable\ Flights");\leavevmode\par
{\hllin 63\ }}{\hlstd\ \ \ }{\hlstd empty\ $\mathord{=}$\ doc.createAndAddFolder()\leavevmode\par
{\hllin 64\ }}{\hlstd\ \ \ \ \ }{\hlstd .withName("No\ Positions");\leavevmode\par
{\hllin 65\ }}{\hlstd\ \ \ }{\hlstd \leavevmode\par
{\hllin 66\ }}{\hlstd\ \ \ }{\hlstd date\_{}formatter\ $\mathord{=}$\ new\ SimpleDateFormat("yyyy$\mathord{-}$MM$\mathord{-}$dd'T'HH:mm:ssZ");\leavevmode\par
{\hllin 67\ }}{\hlstd\ \ }{\hlstd $\}$\leavevmode\par
{\hllin 68\ }}{\hlstd\ \ }{\hlstd \leavevmode\par
{\hllin 69\ }}{\hlstd\ \ }{\hlstd public\ void\ addFlight(Flight\ flight)\ $\{$\leavevmode\par
{\hllin 70\ }}{\hlstd\ \ \ }{\hlstd Date\ begin\ $\mathord{=}$\ new\ Date((long)(flight.first*1000));\leavevmode\par
{\hllin 71\ }}{\hlstd\ \ \ }{\hlstd Date\ end\ $\mathord{=}$\ new\ Date((long)(flight.last*1000));\leavevmode\par
{\hllin 72\ }}{\hlstd\ \ \ }{\hlstd \leavevmode\par
{\hllin 73\ }}{\hlstd\ \ \ }{\hlstd Folder\ which;\leavevmode\par
{\hllin 74\ }}{\hlstd\ \ \ }{\hlstd if\ (flight.coords.size()$\mathord{>}$0)\leavevmode\par
{\hllin 75\ }}{\hlstd\ \ \ \ }{\hlstd which\ $\mathord{=}$\ flight.contains\_{}unreasonable\ ?\ unreasonable\ :\ reasonable;\leavevmode\par
{\hllin 76\ }}{\hlstd\ \ \ }{\hlstd else\ which\ $\mathord{=}$\ empty;\leavevmode\par
{\hllin 77\ }}{\hlstd\ \ \ }{\hlstd \leavevmode\par
{\hllin 78\ }}{\hlstd\ \ \ }{\hlstd String\ description\ $\mathord{=}$\ "ICAO:\ "$\mathord{+}$flight.icao24$\mathord{+}$"$\mathord{<}$br\ /$\mathord{>}$$\backslash$n"$\mathord{+}$\leavevmode\par
{\hllin 79\ }}{\hlstd\ \ \ \ \ \ \ }{\hlstd "Callsign:\ "$\mathord{+}$new\ String(flight.callsign)$\mathord{+}$"$\mathord{<}$br\ /$\mathord{>}$$\backslash$n"$\mathord{+}$\leavevmode\par
{\hllin 80\ }}{\hlstd\ \ \ \ \ \ \ }{\hlstd "First\ seen:\ "$\mathord{+}$begin.toString()$\mathord{+}$"$\mathord{<}$br\ /$\mathord{>}$$\backslash$n"$\mathord{+}$\leavevmode\par
{\hllin 81\ }}{\hlstd\ \ \ \ \ \ \ }{\hlstd "Last\ seen:\ "$\mathord{+}$end.toString()$\mathord{+}$"$\mathord{<}$br\ /$\mathord{>}$$\backslash$n"$\mathord{+}$\leavevmode\par
{\hllin 82\ }}{\hlstd\ \ \ \ \ \ \ }{\hlstd "Seen\ by\ serials:\ "$\mathord{+}$StringUtils.join(flight.serials,\ ",");\leavevmode\par
{\hllin 83\ }}{\hlstd\ \ \ }{\hlstd \leavevmode\par
{\hllin 84\ }}{\hlstd\ \ \ }{\hlstd Placemark\ placemark\ $\mathord{=}$\ which.createAndAddPlacemark()\leavevmode\par
{\hllin 85\ }}{\hlstd\ \ \ \ \ }{\hlstd .withName(flight.icao24)\leavevmode\par
{\hllin 86\ }}{\hlstd\ \ \ \ \ }{\hlstd .withTimePrimitive(new\ TimeSpan()\leavevmode\par
{\hllin 87\ }}{\hlstd\ \ \ \ \ \ \ }{\hlstd .withBegin(date\_{}formatter.format(begin))\leavevmode\par
{\hllin 88\ }}{\hlstd\ \ \ \ \ \ \ }{\hlstd .withEnd(date\_{}formatter.format(end)))\leavevmode\par
{\hllin 89\ }}{\hlstd\ \ \ \ \ }{\hlstd .withDescription(description)\leavevmode\par
{\hllin 90\ }}{\hlstd\ \ \ \ \ }{\hlstd .withStyleUrl(flight.contains\_{}unreasonable\ ?\ "\#{}unreasonable"\ :\ "\#{}reasonable");\leavevmode\par
{\hllin 91\ }}{\hlstd\ \ \ }{\hlstd \leavevmode\par
{\hllin 92\ }}{\hlstd\ \ \ }{\hlstd placemark.createAndSetLineString()\leavevmode\par
{\hllin 93\ }}{\hlstd\ \ \ \ }{\hlstd .withCoordinates(flight.coords)\leavevmode\par
{\hllin 94\ }}{\hlstd\ \ \ \ }{\hlstd .withAltitudeMode(AltitudeMode.fromValue(AltitudeMode.ABSOLUTE.value()))\leavevmode\par
{\hllin 95\ }}{\hlstd\ \ \ \ }{\hlstd .withId(flight.icao24)\leavevmode\par
{\hllin 96\ }}{\hlstd\ \ \ \ }{\hlstd .withExtrude(false);\leavevmode\par
{\hllin 97\ }}{\hlstd\ \ \ }{\hlstd \leavevmode\par
{\hllin 98\ }}{\hlstd\ \ \ }{\hlstd num\_{}flights$\mathord{+}$$\mathord{+}$;\leavevmode\par
{\hllin 99\ }}{\hlstd\ \ }{\hlstd $\}$\leavevmode\par
{\hllin 100\ }}{\hlstd\ \ }{\hlstd \leavevmode\par
{\hllin 101\ }}{\hlstd\ \ }{\hlstd public\ void\ writeToFile(File\ file)\ $\{$\leavevmode\par
{\hllin 102\ }}{\hlstd\ \ \ }{\hlstd try\ $\{$\leavevmode\par
{\hllin 103\ }}{\hlstd\ \ \ \ }{\hlstd kml.marshal(file);\leavevmode\par
{\hllin 104\ }}{\hlstd\ \ \ }{\hlstd $\}$\ catch\ (FileNotFoundException\ e)\ $\{$\leavevmode\par
{\hllin 105\ }}{\hlstd\ \ \ \ }{\hlstd System.err.println("Could\ not\ write\ to\ file\ '"$\mathord{+}$file.getAbsolutePath()$\mathord{+}$"'.$\backslash$nReason:\ "$\mathord{+}$e.toString());\leavevmode\par
{\hllin 106\ }}{\hlstd\ \ \ }{\hlstd $\}$\leavevmode\par
{\hllin 107\ }}{\hlstd\ \ }{\hlstd $\}$\leavevmode\par
{\hllin 108\ }}{\hlstd\ \ }{\hlstd \leavevmode\par
{\hllin 109\ }}{\hlstd\ \ }{\hlstd public\ int\ getNumberOfFlights()\ $\{$\leavevmode\par
{\hllin 110\ }}{\hlstd\ \ \ }{\hlstd return\ num\_{}flights;\leavevmode\par
{\hllin 111\ }}{\hlstd\ \ }{\hlstd $\}$\leavevmode\par
{\hllin 112\ }\ $\}$\leavevmode\par
{\hllin 113\ }\leavevmode\par
{\hllin 114\ }\ public\ static\ void\ main(String[]\ args)\ $\{$\leavevmode\par
{\hllin 115\ }}{\hlstd\ \ }{\hlstd \leavevmode\par
{\hllin 116\ }}{\hlstd\ \ }{\hlstd //\ define\ command\ line\ options\leavevmode\par
{\hllin 117\ }}{\hlstd\ \ }{\hlstd Options\ opts\ $\mathord{=}$\ new\ Options();\leavevmode\par
{\hllin 118\ }}{\hlstd\ \ }{\hlstd opts.addOption("h",\ "help",\ false,\ "print\ this\ message"\ );\leavevmode\par
{\hllin 119\ }}{\hlstd\ \ }{\hlstd opts.addOption("0",\ "nopos",\ false,\ "do\ not\ include\ flight\ without\ positions"\ );\leavevmode\par
{\hllin 120\ }}{\hlstd\ \ }{\hlstd opts.addOption("i",\ "icao24",\ true,\ "filter\ by\ icao\ 24$\mathord{-}$bit\ address\ (hex)");\leavevmode\par
{\hllin 121\ }}{\hlstd\ \ }{\hlstd opts.addOption("s",\ "start",\ true,\ "only\ messages\ received\ after\ this\ time\ (unix\ timestamp)");\leavevmode\par
{\hllin 122\ }}{\hlstd\ \ }{\hlstd opts.addOption("e",\ "end",\ true,\ "only\ messages\ received\ before\ this\ time\ (unix\ timestamp)");\leavevmode\par
{\hllin 123\ }}{\hlstd\ \ }{\hlstd opts.addOption("n",\ "max$\mathord{-}$num",\ true,\ "max\ number\ of\ flights\ written\ to\ KML");\leavevmode\par
{\hllin 124\ }}{\hlstd\ \ }{\hlstd \leavevmode\par
{\hllin 125\ }}{\hlstd\ \ }{\hlstd //\ parse\ command\ line\ options\leavevmode\par
{\hllin 126\ }}{\hlstd\ \ }{\hlstd CommandLineParser\ parser\ $\mathord{=}$\ new\ DefaultParser();\leavevmode\par
{\hllin 127\ }}{\hlstd\ \ }{\hlstd CommandLine\ cmd;\leavevmode\par
{\hllin 128\ }}{\hlstd\ \ }{\hlstd File\ avro\ $\mathord{=}$\ null,\ kmlfile\ $\mathord{=}$\ null;\leavevmode\par
{\hllin 129\ }}{\hlstd\ \ }{\hlstd String\ filter\_{}icao24\ $\mathord{=}$\ null;\leavevmode\par
{\hllin 130\ }}{\hlstd\ \ }{\hlstd Long\ filter\_{}max\ $\mathord{=}$\ null;\leavevmode\par
{\hllin 131\ }}{\hlstd\ \ }{\hlstd Double\ filter\_{}start\ $\mathord{=}$\ null,\ filter\_{}end\ $\mathord{=}$\ null;\leavevmode\par
{\hllin 132\ }}{\hlstd\ \ }{\hlstd String\ file\ $\mathord{=}$\ null,\ out\ $\mathord{=}$\ null;\leavevmode\par
{\hllin 133\ }}{\hlstd\ \ }{\hlstd boolean\ option\_{}nopos\ $\mathord{=}$\ true;\leavevmode\par
{\hllin 134\ }}{\hlstd\ \ }{\hlstd try\ $\{$\leavevmode\par
{\hllin 135\ }}{\hlstd\ \ \ }{\hlstd cmd\ $\mathord{=}$\ parser.parse(opts,\ args);\leavevmode\par
{\hllin 136\ }}{\hlstd\ \ \ }{\hlstd \leavevmode\par
{\hllin 137\ }}{\hlstd\ \ \ }{\hlstd //\ parse\ arguments\leavevmode\par
{\hllin 138\ }}{\hlstd\ \ \ }{\hlstd try\ $\{$\leavevmode\par
{\hllin 139\ }}{\hlstd\ \ \ \ }{\hlstd if\ (cmd.hasOption("i"))\ filter\_{}icao24\ $\mathord{=}$\ cmd.getOptionValue("i");\leavevmode\par
{\hllin 140\ }}{\hlstd\ \ \ \ }{\hlstd if\ (cmd.hasOption("s"))\ filter\_{}start\ $\mathord{=}$\ Double.parseDouble(cmd.getOptionValue("s"));\leavevmode\par
{\hllin 141\ }}{\hlstd\ \ \ \ }{\hlstd if\ (cmd.hasOption("e"))\ filter\_{}end\ $\mathord{=}$\ Double.parseDouble(cmd.getOptionValue("e"));\leavevmode\par
{\hllin 142\ }}{\hlstd\ \ \ \ }{\hlstd if\ (cmd.hasOption("n"))\ filter\_{}max\ $\mathord{=}$\ Long.parseLong(cmd.getOptionValue("n"));\leavevmode\par
{\hllin 143\ }}{\hlstd\ \ \ \ }{\hlstd if\ (cmd.hasOption("0"))\ option\_{}nopos\ $\mathord{=}$\ false;\leavevmode\par
{\hllin 144\ }}{\hlstd\ \ \ }{\hlstd $\}$\ catch\ (NumberFormatException\ e)\ $\{$\leavevmode\par
{\hllin 145\ }}{\hlstd\ \ \ \ }{\hlstd throw\ new\ ParseException("Invalid\ arguments:\ "$\mathord{+}$e.getMessage());\leavevmode\par
{\hllin 146\ }}{\hlstd\ \ \ }{\hlstd $\}$\leavevmode\par
{\hllin 147\ }}{\hlstd\ \ \ \ \ \ }{\hlstd \leavevmode\par
{\hllin 148\ }}{\hlstd\ \ \ }{\hlstd //\ get\ filename\leavevmode\par
{\hllin 149\ }}{\hlstd\ \ \ }{\hlstd if\ (cmd.getArgList().size()\ !$\mathord{=}$\ 2)\leavevmode\par
{\hllin 150\ }}{\hlstd\ \ \ \ }{\hlstd throw\ new\ ParseException("No\ avro\ file\ given\ or\ invalid\ arguments.");\leavevmode\par
{\hllin 151\ }}{\hlstd\ \ \ }{\hlstd file\ $\mathord{=}$\ cmd.getArgList().get(0);\leavevmode\par
{\hllin 152\ }}{\hlstd\ \ \ }{\hlstd out\ $\mathord{=}$\ cmd.getArgList().get(1);\leavevmode\par
{\hllin 153\ }}{\hlstd\ \ \ }{\hlstd \leavevmode\par
{\hllin 154\ }}{\hlstd\ \ }{\hlstd $\}$\ catch\ (ParseException\ e)\ $\{$\leavevmode\par
{\hllin 155\ }}{\hlstd\ \ \ }{\hlstd //\ parsing\ failed\leavevmode\par
{\hllin 156\ }}{\hlstd\ \ \ }{\hlstd System.err.println(e.getMessage()$\mathord{+}$"$\backslash$n");\leavevmode\par
{\hllin 157\ }}{\hlstd\ \ \ }{\hlstd printHelp(opts);\leavevmode\par
{\hllin 158\ }}{\hlstd\ \ \ }{\hlstd System.exit(1);\leavevmode\par
{\hllin 159\ }}{\hlstd\ \ }{\hlstd $\}$\leavevmode\par
{\hllin 160\ }}{\hlstd\ \ }{\hlstd \leavevmode\par
{\hllin 161\ }}{\hlstd\ \ }{\hlstd System.out.println("Opening\ avro\ file.");\leavevmode\par
{\hllin 162\ }}{\hlstd\ \ }{\hlstd \leavevmode\par
{\hllin 163\ }}{\hlstd\ \ }{\hlstd //\ check\ if\ file\ exists\leavevmode\par
{\hllin 164\ }}{\hlstd\ \ }{\hlstd try\ $\{$\leavevmode\par
{\hllin 165\ }}{\hlstd\ \ \ }{\hlstd avro\ $\mathord{=}$\ new\ File(file);\leavevmode\par
{\hllin 166\ }}{\hlstd\ \ \ }{\hlstd if(!avro.exists()\ ||\ avro.isDirectory()\ ||\ !avro.canRead())\ $\{$\leavevmode\par
{\hllin 167\ }}{\hlstd\ \ \ \ }{\hlstd throw\ new\ FileNotFoundException("Avro\ file\ not\ found\ or\ cannot\ be\ read.");\leavevmode\par
{\hllin 168\ }}{\hlstd\ \ \ }{\hlstd $\}$\leavevmode\par
{\hllin 169\ }}{\hlstd\ \ \ }{\hlstd \leavevmode\par
{\hllin 170\ }}{\hlstd\ \ \ }{\hlstd kmlfile\ $\mathord{=}$\ new\ File(out);\leavevmode\par
{\hllin 171\ }}{\hlstd\ \ \ }{\hlstd if(kmlfile.exists()\ ||\ kmlfile.isDirectory())\leavevmode\par
{\hllin 172\ }}{\hlstd\ \ \ \ }{\hlstd throw\ new\ java.io.IOException("KML\ is\ a\ directory\ or\ file\ exists.");\leavevmode\par
{\hllin 173\ }}{\hlstd\ \ }{\hlstd $\}$\ catch\ (FileNotFoundException\ e)\ $\{$\leavevmode\par
{\hllin 174\ }}{\hlstd\ \ \ }{\hlstd //\ avro\ file\ not\ found\leavevmode\par
{\hllin 175\ }}{\hlstd\ \ \ }{\hlstd System.err.println("Error:\ "$\mathord{+}$e.getMessage()$\mathord{+}$"$\backslash$n");\leavevmode\par
{\hllin 176\ }}{\hlstd\ \ \ }{\hlstd System.exit(1);\leavevmode\par
{\hllin 177\ }}{\hlstd\ \ }{\hlstd $\}$\ catch\ (IOException\ e)\ $\{$\leavevmode\par
{\hllin 178\ }}{\hlstd\ \ \ }{\hlstd //\ cannot\ write\ to\ KML\leavevmode\par
{\hllin 179\ }}{\hlstd\ \ \ }{\hlstd System.err.println("Error:\ "$\mathord{+}$e.getMessage()$\mathord{+}$"$\backslash$n");\leavevmode\par
{\hllin 180\ }}{\hlstd\ \ \ }{\hlstd System.exit(1);\leavevmode\par
{\hllin 181\ }}{\hlstd\ \ }{\hlstd $\}$\leavevmode\par
{\hllin 182\ }}{\hlstd\ \ }{\hlstd \leavevmode\par
{\hllin 183\ }}{\hlstd\ \ }{\hlstd DatumReader$\mathord{<}$ModeSEncodedMessage$\mathord{>}$\ datumReader\ $\mathord{=}$\ new\ SpecificDatumReader$\mathord{<}$ModeSEncodedMessage$\mathord{>}$(ModeSEncodedMessage.class);\leavevmode\par
{\hllin 184\ }}{\hlstd\ \ }{\hlstd long\ msgCount\ $\mathord{=}$\ 0,\ good\_{}pos\_{}cnt\ $\mathord{=}$\ 0,\ bad\_{}pos\_{}cnt\ $\mathord{=}$\ 0,\ flights\_{}cnt\ $\mathord{=}$\ 0,\ err\_{}pos\_{}cnt\ $\mathord{=}$\ 0;\leavevmode\par
{\hllin 185\ }}{\hlstd\ \ }{\hlstd try\ $\{$\leavevmode\par
{\hllin 186\ }}{\hlstd\ \ \ }{\hlstd DataFileReader$\mathord{<}$ModeSEncodedMessage$\mathord{>}$\ fileReader\ $\mathord{=}$\ new\ DataFileReader$\mathord{<}$ModeSEncodedMessage$\mathord{>}$(avro,\ datumReader);\leavevmode\par
{\hllin 187\ }}{\hlstd\ \ \ }{\hlstd \leavevmode\par
{\hllin 188\ }}{\hlstd\ \ \ }{\hlstd System.err.println("Options\ are:$\backslash$n"\ $\mathord{+}$\ \leavevmode\par
{\hllin 189\ }}{\hlstd\ \ \ \ \ }{\hlstd "$\backslash$tfile:\ "$\mathord{+}$file$\mathord{+}$"$\backslash$n"$\mathord{+}$\leavevmode\par
{\hllin 190\ }}{\hlstd\ \ \ \ \ }{\hlstd "$\backslash$ticao24:\ "$\mathord{+}$filter\_{}icao24$\mathord{+}$"$\backslash$n"$\mathord{+}$\leavevmode\par
{\hllin 191\ }}{\hlstd\ \ \ \ \ }{\hlstd "$\backslash$tstart:\ "$\mathord{+}$filter\_{}start$\mathord{+}$"$\backslash$n"$\mathord{+}$\leavevmode\par
{\hllin 192\ }}{\hlstd\ \ \ \ \ }{\hlstd "$\backslash$tend:\ "$\mathord{+}$filter\_{}end$\mathord{+}$"$\backslash$n"$\mathord{+}$\leavevmode\par
{\hllin 193\ }}{\hlstd\ \ \ \ \ }{\hlstd "$\backslash$tmax:\ "$\mathord{+}$filter\_{}max$\mathord{+}$"$\backslash$n");\leavevmode\par
{\hllin 194\ }}{\hlstd\ \ \ }{\hlstd \leavevmode\par
{\hllin 195\ }}{\hlstd\ \ \ }{\hlstd //\ stuff\ for\ handling\ flights\leavevmode\par
{\hllin 196\ }}{\hlstd\ \ \ }{\hlstd ModeSEncodedMessage\ record\ $\mathord{=}$\ new\ ModeSEncodedMessage();\leavevmode\par
{\hllin 197\ }}{\hlstd\ \ \ }{\hlstd HashMap$\mathord{<}$String,\ Flight$\mathord{>}$\ flights\ $\mathord{=}$\ new\ HashMap$\mathord{<}$String,\ Flight$\mathord{>}$();\leavevmode\par
{\hllin 198\ }}{\hlstd\ \ \ }{\hlstd Flight\ flight;\leavevmode\par
{\hllin 199\ }}{\hlstd\ \ \ }{\hlstd String\ icao24;\leavevmode\par
{\hllin 200\ }}{\hlstd\ \ \ }{\hlstd \leavevmode\par
{\hllin 201\ }}{\hlstd\ \ \ }{\hlstd //\ message\ registers\leavevmode\par
{\hllin 202\ }}{\hlstd\ \ \ }{\hlstd ModeSReply\ msg;\leavevmode\par
{\hllin 203\ }}{\hlstd\ \ \ }{\hlstd AirbornePositionMsg\ airpos;\leavevmode\par
{\hllin 204\ }}{\hlstd\ \ \ }{\hlstd SurfacePositionMsg\ surfacepos;\leavevmode\par
{\hllin 205\ }}{\hlstd\ \ \ }{\hlstd IdentificationMsg\ ident;\leavevmode\par
{\hllin 206\ }}{\hlstd\ \ \ }{\hlstd \leavevmode\par
{\hllin 207\ }}{\hlstd\ \ \ }{\hlstd //\ KML\ stuff\leavevmode\par
{\hllin 208\ }}{\hlstd\ \ \ }{\hlstd Avro2Kml\ a2k\ $\mathord{=}$\ new\ Avro2Kml();\leavevmode\par
{\hllin 209\ }}{\hlstd\ \ \ }{\hlstd OskyKml\ kml\ $\mathord{=}$\ a2k.new\ OskyKml();\leavevmode\par
{\hllin 210\ }}{\hlstd\ \ \ }{\hlstd \leavevmode\par
{\hllin 211\ }}{\hlstd\ \ \ }{\hlstd mainloop:\leavevmode\par
{\hllin 212\ }}{\hlstd\ \ \ }{\hlstd while\ (fileReader.hasNext())\ $\{$\leavevmode\par
{\hllin 213\ }}{\hlstd\ \ \ \ }{\hlstd msgCount$\mathord{+}$$\mathord{+}$;\leavevmode\par
{\hllin 214\ }}{\hlstd\ \ \ \ }{\hlstd \leavevmode\par
{\hllin 215\ }}{\hlstd\ \ \ \ }{\hlstd //\ get\ next\ record\ from\ file\leavevmode\par
{\hllin 216\ }}{\hlstd\ \ \ \ }{\hlstd record\ $\mathord{=}$\ fileReader.next(record);\leavevmode\par
{\hllin 217\ }}{\hlstd\ \ \ \ }{\hlstd \leavevmode\par
{\hllin 218\ }}{\hlstd\ \ \ \ }{\hlstd //\ time\ filters\leavevmode\par
{\hllin 219\ }}{\hlstd\ \ \ \ }{\hlstd if\ (filter\_{}start\ !$\mathord{=}$\ null\ \&{}\&{}\ record.getTimeAtServer()$\mathord{<}$filter\_{}start)\leavevmode\par
{\hllin 220\ }}{\hlstd\ \ \ \ \ }{\hlstd continue;\leavevmode\par
{\hllin 221\ }}{\hlstd\ \ \ \ }{\hlstd \leavevmode\par
{\hllin 222\ }}{\hlstd\ \ \ \ }{\hlstd if\ (filter\_{}end\ !$\mathord{=}$\ null\ \&{}\&{}\ record.getTimeAtServer()$\mathord{>}$filter\_{}end)\leavevmode\par
{\hllin 223\ }}{\hlstd\ \ \ \ \ }{\hlstd continue;\leavevmode\par
{\hllin 224\ }}{\hlstd\ \ \ \ }{\hlstd \leavevmode\par
{\hllin 225\ }}{\hlstd\ \ \ \ }{\hlstd //\ cleanup\ decoders\ every\ 1.000.000\ messages\ to\ avoid\ excessive\ memory\ usage\leavevmode\par
{\hllin 226\ }}{\hlstd\ \ \ \ }{\hlstd //\ therefore,\ remove\ decoders\ which\ have\ not\ been\ used\ for\ more\ than\ one\ hour.\leavevmode\par
{\hllin 227\ }}{\hlstd\ \ \ \ }{\hlstd if\ (msgCount\ \%{}\ 1000000\ $\mathord{=}$$\mathord{=}$\ 0)\ $\{$\leavevmode\par
{\hllin 228\ }}{\hlstd\ \ \ \ \ }{\hlstd List$\mathord{<}$String$\mathord{>}$\ to\_{}remove\ $\mathord{=}$\ new\ ArrayList$\mathord{<}$String$\mathord{>}$();\leavevmode\par
{\hllin 229\ }}{\hlstd\ \ \ \ \ }{\hlstd for\ (String\ key\ :\ flights.keySet())\ $\{$\leavevmode\par
{\hllin 230\ }}{\hlstd\ \ \ \ \ \ }{\hlstd if\ (flights.get(key).last$\mathord{<}$record.getTimeAtServer()$\mathord{-}$3600)\ $\{$\leavevmode\par
{\hllin 231\ }}{\hlstd\ \ \ \ \ \ \ }{\hlstd to\_{}remove.add(key);\leavevmode\par
{\hllin 232\ }}{\hlstd\ \ \ \ \ \ }{\hlstd $\}$\leavevmode\par
{\hllin 233\ }}{\hlstd\ \ \ \ \ }{\hlstd $\}$\leavevmode\par
{\hllin 234\ }\leavevmode\par
{\hllin 235\ }}{\hlstd\ \ \ \ \ }{\hlstd for\ (String\ key\ :\ to\_{}remove)\ $\{$\leavevmode\par
{\hllin 236\ }}{\hlstd\ \ \ \ \ \ }{\hlstd //\ number\ of\ flights\ filter\leavevmode\par
{\hllin 237\ }}{\hlstd\ \ \ \ \ \ }{\hlstd if\ (filter\_{}max\ !$\mathord{=}$\ null\ \&{}\&{}\ kml.getNumberOfFlights()$\mathord{>}$$\mathord{=}$filter\_{}max)\leavevmode\par
{\hllin 238\ }}{\hlstd\ \ \ \ \ \ \ }{\hlstd break\ mainloop;\leavevmode\par
{\hllin 239\ }\leavevmode\par
{\hllin 240\ }}{\hlstd\ \ \ \ \ \ }{\hlstd if\ (option\_{}nopos\ |\ flights.get(key).coords.size()\ $\mathord{>}$\ 0)\leavevmode\par
{\hllin 241\ }}{\hlstd\ \ \ \ \ \ \ }{\hlstd kml.addFlight(flights.get(key));\leavevmode\par
{\hllin 242\ }}{\hlstd\ \ \ \ \ \ }{\hlstd flights.remove(key);\leavevmode\par
{\hllin 243\ }}{\hlstd\ \ \ \ \ }{\hlstd $\}$\leavevmode\par
{\hllin 244\ }}{\hlstd\ \ \ \ }{\hlstd $\}$\leavevmode\par
{\hllin 245\ }}{\hlstd\ \ \ \ }{\hlstd \leavevmode\par
{\hllin 246\ }}{\hlstd\ \ \ \ }{\hlstd try\ $\{$\leavevmode\par
{\hllin 247\ }}{\hlstd\ \ \ \ \ }{\hlstd msg\ $\mathord{=}$\ Decoder.genericDecoder(record.getRawMessage().toString());\leavevmode\par
{\hllin 248\ }}{\hlstd\ \ \ \ }{\hlstd $\}$\ catch\ (BadFormatException\ e)\ $\{$\leavevmode\par
{\hllin 249\ }}{\hlstd\ \ \ \ \ }{\hlstd continue;\leavevmode\par
{\hllin 250\ }}{\hlstd\ \ \ \ }{\hlstd $\}$\leavevmode\par
{\hllin 251\ }\leavevmode\par
{\hllin 252\ }}{\hlstd\ \ \ \ }{\hlstd icao24\ $\mathord{=}$\ tools.toHexString(msg.getIcao24());\leavevmode\par
{\hllin 253\ }}{\hlstd\ \ \ \ }{\hlstd \leavevmode\par
{\hllin 254\ }}{\hlstd\ \ \ \ }{\hlstd //\ icao24\ filter\leavevmode\par
{\hllin 255\ }}{\hlstd\ \ \ \ }{\hlstd if\ (filter\_{}icao24\ !$\mathord{=}$\ null\ \&{}\&{}\ !icao24.equals(filter\_{}icao24))\leavevmode\par
{\hllin 256\ }}{\hlstd\ \ \ \ \ }{\hlstd continue;\leavevmode\par
{\hllin 257\ }}{\hlstd\ \ \ \ }{\hlstd \leavevmode\par
{\hllin 258\ }}{\hlstd\ \ \ \ }{\hlstd //\ select\ current\ flight\leavevmode\par
{\hllin 259\ }}{\hlstd\ \ \ \ }{\hlstd if\ (flights.containsKey(icao24))\leavevmode\par
{\hllin 260\ }}{\hlstd\ \ \ \ \ }{\hlstd flight\ $\mathord{=}$\ flights.get(icao24);\leavevmode\par
{\hllin 261\ }}{\hlstd\ \ \ \ }{\hlstd else\ $\{$\leavevmode\par
{\hllin 262\ }}{\hlstd\ \ \ \ \ }{\hlstd flight\ $\mathord{=}$\ a2k.new\ Flight();\leavevmode\par
{\hllin 263\ }}{\hlstd\ \ \ \ \ }{\hlstd flight.icao24\ $\mathord{=}$\ icao24;\leavevmode\par
{\hllin 264\ }}{\hlstd\ \ \ \ \ }{\hlstd flight.first\ $\mathord{=}$\ record.getTimeAtServer();\leavevmode\par
{\hllin 265\ }}{\hlstd\ \ \ \ \ }{\hlstd flights.put(icao24,\ flight);\leavevmode\par
{\hllin 266\ }}{\hlstd\ \ \ \ \ }{\hlstd $\mathord{+}$$\mathord{+}$flights\_{}cnt;\leavevmode\par
{\hllin 267\ }}{\hlstd\ \ \ \ }{\hlstd $\}$\leavevmode\par
{\hllin 268\ }\leavevmode\par
{\hllin 269\ }}{\hlstd\ \ \ \ }{\hlstd flight.last\ $\mathord{=}$\ record.getTimeAtServer();\leavevmode\par
{\hllin 270\ }}{\hlstd\ \ \ \ }{\hlstd \leavevmode\par
{\hllin 271\ }}{\hlstd\ \ \ \ }{\hlstd if\ (!flight.serials.contains(record.getSensorSerialNumber()))\leavevmode\par
{\hllin 272\ }}{\hlstd\ \ \ \ \ }{\hlstd flight.serials.add(record.getSensorSerialNumber());\leavevmode\par
{\hllin 273\ }\leavevmode\par
{\hllin 274\ }}{\hlstd\ \ \ \ }{\hlstd /////////\ Airborne\ Position\ Messages\leavevmode\par
{\hllin 275\ }}{\hlstd\ \ \ \ }{\hlstd if\ (msg.getType()\ $\mathord{=}$$\mathord{=}$\ ModeSReply.subtype.ADSB\_{}AIRBORN\_{}POSITION)\ $\{$\leavevmode\par
{\hllin 276\ }}{\hlstd\ \ \ \ \ }{\hlstd airpos\ $\mathord{=}$\ (AirbornePositionMsg)\ msg;\leavevmode\par
{\hllin 277\ }}{\hlstd\ \ \ \ \ }{\hlstd Position\ rec\ $\mathord{=}$\ record.getSensorLatitude()\ !$\mathord{=}$\ null\ ?\leavevmode\par
{\hllin 278\ }}{\hlstd\ \ \ \ \ \ \ }{\hlstd new\ Position(\leavevmode\par
{\hllin 279\ }}{\hlstd\ \ \ \ \ \ \ \ \ }{\hlstd record.getSensorLongitude(),\leavevmode\par
{\hllin 280\ }}{\hlstd\ \ \ \ \ \ \ \ \ }{\hlstd record.getSensorLatitude(),\leavevmode\par
{\hllin 281\ }}{\hlstd\ \ \ \ \ \ \ \ \ }{\hlstd record.getSensorAltitude())\ :\ null;\leavevmode\par
{\hllin 282\ }}{\hlstd\ \ \ \ \ }{\hlstd \leavevmode\par
{\hllin 283\ }}{\hlstd\ \ \ \ \ }{\hlstd airpos.setNICSupplementA(flight.dec.getNICSupplementA());\leavevmode\par
{\hllin 284\ }}{\hlstd\ \ \ \ \ }{\hlstd Position\ pos\ $\mathord{=}$\ flight.dec.decodePosition(record.getTimeAtServer(),\ rec,\ airpos);\leavevmode\par
{\hllin 285\ }}{\hlstd\ \ \ \ \ }{\hlstd if\ (pos\ $\mathord{=}$$\mathord{=}$\ null)\leavevmode\par
{\hllin 286\ }}{\hlstd\ \ \ \ \ \ }{\hlstd $\mathord{+}$$\mathord{+}$err\_{}pos\_{}cnt;\leavevmode\par
{\hllin 287\ }}{\hlstd\ \ \ \ \ }{\hlstd else\ $\{$\leavevmode\par
{\hllin 288\ }}{\hlstd\ \ \ \ \ \ }{\hlstd if\ (pos.isReasonable())\ $\{$\leavevmode\par
{\hllin 289\ }}{\hlstd\ \ \ \ \ \ \ }{\hlstd Coordinate\ coord\ $\mathord{=}$\ new\ Coordinate(pos.getLongitude(),\ pos.getLatitude(),\leavevmode\par
{\hllin 290\ }}{\hlstd\ \ \ \ \ \ \ \ \ }{\hlstd //\ set\ altitude\ to\ 0\ if\ negative...\ looks\ nicer\ in\ google\ earth\leavevmode\par
{\hllin 291\ }}{\hlstd\ \ \ \ \ \ \ \ \ }{\hlstd pos.getAltitude()\ !$\mathord{=}$\ null\ \&{}\&{}\ pos.getAltitude()$\mathord{>}$0\ ?\ pos.getAltitude()\ :\ 0);\leavevmode\par
{\hllin 292\ }}{\hlstd\ \ \ \ \ \ \ }{\hlstd if\ (!flight.coords.contains(coord))\ $\{$\ //\ remove\ duplicates\ to\ safe\ memory\leavevmode\par
{\hllin 293\ }}{\hlstd\ \ \ \ \ \ \ \ }{\hlstd flight.coords.add(coord);\leavevmode\par
{\hllin 294\ }}{\hlstd\ \ \ \ \ \ \ \ }{\hlstd $\mathord{+}$$\mathord{+}$good\_{}pos\_{}cnt;\leavevmode\par
{\hllin 295\ }}{\hlstd\ \ \ \ \ \ \ }{\hlstd $\}$\leavevmode\par
{\hllin 296\ }}{\hlstd\ \ \ \ \ \ }{\hlstd $\}$\leavevmode\par
{\hllin 297\ }}{\hlstd\ \ \ \ \ \ }{\hlstd else\ $\{$\leavevmode\par
{\hllin 298\ }}{\hlstd\ \ \ \ \ \ \ }{\hlstd flight.contains\_{}unreasonable\ $\mathord{=}$\ true;\leavevmode\par
{\hllin 299\ }}{\hlstd\ \ \ \ \ \ \ }{\hlstd $\mathord{+}$$\mathord{+}$bad\_{}pos\_{}cnt;\leavevmode\par
{\hllin 300\ }}{\hlstd\ \ \ \ \ \ }{\hlstd $\}$\leavevmode\par
{\hllin 301\ }}{\hlstd\ \ \ \ \ }{\hlstd $\}$\leavevmode\par
{\hllin 302\ }}{\hlstd\ \ \ \ }{\hlstd $\}$\leavevmode\par
{\hllin 303\ }}{\hlstd\ \ \ \ }{\hlstd /////////\ Surface\ Position\ Messages\leavevmode\par
{\hllin 304\ }}{\hlstd\ \ \ \ }{\hlstd else\ if\ (msg.getType()\ $\mathord{=}$$\mathord{=}$\ ModeSReply.subtype.ADSB\_{}SURFACE\_{}POSITION)\ $\{$\leavevmode\par
{\hllin 305\ }}{\hlstd\ \ \ \ \ }{\hlstd surfacepos\ $\mathord{=}$\ (SurfacePositionMsg)\ msg;\leavevmode\par
{\hllin 306\ }}{\hlstd\ \ \ \ \ }{\hlstd Position\ rec\ $\mathord{=}$\ record.getSensorLatitude()\ !$\mathord{=}$\ null\ ?\leavevmode\par
{\hllin 307\ }}{\hlstd\ \ \ \ \ \ \ }{\hlstd new\ Position(\leavevmode\par
{\hllin 308\ }}{\hlstd\ \ \ \ \ \ \ \ \ }{\hlstd record.getSensorLongitude(),\leavevmode\par
{\hllin 309\ }}{\hlstd\ \ \ \ \ \ \ \ \ }{\hlstd record.getSensorLatitude(),\leavevmode\par
{\hllin 310\ }}{\hlstd\ \ \ \ \ \ \ \ \ }{\hlstd record.getSensorAltitude())\ :\ null;\leavevmode\par
{\hllin 311\ }}{\hlstd\ \ \ \ \ }{\hlstd \leavevmode\par
{\hllin 312\ }}{\hlstd\ \ \ \ \ }{\hlstd Position\ pos\ $\mathord{=}$\ flight.dec.decodePosition(record.getTimeAtServer(),\ rec,\ surfacepos);\leavevmode\par
{\hllin 313\ }}{\hlstd\ \ \ \ \ }{\hlstd if\ (pos\ $\mathord{=}$$\mathord{=}$\ null)\leavevmode\par
{\hllin 314\ }}{\hlstd\ \ \ \ \ \ }{\hlstd $\mathord{+}$$\mathord{+}$err\_{}pos\_{}cnt;\leavevmode\par
{\hllin 315\ }}{\hlstd\ \ \ \ \ }{\hlstd else\ $\{$\leavevmode\par
{\hllin 316\ }}{\hlstd\ \ \ \ \ \ }{\hlstd if\ (pos.isReasonable())\ $\{$\leavevmode\par
{\hllin 317\ }}{\hlstd\ \ \ \ \ \ \ }{\hlstd Coordinate\ coord\ $\mathord{=}$\ new\ Coordinate(pos.getLongitude(),\ pos.getLatitude(),\ 0);\leavevmode\par
{\hllin 318\ }}{\hlstd\ \ \ \ \ \ \ }{\hlstd \leavevmode\par
{\hllin 319\ }}{\hlstd\ \ \ \ \ \ \ }{\hlstd if\ (!flight.coords.contains(coord))\ $\{$\ //\ remove\ duplicates\ to\ safe\ memory\leavevmode\par
{\hllin 320\ }}{\hlstd\ \ \ \ \ \ \ \ }{\hlstd flight.coords.add(coord);\leavevmode\par
{\hllin 321\ }}{\hlstd\ \ \ \ \ \ \ \ }{\hlstd $\mathord{+}$$\mathord{+}$good\_{}pos\_{}cnt;\leavevmode\par
{\hllin 322\ }}{\hlstd\ \ \ \ \ \ \ }{\hlstd $\}$\leavevmode\par
{\hllin 323\ }}{\hlstd\ \ \ \ \ \ }{\hlstd $\}$\leavevmode\par
{\hllin 324\ }}{\hlstd\ \ \ \ \ \ }{\hlstd else\ $\{$\leavevmode\par
{\hllin 325\ }}{\hlstd\ \ \ \ \ \ \ }{\hlstd flight.contains\_{}unreasonable\ $\mathord{=}$\ true;\leavevmode\par
{\hllin 326\ }}{\hlstd\ \ \ \ \ \ \ }{\hlstd $\mathord{+}$$\mathord{+}$bad\_{}pos\_{}cnt;\leavevmode\par
{\hllin 327\ }}{\hlstd\ \ \ \ \ \ }{\hlstd $\}$\leavevmode\par
{\hllin 328\ }}{\hlstd\ \ \ \ \ }{\hlstd $\}$\leavevmode\par
{\hllin 329\ }}{\hlstd\ \ \ \ }{\hlstd $\}$\leavevmode\par
{\hllin 330\ }}{\hlstd\ \ \ \ }{\hlstd /////////\ Identification\ Messages\leavevmode\par
{\hllin 331\ }}{\hlstd\ \ \ \ }{\hlstd else\ if\ (msg.getType()\ $\mathord{=}$$\mathord{=}$\ ModeSReply.subtype.ADSB\_{}IDENTIFICATION)\ $\{$\leavevmode\par
{\hllin 332\ }}{\hlstd\ \ \ \ \ }{\hlstd ident\ $\mathord{=}$\ (IdentificationMsg)\ msg;\leavevmode\par
{\hllin 333\ }}{\hlstd\ \ \ \ \ }{\hlstd flight.callsign\ $\mathord{=}$\ ident.getIdentity();\leavevmode\par
{\hllin 334\ }}{\hlstd\ \ \ \ }{\hlstd $\}$\leavevmode\par
{\hllin 335\ }}{\hlstd\ \ \ }{\hlstd $\}$\leavevmode\par
{\hllin 336\ }}{\hlstd\ \ \ }{\hlstd \leavevmode\par
{\hllin 337\ }}{\hlstd\ \ \ }{\hlstd //\ write\ residual\ flights\ to\ KML\leavevmode\par
{\hllin 338\ }}{\hlstd\ \ \ }{\hlstd for\ (String\ key\ :\ flights.keySet())\ $\{$\leavevmode\par
{\hllin 339\ }}{\hlstd\ \ \ \ }{\hlstd //\ number\ of\ flights\ filter\leavevmode\par
{\hllin 340\ }}{\hlstd\ \ \ \ }{\hlstd if\ (filter\_{}max\ !$\mathord{=}$\ null\ \&{}\&{}\ kml.getNumberOfFlights()$\mathord{>}$$\mathord{=}$filter\_{}max)\leavevmode\par
{\hllin 341\ }}{\hlstd\ \ \ \ \ }{\hlstd break;\leavevmode\par
{\hllin 342\ }}{\hlstd\ \ \ \ }{\hlstd if\ (option\_{}nopos\ |\ flights.get(key).coords.size()\ $\mathord{>}$\ 0)\leavevmode\par
{\hllin 343\ }}{\hlstd\ \ \ \ \ }{\hlstd kml.addFlight(flights.get(key));\leavevmode\par
{\hllin 344\ }}{\hlstd\ \ \ }{\hlstd $\}$\leavevmode\par
{\hllin 345\ }}{\hlstd\ \ \ }{\hlstd \leavevmode\par
{\hllin 346\ }}{\hlstd\ \ \ }{\hlstd fileReader.close();\leavevmode\par
{\hllin 347\ }}{\hlstd\ \ \ }{\hlstd kml.writeToFile(kmlfile);\leavevmode\par
{\hllin 348\ }}{\hlstd\ \ \ }{\hlstd \leavevmode\par
{\hllin 349\ }}{\hlstd\ \ }{\hlstd $\}$\ catch\ (IOException\ e)\ $\{$\leavevmode\par
{\hllin 350\ }}{\hlstd\ \ \ }{\hlstd //\ error\ while\ trying\ to\ read\ file\leavevmode\par
{\hllin 351\ }}{\hlstd\ \ \ }{\hlstd System.err.println("IO\ Error:\ "$\mathord{+}$e.getMessage());\leavevmode\par
{\hllin 352\ }}{\hlstd\ \ \ }{\hlstd System.exit(1);\leavevmode\par
{\hllin 353\ }}{\hlstd\ \ }{\hlstd $\}$\ catch\ (Exception\ e)\ $\{$\leavevmode\par
{\hllin 354\ }}{\hlstd\ \ \ }{\hlstd //\ something\ went\ wrong\leavevmode\par
{\hllin 355\ }}{\hlstd\ \ \ }{\hlstd System.err.println("Something\ went\ wrong:\ "$\mathord{+}$e.getMessage());\leavevmode\par
{\hllin 356\ }}{\hlstd\ \ \ }{\hlstd e.printStackTrace();\leavevmode\par
{\hllin 357\ }}{\hlstd\ \ \ }{\hlstd System.exit(1);\leavevmode\par
{\hllin 358\ }}{\hlstd\ \ }{\hlstd $\}$\leavevmode\par
{\hllin 359\ }}{\hlstd\ \ }{\hlstd \leavevmode\par
{\hllin 360\ }}{\hlstd\ \ }{\hlstd System.err.println("Read\ "$\mathord{+}$msgCount$\mathord{+}$"\ messages.");\leavevmode\par
{\hllin 361\ }}{\hlstd\ \ }{\hlstd System.err.println("Good\ positions:\ "$\mathord{+}$good\_{}pos\_{}cnt);\leavevmode\par
{\hllin 362\ }}{\hlstd\ \ }{\hlstd System.err.println("Bad\ positions:\ "$\mathord{+}$bad\_{}pos\_{}cnt);\leavevmode\par
{\hllin 363\ }}{\hlstd\ \ }{\hlstd System.err.println("Erroneous\ positions:\ "$\mathord{+}$err\_{}pos\_{}cnt);\leavevmode\par
{\hllin 364\ }}{\hlstd\ \ }{\hlstd System.err.println("Flights:\ "$\mathord{+}$flights\_{}cnt);\leavevmode\par
{\hllin 365\ }\ $\}$\leavevmode\par
{\hllin 366\ }$\}$}\leavevmode\par
}


\section{本章小结}
本章首先介绍了本系统的建立的基础,即三个主要依据的模型,分别是系统模型、网络模型和ADS-B模型,这些模型为系统假设了一个理想的设计应用环境模型。主要对本系统的加解密模块、解码模块以及可视化模块的设计原理以及具体编程算法进行详细的介绍。
