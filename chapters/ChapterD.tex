% !Mode:: "TeX:UTF-8"

\chapter{系统测试}
\section{测试环境}
本项目的所有调试以及测试均在个人笔记本下完成。\newline
硬件:\newline
联想ThinkPadT440P笔记本配置如下:\newline
(a)Intel(R) Core(TM) i3-4000M CPU @2.40GHz 2.39GHz \newline
(b)4.00G内存 \newline
(c)500G硬盘  \newline
软件环境:\newline
(a)Windows 10专业版操作系统(64位)\newline
(b)Python 2.7\newline
(c)java 1.8\newline
(d)jython 2.5.2
\section{测试结果}
\subsection{加密}
\subsubsection{功能实现}
通过结果截图可以看到,Google Earth上显示的飞行器的轨迹与未经过与原始的轨迹一样,即即使对ADS-B消息进行加密处理也满足不改变飞行器轨迹的要求,只对唯一标识ICAO24实现了格式保护加密。加密后的匿名ICAO24与原ICAO24不同,长度不变,字符空间不变,依旧在$\{0,1,2,3,4,5,6,7,8,9,a,b,c,d,e,f\}$字符集内,故加密结果符合格式加密的功能要求。

\pic[htbp]{单条未经加密的原轨迹和原飞行标志ICAO24}{width=0.8\textwidth}{p13}

\pic[htbp]{单条经过加密的轨迹和飞行标志ICAO24}{width=0.8\textwidth}{p14}
\subsubsection{稳定性}
如图所示,对不同的单条轨迹、多条轨迹的ADS-B消息进行FFX算法的格式保护加密,结果均符合功能要求,故此ADS-B隐私保护系统的稳定性较强。
\begin{pics}[h]{多条未经加密的原轨迹和原飞行标志ICAO24}{p15}
\addsubpic{轨迹1}{height=0.4\textheight}{p15_1}
\addsubpic{轨迹2}{height=0.4\textheight}{p15_2}
\end{pics}
\begin{pics}[h]{多条经过加密的轨迹和飞行标志ICAO24}{p16}
\addsubpic{轨迹1}{height=0.4\textheight}{p16_1}
\addsubpic{轨迹2}{height=0.4\textheight}{p16_2}
\end{pics}

\subsubsection{时效性}
如图所示,每条ADS-B消息的加密时间介于0.00ms-1.00ms之间,故将FFX加解密算法应用与ADS-B的隐私保护中对ADS-B消息的发送和接收的时效性无明显影响。

\pic[htbp]{加解密一条消息的用时}{width=0.8\textwidth}{p17}

\subsection{解密}
\subsubsection{功能实现}
通过结果截图看到,经过解密飞行轨迹还是与原来一样,ICAO24也与未经过加密的ICAO24一样,即对ICAO24进行了解密还原,其他均不变,因此实现了解密的功能要求。

\pic[htbp]{单条经过解密的轨迹和飞行标志ICAO24}{width=0.8\textwidth}{p18}

\subsubsection{稳定性}
如图所示,对不同的单条轨迹、多条轨迹的ADS-B消息进行FFX算法的格式保护解密,结果均符合功能要求,故此ADS-B隐私保护系统的稳定性较强。

\begin{pics}[h]{多条经过解密的轨迹和飞行标志ICAO24}{p19}
\addsubpic{轨迹1}{height=0.4\textheight}{p19_1}
\addsubpic{轨迹2}{height=0.4\textheight}{p19_2}
\end{pics}

\subsubsection{时效性}
如图15所示,每条ADS-B消息的解密时间与该消息的加密时间相同且时间介于0.00ms-1.00ms之间,故将FFX加解密算法应用与ADS-B的隐私保护中对ADS-B消息的发送和接收的时效性无明显影响。
\subsection{解码及解码结果的可视化}
\subsubsection{功能实现}
如上图所示,输入的.avro文件包含ADS-B消息和接收时间,接收地坐标等信息,输出的.kml文件在Google Earth上显示出来的是一条条轨迹和每条轨迹上代表飞行器的信息如ICAO24,因此实现了把ADS-B原始消息解码成具体的经纬度坐标或ICAO24飞行器标识信息并把经纬度信息和飞行器标识信息在地图软件上进行展示,因此也实现了结果可视化功能。
\section{总体分析}
本系统实现了对ADS-B数据链的隐私保护的基本课题要求。通过使用FFX格式保护算法对ADS-B消息的ICAO24部分进行选择性加密从而实现了让飞行器对无关者匿名的效果,从而保证了ADS-B数据链的隐私需求。同时,为了对加解密结果进行检验和效果进行展示,本系统还实现了对ADS-B消息进行解码和可视化处理的功能,通过解码和可视化处理,把ADS-B消息所包含的信息转化成地图软件上的一条条飞行轨迹和每条轨迹上代表的飞行器的信息如ICAO24直观的展示出来,从而证明了使用FFX格式保护算法对ADS-B消息的ICAO24部分进行加密不会影响飞行器的飞行轨迹信息,并且使用格式不糊算法这种加密方式使加密过后的ICAO24的字符数和字符域均不变,从而使加密后的ICAO24值也符合ADS-B消息格式对ICAO24格式的长度要求和字符域要求,使加密后的ICAO24也能成功解码,且解码后的ICAO24也符合ICAO24的命名标准,从而真正实现匿名的效果。由FFX格式保护算法的加密特性可得当输入的明文和密文一致时,对该明文进行多次加密的结果是相同的,故FFX格式保护算法是静态加密算法。因为FFX算法是静态加密算法,故长期使用同一密钥对同一飞行器的ADS-B消息进行加密会随着时间的增大而增大匿名ICAO24与真实ICAO24的对应破解的可能性,然而,对密钥的更换会产生一定的传输代价,因此需要对密钥更换周期做出合理的分析,使在这个周期频率内对密钥进行更新即能把对ICAO24进行对应破解的可能性降到一个可以接收的范围内有能把对密钥更新的成本控制在一定范围内,即寻找安全和经济的最佳平衡点。还有一点是ADS-B系统对消息收发的时效性是有很高的要求的,如果某种加密算法的加解密时效性不好,从而影响到整个ADS-B系统的时效性能的话则这种加密算法是不可取的,因为飞行器及乘客的人身安全保障是永远排在第一位的。由上面的数据可得,在个人笔记本的硬件配置下,使用的FFX加解密算法的加解密时间相同且都少于1ms,可以推测在ADS-B OUT的硬件配置下运行该加解密运算的耗时会更小,因此把该加解密算法嵌入ADS-B系统中对ADS-B的效率不会产生明显影响,因此满足实际应用的要求。




















