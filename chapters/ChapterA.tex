% !Mode:: "TeX:UTF-8"

\chapter{绪论}
\section{课题的背景}
早期对飞机的监视,高交通密度区靠雷达监视,低交通密度区、边远地区或海洋只能靠驾驶员利用高频(HF)无线电台用电报或话音方式来发送飞机位置报告。高频无线电这种通信方式非常不可靠,极易受天气或其他突发事件影响。 雷达监视是地面独立的对空监视,从一次监视雷达发展到二次监视雷达,从 A/C 模式发展到未来的 S 模式,是目前国际上普遍采用的技术。为了弥补边远地区和海洋等无法安装雷达的地区的监视之需,新航行系统(FANS)委员会推荐采用可靠空/地通信(如卫星通信)自动周期性数字式数据报告飞机位置,虽是非独立监视,但仍然是有效的监视技术,为此推荐了自动相关监视 ADS(Automatic dependent surveillance)技术\cite{zhang2013adsb}。 

国际民航组织 ICAO 定义 ADS 技术为:“ADS 是一种监视技术,由飞机将机上导航定位系统导出的数据通过数据链自动发送,这些数据至少包括飞机识别码,四维位置和所需附加数据”。ADS 技术是基于卫星定位和地/空数据链通信的航空器运行监视技术,是为越洋飞行的航空器在无法进行雷达监视的情况下,希望利用卫星实施监视所提出的解决方案。我国在沿青藏高原飞行的欧亚航路上建立的 L888 航路即采用了 ADS 的监视方式\cite{mccallie2011security}。 

ADS 可以分为 ADS-A(ADS-Addressing 选址式)和 ADS-C(ADS-Contract 合约式)两种模式。ADS-A/C 是飞机与航空管理单位之间首先建立起点对点的通信链接,在建立起链接之后,根据约定,飞机上导航设备自动地将飞机的有关信息传输给空中交通管制部门,同时地面也可以向飞行发送上行信息。通信链接的方式可以是甚高频数据链(VHF)、高频数据链(HF)、二次雷达(SSR)S 模式应答机、移动卫(AMSS)通信等。飞机向下报告的信息通过地面网络传到空管中心,经过数据解码转化等处理,飞机的位置等信息就可以显示在屏幕上。这种飞行员和管制员之间建立的数据链通信也叫做管制员飞行员数据链通信(CDPLC)。ADS-A 与 ADS-C 的区别是在建立起地空链接后,触发飞机向下报告的方式不同。ADS-A 是根据事先的约定来触发向下报告,这个约定可以是一定的时间间隔、过每个航路点等自动下传;ADS-C 是根据地面管制单位的询问来进行应答下传\cite{huang2008guang}。 

ADS 是建立在地对空监视基础上的,八十年代后期发展的空中交通警告与防撞系统(TCAS)是建立在空对空监视基础上的,而用于机场场面活动监视是地对地监视。随着技术的发展,开始将这三种技术结合成一体,使活动着的飞机(包括滑行中的飞机)利用全球定位系统(Global Positioning System, GPS)和其他全球导航卫星系统(Global Navigation Satellite Systems ,GNSS) 获取飞机本身的监视数据。然后利用 ADS-B OUT 转发器将飞机自身的身份标识符、3D位置、速度、飞行目的地和其他空间信息以一定的周期广播出去。地面上的空中交通控制中心(Air Traffic Control, ATC) 和其他配备了ADS-B IN 的飞机都能接收到ADS-B信息。通过几年的试验。ICAO 的 ADS 专家组认可并定名该技术为广播式自动相关监视 ADS-B。 广播式自动相关监视 ADS-B(Automatic dependent surveillance-Broadcast),即航空器自动广播由机载导航设备和 GNSS 定位系统生成的精确定位信息,地面设备和其他航空器通过航空数据链接收此信息,飞机以及地面系统通过高速数据链进行空对空、空对地以及地面的一体化协同监视。 

ADS-B 与 ADS-A/C 最大的不同在于它不是采用点对点的通信方式,而是采用广播的方式。如此,不仅可以实现地面对飞机的监视,同时也可以实现飞机与飞机之间的互相监视。由于 ADS 监视方式存在的诸多不足,如飞行信息处理时间过长不能满足终端区管制要求、监视方式费用较高等问题。因此,除了在洋区、偏远山区等不易安装地面设备的地区,考虑利用 ADS-A/C 对飞机进行监视之外,其他地区 ICAO 都推荐使用 ADS-B\cite{dai2011guang}。
\section{课题的研究价值和意义}
ADS-B 技术是新航行系统中非常重要的通信和监视技术,把冲突探测、冲突避免、冲突解脱、ATC 监视和 ATC 一致性监视以及 CDTI 综合信息显示有机的结合起来,为新航行系统增强和扩展了非常丰富的功能,同时也带来了潜在的经济效益和社会效益。 ADS-B 不仅可以应用于无雷达地区的远程航空器运行监视,而且与传统雷达监视技术相比,ADS-B技术具有使用成本低、精度误差小、监视能力强等明显优势,可以有效地提高了空管安全监视的水平。 ADS-B 技术能提高飞行中的航空器之间的相互监视能力。与 TCAS 相比,ADS-B 的位置报告是自发广播式的,航空器之间无须发出问询即可接收和处理渐近航空器的位置报告。将 ADS-B 应用于 TCAS,可以提高 TCAS 的性能,增强飞机之间的防撞能力,ADS-B 系统的这一能力,使保持飞行安全间隔的责任更多地向空中转移,这是实现“自由飞行”不可或缺的技术基础。 ADS-B 技术用于机场地面活动区,可以降低成本实现航空器的场面活动监视。利用 ADS-B 技术,通过接收和处理 ADS-B 广播信息,将活动航空器的监视从空中一直延伸到机场登机桥,因此能辅助场面监视雷达,实现“门到门”的空中交通管理。甚至可以不依赖场面监视雷达,实现机场地面移动目标的管理\cite{huang2008guang}。 

FAA 认为,ADS-B 作为一种全新的监视技术,是实施自由飞行的奠基石。其主要功能(ADS-B、TIS-B、FIS-B)可以在监视、防撞、辅助进近上发挥作用,而且较原先一次雷达、二次雷达监视又具有明显的优势,对未来的飞行安全产生重大影响。ADS-B 作为未来监视系统的发展方向必然是大势所趋,对于我国发展大型民机也有着良好的借鉴作用,C919 飞机上也已确定安装 ADS-B 相关产品。因此,研究 ADS-B 技术有着极为重要的意义。世界的空中交通管理正在从目前的非合作、独立的主监视雷达(primary surveillance radar, PSR) 变成合作的、非独立的 自动非独立监视广播(automatic dependent surveillance-broadcast ,ADS-B).根据美国联邦航空管理局(Federal Aviation Administration)关于下一代空中交通系统的实施规定,到2020年,所有经过美国的飞机必须配备ADS-B设备\cite{mccallie2011security},这意味着新一代现代化航空的到来。

但是ADS-B的通信协议设计之初并没有把安全问题考虑在内。根据它的开放式的协议,ADS-B消息都是没有考虑加密、数字签名等安全机制直接广播出去,任何具备ADS-B OUT设备的人都可以广播消息,具备 ADS-B IN 的人也可以接受任意飞机的消息。这样的开放式协议虽然拥有国际互操作性的优点,但是由于缺乏机密性、认证性也要承受各种攻击如信息泄露等被动攻击和欺骗假目标飞机(spoofing false target aircraft)等主动攻击所带来的影响。因此本课题将研究 ADS-B 数据链隐私保护的新方法,其主要思想是通过对数据链的加密实现飞机假名的更新。
\section{课题的国内外研究现状}
对 ADS-B 理论技术的研究除了已制定并正在逐步完善的相关技术标准和规范之外(如 SARPS,MASPS,MOPS,TSO 等),还有对 ADS-B 数据传输,ADS-B 数据链及仿真技术和 ADS-B 与雷达数据融合等方面的研究。
\subsection{ADS-B 数据传输技术} 
自由飞行中, ADS-B 数据准确性对于飞行安全起到至关重要的作用,但是基于目前的设计,数据丢包,错误输入和数据欺骗都会降低数据的完整性,使得 ADS-B 无法从其它飞机上获得准确无误的状态和意图数据。 鉴于此,K.Tysen Muellerh 和 Jimmy Krozel 等较早提出了“基于卡尔曼跟踪滤波的 ADS-B 意图信息完整性校验方法”\cite{mueller2000aircraft}。之后,Jimmy Krozel等还提出了“数据完整性校验的方法”,对 ADS-B 数据建立连续的实时状态估计系统,该系统采用卡尔曼滤波的方法平滑测量到的 ADS-B 数据中的噪声,识别错误的数据和数据丢包\cite{krozel2004aircraft},
提供当前最佳的状态估计。此外,针对在二次雷达(SSR)无法覆盖的地方采用 ADS-B 监视的方法无法一直保证信号准确性的问题,Jimmy Krozel 等还提出了“对应于二次雷达跟踪监视的确认和授权批准技术(V&V technique)”\cite{krozel2005independent},该技术不仅可以应用于机载系统,还可以引用于场面监视多点定位系统。上述研究都是从数据传输准确性的角度出发进行的。 在国内,彭良福,郑超等人提出的基于 1090ES 数据链 ADS-B 的 CPR 算法\cite{liu20151090es},对于飞机经度和纬度消息,采用简洁位置报告(Compact position reporting,CPR)的信源编码形式,针对全球位置和本地位置两种情况,实现飞机的经度和纬度消息的CPR 格式的编码和解码,提高了数据传输的效率。白松浩等人从工程实际中出发,为解决广播式自动相关监视信息和合约式自动相关监视信息相互转换提供了方法,实现了两种体制信息相互转换和信息共享,取得较好效果\cite{bai2005guang}。
\subsection{ADS-B 数据链及仿真技术} 
ADS-B 技术可选的数据链技术有以下三种:1090 ES、UAT、VDL MODE 4。UAT 是专门为 ADS-B 设计的一种数据链系统,美国在其安装的通用飞机上采用的工作频率为 978MHz。陈志杰和朱晓辉提出了基于现有机载设备信道构建 UAT 数据链路的实现方案\cite{chen2008jiyu},目的在于缩短设备研制周期、减少技术风险、节约投资。该方案在对标准 UAT 协议进行了剪裁和修改之后,利用了某型机载设备的发射机及其天线,通过增加 UAT 接收机和通信管理单元构成 UAT 机载端机,并单独配置地面端机后实现了方案的设计。通过仿真分析和原理性试验,验证得出该方案设计的 UAT 数据链系统与某型号系统可以协同工作,并且满足 UAT 系统设计要求,工作稳定可靠,有利于加快新一代空中交通管理系统的整体建设。 黄飞、张军等人提出了基于 OPNET 的 UAT 数据链仿真模型\cite{huang2008modeling}。在建立的该模型基础上,通过比较在不同飞机数量情况下,端到端的时延、比特误差率、数据丢包率,通道利用率等参数的情况,得出结论:基于 UAT 的数据链可以为飞机和地面站之间提供了很好的通信服务,适宜为空管提供监视和态势感知服务。
\subsection{ADS-B安全}
Sampigethaya 和 Poovendran\cite{sampigethaya2011security} 第一次分析了 ADS-B 安全和隐私\cite{sampigethaya2009framework}。McCallie 等人\cite{mccallie2011security}提出了目前的安全性分析,着重与分析可能遇到的攻击的本质和困难。他们都针对ADS-B的问题给出了一个系统级的高层次的和概括性的修补建议。Costin、Francillon和Schaefer\cite{costin2012ghost}也分析了 ADS-B 的安全性,主要研究让 ADS-B 利用目前的硬件和软件来提供一些可能的应对措施\cite{schafer2013experimental}。Kovell 等人\cite{kovell2012comparative}提及其他系统的数据融合和多侧面以及各种加密方案,他们也进一步的进行了一个对 Kalman 过滤和组认证的概念和可行操作的更加全面的分析。Nuseibeh 等人进行了一个 ADS-B 的正式安全需求例子分析,提出了多侧面方案来处理可能的攻击情景\cite{nuseibeh2009securing}。Burbank 等人提出了一个沟通网络通用概念来满足未来航空系统的要求,比如一个在终端区域和再规划路线的航空区域中的移动端的版本和无线网络的概念\cite{burbank2005advanced}。Li 和 Kamal\cite{li2011integrated}分析了以 ADS-B 为核心的整个 FAA 的下一代航空系统的安全性。他们建立了一个高层次的防御方案去分析下一代航空系统也提到了一些通用的安全通信措施比如加密,认证和扩散范围等需要被深层次检验的ADS-B级别的安全方案。国内的 Tso-Cho Chen 和 HsinChu\cite{chen2012authenticated}提出了一种通过通信双方共享一对私钥的的一种算法来为 ADS-B 提供机密性和认证性保护,虽然按照他们的算法确实是可以起到机密性和认证性的目的但是他们在算法中并没有考虑 ADS-B 消息的实际使用情况,例如他们在加密过程中采用的是把整个 ADS-B 消息加密的方式,但是 ADS-B 消息的前八位是不能加密的,否则接收者将无法辨认这是ADS-B 消息因此导致无法解释和处理该消息;再例如他们的认证性的实现是通过在 ADS-B 消息后面附加一个认证码的方式来为 ADS-B 消息提供认证性,但是他们没有考虑到 ADS-B 数据链路的带宽是一定的,只有112bits,因此,多出来的认证码部分将无法随原 ADS-B 消息一起发送出去。因此,本文将结合 ADS-B 消息的实际使用情况来解决 ADS-B 消息的机密性问题。

与 ADS-B 安全性有关的工作的数量正在逐渐增加直到它在美国和其他欧洲国家被强制性使用,而且当时间越紧迫问题就会变得越迫切。当已经完成的工作属于安全必须和属于不同方面,尤其是多侧面和数据融合提供内在深层次的含义。目前的工作寻求去拓宽通信社区的理解和在一个更深更广的层面上强调这个问题。现在我们需要做的事是去整合目前考虑到的所有方面,在网络安全领域寻求一些可行的,有远见的想法,然后权衡利弊和检验应对他们的方法。

\section{课题难点、重点、核心问题及方向}
\subsection{课题的重点和难点}
课题的重点和难点都是要对 ADS-B 系统的运行以及对 ADS-B 消息的格式有很深入的认识,再针对 ADS-B 系统及网络传输的特点以及 ADS-B 的消息格式及含义选出适合且切合实际的解决方案。本课题要求对     ADS-B 数据链的隐私进行保护,即保护 ADS-B 消息的机密性,保护消息的机密性无外乎加密算法的选择问题。第二章会讨论到各种加密算法的异同以及把他们应用在 ADS-B 上的优势以及瓶颈。
\subsection{课题的核心问题及方向}
课题的核心是找到一种切合 ADS-B 应用实际的 ADS-B 数据链加解密算法并对该算法进行代码实现。然后进一步通过对消息进行解码和图像化显示来对用该加密算法加密后的 ADS-B 消息的实验结果进行验证和比较。
